\subsection{El reparto}
\subsubsection{Protocolo de reparto de cartas}
El siguiente es el protocolo que da como resultado tres cartas para cada jugador elegidas de forma azaroza.

\begin{enumerate}

\item B le pide conexi'on a A

\begin{verbatim}

------------
|  "INIT"  |
------------
4 bytes

\end{verbatim}




\item A genera una clave sim'etrica $k$ (por ejemplo, con AES), encripta las 40 cartas $M_1...M_40$ con $k$ y las env'ia a B ($k$ sirve para asegurar que el mazo enviado por A en este paso es v'alido).

\begin{verbatim}

------------------------------------------------
|   k(M1)   |   k(M2)   |   ...   |   k(M40)   |
------------------------------------------------
  160 bits    160 bits               160 bits

\end{verbatim}

  


\item B genera un $p$ primo grande y genera $e^1_b$, $d^1_b$ (utilizadas para asegurar una repartici'on justa) y  $e^2_b$, $d^2_b$ (utilizadas para asegurar que se jueguen las cartas tocadas) tal que

$$	e^1_b * d^1_b = 1 [mod\ p-1] = e^2_b * d^2_b $$

Env'ia a A el $p$ y las cartas ya encriptadas con $k$ encriptadas a su vez con $e^1_b$ (usando RSA)

$$	e^1_b(k(M_i)) = k(M_i)^{e^1_b} $$
	

\begin{verbatim}
-----------------------------------------------------------------------
|   p   |   e1b(k(M1))   |   e1b(k(M2))   |   ...   |   e1b(k(M40))   |
-----------------------------------------------------------------------
  1024 b     1024 b           1024 b         1024 b       1024 b       
\end{verbatim}


  


\item A usa $p$ para generarse sus propias claves

$$	e^1_a * d^1_a = 1 [mod\ p-1] = e^2_a * d^2_a $$

Elige al azar 3 cartas de las enviadas por B ($B_1, B_2, B_3$) y las firma con $e^2_a$. Env'ia cada carta como una tupla

$$	<e^1_b(k(Bi)), 			e^2_a(e^1_b(k(Bi)))> = 
	<k(Bi)^{e^1_b} [mod\ p], 	k(Bi)^(e^1_b * e^2_a) [mod\ p]> $$
	
A su vez, repite el paso anterior realizado por B enviandole a este el resto de las cartas (Ri) encriptadas con su clave:

$$	e^1_a(e^1_b(k(Ri))) = k(Ri)^(e^1_b * e^1_a) [mod\ p] $$
	

\begin{verbatim}
--------------------------------------------------------------------------
|   e1b(k(B1))   |   e1b(k(B2))   |   e1b(k(B3))   |   e2a(e1b(k(B1)))   |
--------------------------------------------------------------------------
     1024 b           1024 b           1024 b              1024 b         

--------------------------------------------
   e2a(e1b(k(B2)))   |   e2a(e1b(k(B3)))   |
--------------------------------------------
       1024 b                1024 b         

------------------------------------------------------------------------------
|    e1a(e1b(k(R1)))   |   e1a(e1b(k(R2)))   |   ...   |   e1a(e1b(k(R37)))  |
------------------------------------------------------------------------------
        1024 b                1024 b                         1024 b           
\end{verbatim}
	
	
	
	
	
\item B recibe sus cartas (las tuplas) y les aplica la desencripcion de $e^1_b$ con $d^1_b$:

$$	<d^1_b(k(B_i)^{e^1_b} [mod\ p]), 		d^1_b(k(B_i)^{e^1_b * e^2_a} [mod\ p])> = $$
$$	<k(B_i)^{e^1_b * d^1_b} [mod\ p], 	k(B_i)^{e^1_b * e^2_a * d^1_b} [mod\ p]> = $$
$$	<k(B_i), 						k(B_i)^{e^2_a} [mod\ p]> $$

A su vez, elige 3 cartas al azar ($A_1, A_2, A_3$) del resto (las $R_i$) y les aplica tambien $d^1_b$:
	
$$	d^1_b(k(A_i)^{e^1_b * e^1_a} [mod\ p]) = 
	k(A_i)^{e^1_b * e^1_a * d^1_b} [mod\ p] =
	k(A_i)^{e^1_a} [mod\ p] $$

Para completar la mano de A, debe completar las tuplas con su firma:

$$	e^2_b(k(A_i)^{e^1_a} [mod\ p]) =
	k(A_i)^{e^1_a * e^2_b} [mod\ p] $$
	
y se las env'ia a A:

$$	<k(A_i)^{e^1_a} [mod\ p],			k(A_i)^{e^1_a * e^2_b} [mod\ p]> $$
	

\begin{verbatim}
--------------------------------------------------------------------------
|   e1a(k(A1))   |   e1a(k(A2))   |   e1a(k(A3))   |   e2b(e1a(k(A1)))   |
--------------------------------------------------------------------------
     1024 b           1024 b           1024 b              1024 b         

--------------------------------------------
   e2b(e1a(k(A2)))   |   e2b(e1a(k(A3)))   |
--------------------------------------------
       1024 b                1024 b         
\end{verbatim}
	
	
	
	
	
\item A recibe sus cartas y les aplica la desencripci'on de $e^1_a$ con $d^1_a$:

$$	<d^1_a(k(A_i)^{e^1_a} [mod\ p]),			d^1_a(k(A_i)^{e^1_a * e^2_b} [mod\ p])> = $$
$$	<k(A_i)^{e^1_a * d^1_a} [mod\ p],			k(A_i)^{e^1_a * e^2_b * d^1_a} [mod\ p]> = $$
$$	<k(A_i),								k(A_i)^{e^2_b} [mod\ p]> $$
	
A utiliza $k$ para ver las cartas que le tocaron, su mano queda

$$	<A_i, k(A_i)^{e^2_b} [mod\ p]> $$

Por ultimo, envia $k$ a B


\begin{verbatim}
---------
|   k   |
---------
  128 b
\end{verbatim}

	 
	 
	 


\item B recibe $k$, con lo que la utiliza para ver los $M_i$ que le mando A en el paso 2 (poseia $k(M_i)$) y verificar que le mand'o un mazo valido. Tambi'en desencripta su mano para ver las cartas que le tocaron:

$$	<B_i, k(B_i)^{e^2_a} [mod\ p]> $$
	
\end{enumerate}	




\subsubsection{Pre-inicio del juego}
Este es final del inicio del juego. Se setean datos para ser usados durante el resto de la mano.

\begin{enumerate}
\item Ahora que ambos tienen sus manos, B se genera un par de claves RSA comunes y corrientes 

$$	(e^3_b, n_b), (d^3_b, n_b)  $$
	
y envia la publica $(e^3_b, n_b)$ a A. Usar'a $d^3_b$ s'olo para firmar sus acciones.

A su vez, genera un paquete con el timestamp inicial decretando esta como la hora de inicio de juego. Lo env'ia firmado con $d^3_b$.

\begin{verbatim}
------------------------------------
|   e3b   |   nb   |   d3b(ts_i)   |
------------------------------------
  2048 b    2048 b      2048 b
\end{verbatim}
  
  
  


\item A lee con la clave p'ublica de B lo firmado por 'el y verifica que el timestamp es v'alido. Se genera tambi'en sus pares de claves RSA

$$	(e^3_a, n_a), (d^3_a, n_a) $$
	
y env'ia su aceptaci'on (firmada con $d^3_a$) de la hora de inicio.


\begin{verbatim}
--------------------
|   e3a   |   na   |
--------------------
  2048 b    2048 b  
\end{verbatim}
  
  


\item Al recibir B la verificaci'on, puede empezar a jugar (notar que es mano).

\end{enumerate}
