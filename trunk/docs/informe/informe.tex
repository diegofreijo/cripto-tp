\documentclass[spanish, a4paper, 11pt]{article}

\usepackage[a4paper,margin=3.5cm,top=3.0cm,bottom=3.0cm]{geometry}	% Define los margenes
\usepackage[spanish,activeacute]{babel}										% Idioma castellano
\usepackage{sty/caratula}															% Caratula de Algo2
%\usepackage[a4paper=true,pagebackref=true]{hyperref}						% Agrega la TOC al PDF e hipervinculos
\usepackage[pdftex]{graphicx} 													% Permite insertar graficos
\usepackage{fancyhdr}																% Permite manejo de cabeceras de pagina
\usepackage{eufrak}																	% Usado en el enunciado del trabajo
\usepackage{latexsym}
\usepackage{graphicx}

% Estilo de pagina para tener las cabeceras
\pagestyle{fancy}
\lhead{Truco Mental}
\rhead{Albanesi - Carbajo - Freijo - Venturini}

% Numeracion de paginas
\pagenumbering{arabic}
\parskip=1.5ex

\newcommand{\imagen}[3]
{
	\begin{figure}[htbp]
	  \centering
	    \includegraphics[scale=0.5]{#1}
	  \caption{#3}
	\end{figure}
}

\newcommand{\nat}{\mathds{N}}
\newcommand{\algoritmo}[3]{\noindent {\bf\underline{#1}:} #2 $\longrightarrow$ #3}
\newcommand{\superindice}[1]{$^\textrm{{\tiny #1}}$}
\newcommand{\subsubsubsection}[1]{

{\bf\small #1}

}
\newcommand{\negrita}[1]{{\bf #1}}

\renewcommand\floatpagefraction{.9}
\renewcommand\topfraction{.9}
\renewcommand\bottomfraction{.9}
\renewcommand\textfraction{.1}   
\setcounter{totalnumber}{50}
\setcounter{topnumber}{50}
\setcounter{bottomnumber}{50}

%%%%%%%%%%%%%%%%%%%%%%%%%%%%%%%%%%%%%%%%%%%%%%%%%%%%%%%%%%%%%%
%%%%%%%%%%%%%%%%%%%%%%%%%%%%%%%%%%%%%%%%%%%%%%%%%%%%%%%%%%%%%%%%%%%%%%%%%%%%%%%%%%%%
%%%%%   Inicio del documento
%%%%%%%%%%%%%%%%%%%%%%%%%%%%%%%%%%%%%%%%%%%%%%%%%%%%%%%%%%%%%%%%%%%%%%%%%%%%%%%%%%%%
%%%%%%%%%%%%%%%%%%%%%%%%%%%%%%%%%%%%%%%%%%%%%%%%%%%%%%%%%%%%%%

\begin{document}

%%%%%%%%%%%%%%%%%%%%%%%%%%%%%%%%%%%%%%%%%%%%%%%%%%%%%%%%%%%%%%%%%%%%%%
% Caratula
%%%%%%%%%%%%%%%%%%%%%%%%%%%%%%%%%%%%%%%%%%%%%%%%%%%%%%%%%%%%%%%%%%%%%%
\materia{Criptograf'ia}
\submateria{Segundo Cuatrimestre de 2007}
\titulo{Trabajo Pr'actico}
\subtitulo{Truco Mental}
\integrante{Albanesi, Mat'ias}{??/??}{@}
\integrante{Carbajo, Pablo}{717/04}{pcarbajo@dc.uba.ar}
\integrante{Freijo, Diego}{4/05}{giga.freijo@gmail.com}
\integrante{Venturini, Maura}{-}{venturinimaura@gmail.com}
\maketitle

\subsection*{Abstracto}
En el presente trabajo se dise'no e implement'o un protocolo para jugar al truco mediante el cual se asegura el cumplimiento de las reglas de juego.


\subsection*{Palabras Clave}
Encriptaci'on, algoritmos de clave p'ublica/privada y sim'etricos, RSA, AES.


%%%%%%%%%%%%%%%%%%%%%%%%%%%%%%%%%%%%%%%%%%%%%%%%%%%%%%%%%%%%%%%%%%%%%%
% Indice
%%%%%%%%%%%%%%%%%%%%%%%%%%%%%%%%%%%%%%%%%%%%%%%%%%%%%%%%%%%%%%%%%%%%%%
\clearpage
\tableofcontents

%%%%%%%%%%%%%%%%%%%%%%%%%%%%%%%%%%%%%%%%%%%%%%%%%%%%%%%%%%%%%%%%%%%%%%
% Introduccion
%%%%%%%%%%%%%%%%%%%%%%%%%%%%%%%%%%%%%%%%%%%%%%%%%%%%%%%%%%%%%%%%%%%%%%
\clearpage
\section{Introducci'on}
La motivaci'on del trabajo surge de la necesidad que imponen ciertos juegos de cartas de poder repartir los naipes entre distintos jugadores en forma azarosa (justa), y que garantice de alguna manera que ninguno de los jugadores haga trampa en lo que al reparto y juego de las cartas se refiere. Lograr esto en una implementaci'on por software de juegos de este tipo requiere obligatoriamente de t'ecnicas criptogr'aficas, ya sea de firma y/o encriptaci'on de mensajes. Dentro de estos juegos, el Truco tiene la particularidad de que puede ser necesario que un jugador muestre ninguna, alguna, o todas las cartas que recibi'o.

El objetivo del trabajo es dise'nar e implementar un protocolo para el reparto de cartas y juego de Truco Mental entre dos jugadores.
El protocolo est'a dise'nado para garantizar una justa repartici'on de cartas y que cada parte juege s'olo cartas que le fueron repartidas. Adem'as, el protocolo permite que un jugador muestre sus cartas selectivamente, y que verifique que las cartas pertenecen a un mazo inalterado.

El lenguaje elegido es Python. La implementaci'on realizada permite que dos jugadores juegen una mano a trav'es de una conexi'on por red. Al ejecutar el programa, 'este pregunta si debe ser ejecutado en modo cliente o modo servidor. Si se ejecutaran varias instancias del servidor en la misma computadora, se podr'ia dar un servicio de truco a varios jugadores simult'anea e independientemente.


%%%%%%%%%%%%%%%%%%%%%%%%%%%%%%%%%%%%%%%%%%%%%%%%%%%%%%%%%%%%%%%%%%%%%%
% El juego del truco
%%%%%%%%%%%%%%%%%%%%%%%%%%%%%%%%%%%%%%%%%%%%%%%%%%%%%%%%%%%%%%%%%%%%%%
\clearpage
\section{El juego del truco}
\subsection{Reglas de juego}

\subsection{L'ogica de juego}

1- Manejo de las Cartas
=======================

Escala:
0 01E
1 01B
2 07E
3 07O
4 03O,03C,03B,03E
5 02O,02C,02B,02E
6 01O,01C
7 12O,12C,12B,12E
8 11O,11C,11B,11E
9 10O,10c,10B,10E
10  07B,07C
11  06O,06C,06B,06E
12  05O,05C,05B,05E
13  04O,04C,04B,04E

  Se define un orden parcial entre las cartas, que indica el valor relativo entre ellas. Esto nos va a servir para despues comparar, en base a las cartas en la mesa, qui'en "mat'o", y a qui'en le corresponde jugar.

  * Para ver qui'en gana la mano, buscamos las dos cartas que est'an sobre la mesa y vemos cu'al tiene el nivel m'as alto (esa es la que gana la mano). En caso de estar en niveles iguales hay parda y le toca jugar al jugador que es mano.
carta (la mano).


2- Manejo de Cantos
===================

Envido
------
  * Se puede cantar Envido hasta antes de jugar la primera mano
  * Se puede seguir contestando hasta que se diga "Quiero"; ac'a se procede a ver qu'e cartas son del mismo palo y se suman los valores (en caso de tenes flor se suman los dos m'as altos), o "No Quiero"; no cuento nada y sigue el juego.
  * Si se cant'o "Truco" no se puede cantar "Envido"
  * Si ya se dijo "Quiero" o "No Quiero", no se puede seguir cantando.

      ----------                       ----------                       ---------------
      | Envido |---------------------> | Envido |---------------------> | Real Envido |
      ----------                       ----------                       ---------------




      -------------          ----------     ----------------
      | No Quiero |          | Quiero |     | Falta Envido |
      -------------          ----------     ----------------

Truco
-----
  * Se puede cantar en cualquier mano e inclusive antes de jugar la primera carta.
  * El ganador de 2 de 3 manos es el que gana la mano.
  * Una vez que los 2 jugaron las 3 cartas no se puede cantar "Truco".

      ---------                       ------------                       ---------------
      | Truco |---------------------> | Re Truco |---------------------> | Vale Cuatro |
      ---------                       ------------                       ---------------



        -------------          ----------
        | No Quiero |          | Quiero |
        -------------          ----------

  Los arcos se dibujan a mano (es mas facil que aca) y contienen los puntajes que se suman en el Score.
* Los cantos deben manejarse como interrupciones. Un jugador puede cantar "Truco" habiendo jugado su carta o no. No es asi el caso del "Envido"; aca hay que controlar que el jugador tenga el token (le toque jugar) y no haya tirado su carta.


3- Registros a Tener en Cuenta:
==============================

  * Cantidad de Manos Ganadas
  * Si se canto "Envido"
  * Si se canto "Truco"
  * Al finalizar la mano:
    - Si se canto "Envido" me fijo si en las cartas sobre la mesa estan los puntos del "Envido". Si no, muestro las que faltan aunque se haya ido al maso.
    - Si se fue al maso (o se canta "Truco" y no se quiere), entonces no se deben mostrar las cartas que no se hayan jugado a menos que ocurra el caso anterior.


4- Decisiones de Implementaci'on:
===============================

** Se van a necesitar las siguientes variables para saber en que estado estamos dentro de Envido durante la
partida:

** Envido:
   ------
  Una variable EstadoEnvido que va a asumir los valores:
    0, si no se canto nada
    1, si se canto Envido
    2, si se canto Envido Envido
    3, si se canto Real Envido
    4, si se canto Falta Envido
    5, si ya no se puede seguir cantando (si el jugador respondio "Quiero" o " No Quiero")
  Una variable que almacene la cantidad de puntos en juego si se dice Quiero (PtosQuieroEnv)
  Una variable que almacene la cantidad de puntos en juego si se dice No Quiero (PtosNQuieroEnv)
  Una variable que me indique si puedo redoblar la apuesta del Envido (Es decir, si tengo el Quiero, PuedoCantarEnv)

Nota:
  La cantidad de Puntos en juego si se dice "No Quiero" es la cantidad de cantos que se hayan efectuado y la cantidad
de puntos en juego si se dice "Quiero" es sumar 2 si se canta Envido, 3 si se canta Real Envido. Se tomo esta decision ya quela cantidad de puntos en juego si se canta Envido, Envido, Real Envido no es la misma que si se canta Envido, Real Envido.

  Supongamos que la variable PtosQuieroEnv no existe y que manejamos el Envido con las otras dos variables. En el primer caso, la variable PtosNQuieroEnv tendria un valor 3 y en el segundo caso 2. La variable EstadoEnvido en el primer caso, pasaria del estado 0 al 1, luego al 2 y finalmente el 3; y en el segundo, pasaria del estado 0 al 1 y finalmente al 3. Ahora, cuando computemos los puntos en juego, suponiendo que se dijo "Quiero" a partir de EstadoEnv, en los dos casos terminaria con el mismo valor!, lo cual es incorrecto. Es por esto que se opto por tener un registro de las estapas del Envido en la que se encuentra (EstadoEnv), la cantidad de puntos en juego si se sice "Quiero" (esto soluciona el problema antes mencionado), y la cantidad de puntos en juego si se dice "No Quiero" que sirve para contar la cantidad de cantos que se efectuaron.

** Truco:
   -----
  Una variable EstadoTruco que va a asumir los siguientes valores:
    0, si no se canto nada
    1, si se canto Truco
    2, si se canto Re Truco
    3, si se canto Vale Cuatro
    4, si ya no se puede seguir cantando (si el jugador respondio "Quiero" o "No Quiero")
  Una variable que cuente los puntos del truco en caso que la respuesta sea "Quiero" (PtosQuieroTru)
  Una variable que cuente los puntos del truco en caso que la respuesta sea "No Quiero" (PtosNQuieroTru)
  Una variable que me indique si puedo redoblar la apuesta del Truco (Es decir, si tengo el Quiero, PuedoCantarTru)

Nota:
  Tanto los puntos del truco querido como los del no querido se van incrementando de a uno. La diferencia esta en que PtosQuieroTru de 0 pasa al valor 2; en cambio, PtosNQuieroTru pasa de 0 a 1.




%%%%%%%%%%%%%%%%%%%%%%%%%%%%%%%%%%%%%%%%%%%%%%%%%%%%%%%%%%%%%%%%%%%%%%
% El protocolo
%%%%%%%%%%%%%%%%%%%%%%%%%%%%%%%%%%%%%%%%%%%%%%%%%%%%%%%%%%%%%%%%%%%%%%
\clearpage
\section{El protocolo}
Se define un procotolo que permite el reparto de cartas previo a una mano de truco y la posterior utilizaci'on de las mismas durante la mano. El dise'no del protocolo se hizo con los objetivos de garantizar la transparencia del juego, tanto en el reparto de las cartas como durante el desarrollo del juego.

El protocolo brinda cierta seguridad de que las cartas han sido repartidas azarosamente, y permite verificar que cada jugador haya recibido las cartas que decide jugar.

Durante la especificai'on del protocolo, se usar'a A para referirse al servidor, y B para referirse al cliente.

\subsection{El reparto}
\subsubsection{Protocolo de reparto de cartas}
El siguiente es el protocolo que da como resultado tres cartas para cada jugador elegidas de forma azaroza.

\imagen{img/ProtocoloReparto.png}{14}{Protocolo de reparto}

Notar que cuando se hace referencia a RSA no es estrictamente al algoritmo RSA, sino a la variante que utiliza un m'odulo primo descripta en Mental Poker, y que en ning'un momento hay intercambio de claves de algoritmos asim'etricos. La 'unica clave transmitida es la utilizada por AES.
En este caso "firmar" equivale a encriptar y enviar a la otra parte tanto el plaintext como el encriptado. En este esquema, el firmate es el 'unico que puede validar su firma, pero esto es suficiente si lo que se desea es validar que la carta jugada por el oponente sea una de las que uno mismo le reparti'o.

\begin{enumerate}

\item B le pide conexi'on a A

\begin{verbatim}

------------
|  "INIT"  |
------------
4 bytes

\end{verbatim}




\item A genera una clave sim'etrica $k$ (por ejemplo, con AES), encripta las 40 cartas $M_1...M_40$ con $k$ y las env'ia a B ($k$ sirve para asegurar que el mazo enviado por A en este paso es v'alido).

\begin{verbatim}

------------------------------------------------
|   k(M1)   |   k(M2)   |   ...   |   k(M40)   |
------------------------------------------------
  160 bits    160 bits               160 bits

\end{verbatim}

  


\item B genera un $p$ primo grande y genera $e^1_b$, $d^1_b$ (utilizadas para asegurar un reparto justo) y  $e^2_b$, $d^2_b$ (utilizadas para asegurar que se jueguen las cartas repartidas) tal que

$$	e^1_b * d^1_b = 1 [mod\ p-1] = e^2_b * d^2_b $$

Env'ia a A el $p$ y las cartas ya encriptadas con $k$ encriptadas a su vez con $e^1_b$ (usando RSA)

$$	e^1_b(k(M_i)) = k(M_i)^{e^1_b} $$
	

\begin{verbatim}
-----------------------------------------------------------------------
|   p   |   e1b(k(M1))   |   e1b(k(M2))   |   ...   |   e1b(k(M40))   |
-----------------------------------------------------------------------
  1024 b     1024 b           1024 b         1024 b       1024 b       
\end{verbatim}


  


\item A usa $p$ para generarse sus propias claves

$$	e^1_a * d^1_a = 1 [mod\ p-1] = e^2_a * d^2_a $$

Elige al azar 3 cartas de las enviadas por B ($B_1, B_2, B_3$) y las firma con $e^2_a$. Env'ia cada carta como una tupla

$$	<e^1_b(k(Bi)), 			e^2_a(e^1_b(k(Bi)))> = 
	<k(Bi)^{e^1_b} [mod\ p], 	k(Bi)^(e^1_b * e^2_a) [mod\ p]> $$
	
A su vez, repite el paso anterior realizado por B envi'andole a este el resto de las cartas (Ri) encriptadas con su clave:

$$	e^1_a(e^1_b(k(Ri))) = k(Ri)^(e^1_b * e^1_a) [mod\ p] $$
	

\begin{verbatim}
--------------------------------------------------------------------------
|   e1b(k(B1))   |   e1b(k(B2))   |   e1b(k(B3))   |   e2a(e1b(k(B1)))   |
--------------------------------------------------------------------------
     1024 b           1024 b           1024 b              1024 b         

--------------------------------------------
   e2a(e1b(k(B2)))   |   e2a(e1b(k(B3)))   |
--------------------------------------------
       1024 b                1024 b         

------------------------------------------------------------------------------
|    e1a(e1b(k(R1)))   |   e1a(e1b(k(R2)))   |   ...   |   e1a(e1b(k(R37)))  |
------------------------------------------------------------------------------
        1024 b                1024 b                         1024 b           
\end{verbatim}
	
	
	
	
	
\item B recibe sus cartas (las tuplas) y les aplica la desencripci'on de $e^1_b$ con $d^1_b$:

$$	<d^1_b(k(B_i)^{e^1_b} [mod\ p]), 		d^1_b(k(B_i)^{e^1_b * e^2_a} [mod\ p])> = $$
$$	<k(B_i)^{e^1_b * d^1_b} [mod\ p], 	k(B_i)^{e^1_b * e^2_a * d^1_b} [mod\ p]> = $$
$$	<k(B_i), 						k(B_i)^{e^2_a} [mod\ p]> $$

A su vez, elige 3 cartas al azar ($A_1, A_2, A_3$) del resto (las $R_i$) y les aplica tambi'en $d^1_b$:
	
$$	d^1_b(k(A_i)^{e^1_b * e^1_a} [mod\ p]) = 
	k(A_i)^{e^1_b * e^1_a * d^1_b} [mod\ p] =
	k(A_i)^{e^1_a} [mod\ p] $$

Para completar la mano de A, debe completar las tuplas con su firma:

$$	e^2_b(k(A_i)^{e^1_a} [mod\ p]) =
	k(A_i)^{e^1_a * e^2_b} [mod\ p] $$
	
y se las env'ia a A:

$$	<k(A_i)^{e^1_a} [mod\ p],			k(A_i)^{e^1_a * e^2_b} [mod\ p]> $$
	

\begin{verbatim}
--------------------------------------------------------------------------
|   e1a(k(A1))   |   e1a(k(A2))   |   e1a(k(A3))   |   e2b(e1a(k(A1)))   |
--------------------------------------------------------------------------
     1024 b           1024 b           1024 b              1024 b         

--------------------------------------------
   e2b(e1a(k(A2)))   |   e2b(e1a(k(A3)))   |
--------------------------------------------
       1024 b                1024 b         
\end{verbatim}
	
	
	
	
	
\item A recibe sus cartas y les aplica la desencripci'on de $e^1_a$ con $d^1_a$:

$$	<d^1_a(k(A_i)^{e^1_a} [mod\ p]),			d^1_a(k(A_i)^{e^1_a * e^2_b} [mod\ p])> = $$
$$	<k(A_i)^{e^1_a * d^1_a} [mod\ p],			k(A_i)^{e^1_a * e^2_b * d^1_a} [mod\ p]> = $$
$$	<k(A_i),								k(A_i)^{e^2_b} [mod\ p]> $$
	
A utiliza $k$ para ver las cartas que le tocaron, su mano queda

$$	<A_i, k(A_i)^{e^2_b} [mod\ p]> $$

Por 'ultimo, envia $k$ a B


\begin{verbatim}
---------
|   k   |
---------
  128 b
\end{verbatim}

	 
	 
	 


\item B recibe $k$, con lo que la utiliza para ver los $M_i$ que le mando A en el paso 2 (poseia $k(M_i)$) y verificar que le mand'o un mazo valido. Tambi'en desencripta su mano para ver las cartas que le tocaron:

$$	<B_i, k(B_i)^{e^2_a} [mod\ p]> $$
	
\end{enumerate}	




\subsubsection{Pre-inicio del juego}
Este es final del inicio del juego. Se setean datos para ser usados durante el resto de la mano.

\begin{enumerate}
\item Ahora que ambos tienen sus manos, B se genera un par de claves RSA comunes y corrientes 

$$	(e^3_b, n_b), (d^3_b, n_b)  $$
	
y env'ia la p'ublica $(e^3_b, n_b)$ a A. Usar'a $d^3_b$ s'olo para firmar sus acciones.

A su vez, genera un paquete con el timestamp inicial decretando esta como la hora de inicio de juego. Lo env'ia firmado con $d^3_b$.

\begin{verbatim}
------------------------------------
|   e3b   |   nb   |   d3b(ts_i)   |
------------------------------------
  2048 b    2048 b      2048 b
\end{verbatim}
  
  
  


\item A lee con la clave p'ublica de B lo firmado por 'el y verifica que el timestamp es v'alido. Se genera tambi'en sus pares de claves RSA

$$	(e^3_a, n_a), (d^3_a, n_a) $$
	
y env'ia su aceptaci'on (firmada con $d^3_a$) de la hora de inicio.


\begin{verbatim}
--------------------
|   e3a   |   na   |
--------------------
  2048 b    2048 b  
\end{verbatim}
  
  


\item Al recibir B la verificaci'on, puede empezar a jugar (notar que es mano).

\end{enumerate}

\subsection{Transcurso del juego}
Durante el transcurso del juego se deber'an cumplir las siguientes reglas:

\begin{itemize}
\item Es mano el que pidi'o conexion.

\item Supongo que B quiere jugar $B_2$. Entonces debe enviar la firma que posee:

$$	k(B_2)^{e^2_a} [mod\ p] $$
	
Cuando A lo recibe, le aplica su desencripci'on:

$$	d^2_a(k(B_2)^{e^2_a} [mod\ p]) = 
	k(B_2)^{e^2_a * d^2_a} [mod\ p] = 
	k(B_2) $$

Y con aplicar $k$, obtiene $B_2$.

\item Cada vez que se canta o se juega una carta, se debe adjuntar un timestamp del momento del canto/juego. Luego, el paquete debe ser firmado (con $d^3_a$ / $d^3_b$ seg'un corresponda) para evitar el repudio del emisor m'as adelante ("...no no, yo no te cante, entendiste mal...") y la reutilizaci'on del canto como prueba falsa m'as adelante ("...s'i s'i, vos cantaste truco; hace 5 horas, pero cantaste, mir'a...").

\item El que finalize el juego (el que se vaya al mazo, el que mate la 'ultima carta del contrincante, el que gane en el falta envido) debe adem'as enviar un timestamp firmado, marc'andolo como el final oficial del juego. El oponente, aunque est'e caliente, debe confirmarle si la hora es v'alida.

\end{itemize}




%%%%%%%%%%%%%%%%%%%%%%%%%%%%%%%%%%%%%%%%%%%%%%%%%%%%%%%%%%%%%%%%%%%%%%
% Consideraciones
%%%%%%%%%%%%%%%%%%%%%%%%%%%%%%%%%%%%%%%%%%%%%%%%%%%%%%%%%%%%%%%%%%%%%%
\clearpage
\section{Consideraciones}
Se decidi'o dejar los aspectos asincr'onicos del juego del truco fuera del alcance del trabajo, ya que el foco est'a en la parte criptogr'afica del mismo.
Por lo tanto, se implement'o una simplificaci'on del juego en la cual los turnos de juego est'an bien definidos, y cada jugador debe esperar su turno para poder jugar una carta o realizar alg'un canto. Esta simplificaci'on no trae grandes inconvenientes a la hora de jugar, pero simplifica notablemente la implementaci'on del juego, ya que permite prescindir del uso de threads, evitando cualquier posible condici'on de carrera (por ejemplo si un jugador canta \"truco\" pero el otro tambi�n cantra "truco" antes de recibir el canto del primero).
De esta manera, el jugador que debe bajar una carta es el que tiene el turno. Si en lugar de jugar una carta, decide hacer un canto, cede su turno moment'aneamente y s'olo hasta que el contrincante conteste el canto. Una vez realizados los intercambios de turno requeridos por el canto realizado, el turno vuelve al jugador que deb'ia bajar una carta.

%%%%%%%%%%%%%%%%%%%%%%%%%%%%%%%%%%%%%%%%%%%%%%%%%%%%%%%%%%%%%%%%%%%%%%
% Manual de juego
%%%%%%%%%%%%%%%%%%%%%%%%%%%%%%%%%%%%%%%%%%%%%%%%%%%%%%%%%%%%%%%%%%%%%%
\clearpage
\section{Manual de uso del software implementado}
Al ejecutar el programa, se pide que el usuario indique si se usar'a el software en modo cliente o en modo servidor.
elegir IP
elegir port tcp
jugar

%%%%%%%%%%%%%%%%%%%%%%%%%%%%%%%%%%%%%%%%%%%%%%%%%%%%%%%%%%%%%%%%%%%%%%
% Bibliograf'ia
%%%%%%%%%%%%%%%%%%%%%%%%%%%%%%%%%%%%%%%%%%%%%%%%%%%%%%%%%%%%%%%%%%%%%%
\clearpage
\begin{verbatim}
Adi Shamir, Ronald L. Rivest, Leonard M. Adleman. Mental Poker. David A. Klarner Ed. The Mathematical Gardner, Prindle, Weber and Smith, Boston, Massachusetts, 1981.

http://usuarios.arnet.com.ar/leo890/truco.htm
\end{verbatim}



%%%%%%%%%%%%%%%%%%%%%%%%%%%%%%%%%%%%%%%%%%%%%%%%%%%%%%%%%%%%%%%%%%%%%%
% EOF
%%%%%%%%%%%%%%%%%%%%%%%%%%%%%%%%%%%%%%%%%%%%%%%%%%%%%%%%%%%%%%%%%%%%%%
\end{document}
