\parskip = 11pt
\parindent = 0pt

\begin{centering}
\bf Laboratorio de M\'etodos Num\'ericos - Primer Cuatrimestre 2007 \\
\bf Trabajo Pr\'actico N\'umero 4: Luigi s\'olo admira a las Ferraris \\
\end{centering}

\vskip 1 cm
\hrule
\vskip 0.5 cm

\textbf{Introducci\'on}

Nos encontramos en el grupo de an\'alisis num\'erico de uno de los equipos de
F\'ormula 1, y estamos a pocas semanas de un gran premio que se realizar\'a
en un aut\'odromo a\'un no visitado por la categor\'\i a. Nos enfrentamos
al serio problema de decidir la mejor forma de tomar las curvas del nuevo
circuito, y para esto solamente podemos realizar simulaciones, dado que
el fin de semana de la carrera ser\'a la primera vez que nuestros autos
tomen contacto con esta pista.

Consideremos una curva del circuito, como la que muestra la Figura 1.
Queremos que nuestro auto comience en alg\'un punto del segmento A con
una cierta velocidad inicial, y que llegue a alg\'un punto del segmento B en
buenas condiciones (es decir, a una buena velocidad final, con las cuatro
ruedas sobre el pavimento y con la carrocer\'\i a en su lugar). Nuestro
objetivo principal es que el auto pueda sortear la curva sin derrapar, dado
que a la velocidad a la que circula se pierde el control muy r\'apidamente
si el auto pierde adherencia.

\vskip 11pt

\begin{centering}

Figura 1: Ejemplo de una curva y los puntos que determinan la trayectoria. \\
\end{centering}

\vskip 11pt

Para esto, trazamos una serie de puntos por los que esperamos que pase
el auto a intervalos regulares de $\omega = 0.1$ sg. Es decir, en el
instante $t=0$ sg.~el auto estar\'a en $p_1$, en el instante
$t=0.1$ sg.~estar\'a en $p_2$, etc. Como los puntos corresponden a
instantes equiespaciados en el tiempo, entonces estar\'an m\'as separados
en los tramos en los que el auto circule a mayor velocidad, y estar\'an
m\'as juntos en los tramos en los que el auto disminuya su velocidad.

Para determinar la trayectoria
completa del auto, utilizamos un spline param\'etrico que interpole estos
puntos, con lo cual tendremos una funci\'on $T:\real\to\real^2$ que nos
indicar\'a expl\'\i citamente la posici\'on del auto en funci\'on del tiempo.
Es decir, en el instante $t$, el auto se encuentra en el punto $T(t)$. Notar
que en funci\'on de este spline tambi\'en podemos calcular el vector velocidad y
el vector aceleraci\'on en cada instante de la trayectoria.

\vfil \eject

\textbf{C\'omo decidir si el auto se sale de la trayectoria}

Sea $t$ un instante del recorrido, en el cual el auto se encuentra en el
punto $T(t)\in\real^2$. Llamemos $v(t)$ y $a(t)$ al m\'odulo del
vector velocidad y al m\'odulo del vector aceleraci\'on en el instante $t$,
respectivamente. En este instante, la fuerza de frenado que ejerce el auto
sobre el piso es $F(t) = m\: a(t)$, siendo $m$ la masa del auto. El auto
no patina en el instante $t$ si
\begin{equation}
F(t) \ \le\ (mg + v(t)^2s) \: \mu_0. \label{condicion}
\end{equation}
En esta f\'ormula, $g=9.81$ m/sg${}^2$ es la aceleraci\'on gravitatoria,
$s$ es el coeficiente de carga din\'amica (de modo tal que $v(t)^2 s\mu_0$ es
la fuerza que ejerce el aire sobre la superficie del veh\'\i culo en
direcci\'on normal a la superficie de la pista, representando la acci\'on
de los alerones), y $\mu_0$ es el coeficiente de rozamiento est\'atico
(que depende del compuesto de los neum\'aticos, de la composici\'on del
pavimento y de las temperaturas del piso y de los neum\'aticos). El auto
se mantiene dentro de la trayectoria sin derrapar si la condici\'on
(\ref{condicion}) se cumple en todos los puntos de la misma.

\textbf{Enunciado}

El objetivo del trabajo pr\'actico es implementar un programa que tome los
datos del auto y de la trayectoria desde un archivo, que calcule un
spline param\'etrico que interpole los puntos de la trayectoria y que
determine si en alg\'un punto del recorrido el auto pierde adherencia. En
caso positivo, el programa deber\'a informar el punto en el cual esto
sucede, y la velocidad y direcci\'on del auto en ese momento.

El programa deber\'a tomar los datos de entrada desde un archivo de texto
cuyo formato queda a criterio del grupo. Debe notarse que los datos de
entrada incluyen, adem\'as de los puntos de la trayectoria, los datos del
auto (masa, coeficiente de rozamiento, coeficiente de carga din\'amica, etc.)
y el intervalo $\omega$ entre puntos consecutivos de la trayectoria.

Se deber\'a utilizar un algoritmo eficiente para el c\'alculo del spline
param\'etrico. Se debe mencionar en el informe el tipo de spline que se usa
para la interpolaci\'on, comentando las diferentes opciones consideradas
y las razones que llevaron a seleccionar esta opci\'on.

El c\'alculo para determinar si el auto derrapa en alg\'un punto de la
trayectoria deber\'a estar basado en un m\'etodo num\'erico a elecci\'on del
grupo para buscar m\'aximos o m\'\i nimos de funciones de una variable.
Sugerimos consultar con los docentes antes de utilizar alg\'un m\'etodo
num\'erico que no se encuentre dentro de esta categor\'\i a.

Adem\'as de estos puntos obligatorios, los invitamos a implementar los
siguientes \'\i tems optativos que pueden ser de inter\'es para el
an\'alisis:
\begin{itemize}
\item Realizar gr\'aficos de la velocidad y la aceleraci\'on en funci\'on
del tiempo. Tambi\'en puede ser interesante graficar la fuerza $F(t)$ en
funci\'on del tiempo, para ver en qu\'e momento el giro se torna peligroso.
\item Mostrar una animaci\'on en pantalla de la trayectoria del auto, junto
con su vector velocidad y la fuerza de frenado, para ver c\'omo evolucionan
a medida que el auto recorre la pista.
\end{itemize}

\vfil \eject

\textbf{Experimentos obligatorios}

Utilizando el programa implementado, analizar la mejor forma de tomar la
curva mostrada en la Figura 2, sabiendo que $\mu_0 = 0.9389$, $s = 3.8851$
kg/m y $m = 512$ kg. Buscar la velocidad inicial que permite
tomar las curvas lo m\'as r\'apidamente posible. Analizar si conviene tomar
cada curva con un radio de giro peque\~no pero a baja velocidad, o si es mejor
describir una trayectoria m\'as abierta pero a mayor velocidad. Buscar en
qu\'e momento conviene frenar, y desde qu\'e punto ya es seguro acelerar.
El grupo que logre la trayectoria m\'as r\'apida obtendr\'a una botella
de champagne y el derecho a mojar a sus compa\~neros a modo de festejo.

\begin{centering}

Figura 2: Curva para los experimentos. \\
\end{centering}

\vskip 1.2 cm
\hrule
\vskip 0.2 cm

Fecha de entrega: Lunes 25 de Junio.
\end{document}
