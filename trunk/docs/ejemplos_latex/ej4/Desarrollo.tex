\section{Desarrollo}

\subsection{An\'alisis del problema}
	
		\subsubsection{Modelo matem\'atico}
		
		Como paso previo al dise\~no de una soluci\'on en computadora del problema, analicemos el problema y sus implicaciones.

		El problema a resolver consiste en analizar si una trayectoria de un autom\'ovil (puntos en el plano) puede completarse cumpliendo una restricci\'on (inecuaci\'on) que determina si el auto derrapa debido a una velocidad o giro excesivos.

		Este an\'alisis debe hacerse no s\'olo en los puntos dados sino en todos los puntos de la trayectoria; para esto debe plantearse un spline param\'etrico que interpole los puntos dados como dato para obtener una ecuaci\'on de la trayectoria que pueda utilizarse para controlar si se cumple la restricci\'on de no p\'erdida de adherencia.

		El intervalo de tiempo entre puntos es constante, y se lo denomina $\omega$.

		Se pide chequear si el auto derrapa en alg\'un punto mediante un m\'etodo de b\'usqueda de m\'aximos o m\'inimos de una variable.

		Por \'ultimo debe aplicarse el programa para analizar la mejor manera de recorrer un tramo de pista, o sea encontrar una serie de puntos dentro de la pista y un valor para $\omega$ tal que el auto no derrape, cuyos datos pueden encontrarse en el Enunciado.

\newpage

\subsection{An\'alisis previo al desarrollo en computadora}

		\subsubsection{Plataforma y compilador}

		La primer decisi\'on a tomar consiste en plataforma y compilador a utilizar. Como el desarrollo debe cumplir con el requisito de poder compilarse y ejecutarse con el software disponible en los laboratorios del Departamento de Computaci\'on, optamos por computadoras con procesadores compatibles con la arquitectura Intel x86, y el compilador g++ de la Free Software Foundation (disponible para Windows y Linux).

		\subsubsection{Entrada y salida}

		El formato es simple: un archivo de texto donde se especifica con una palabra clave el valor se define (por ejemplo, \textbf{omega 0.75}), uno por l\'inea, permitiendo comentarios estilo C++.
		Todos los valores son en punto flotante exceptuando \textbf{cantidad\_muestras} que es entero.
		Un ejemplo de archivo de entrada es el siguiente:

\begin{center}
	\begin{tabular}{|c|}
	\hline
Init\\
\\
masa 512\\
coeficiente\_carga 3.8851\\
coeficiente\_rozamiento 0.9389\\
omega 1.1\\
\\
cantidad\_muestras 10\\
\\
direccion\_inicial 1 -0.25\\
x,y 10 3\\
x,y 11 1\\
x,y 12 -2\\
x,y 13 -5\\
x,y 14 1\\
x,y 15 3\\
x,y 17 7\\
x,y 18 9\\
x,y 19 11\\
x,y 25 13\\
\\
End\\
	\hline
	\end{tabular}
\end{center}
		

\subsubsection{Implementaci\'on del spline param\'etrico}

		Se dise\~naron funciones (sin llegar a ser una clase C++) para calcular los coeficientes de un spline en una variable, para utilizar posteriormente dos Splines, uno para la coordenada X de la trayectoria y otro para la coordenada Y.

		La primera decisi\'on a tomar fue el tipo de frontera a utilizar en el spline: frontera libre o frontera sujeta. Optamos por utilizar una frontera sujeta en el extremo inicial de la trayectoria, ya que era de suponer que el veh\'iculo tomar\'ia la curva con una velocidad conocida, o este dato tambi\'en ser\'ia \' deseable manipularlo para analizar el \'optimo (adem\'as esto se pide por enunciado).
		Sin embargo, para el extremo final de la trayectoria optamos por una frontera libre. Esto es factible ya que el sistema de ecuaciones resultante de plantear las condiciones de este Spline es diagonal dominante y por lo tanto existe soluci\'on.

		Optamos por realizar un spline que reciba una tabla de pares ($t$, $f(t)$) en lugar de pedir que los valores de $t$ sean equidistantes, por generalidad y porque no representaba un costo alto de desarrollo.

		Tambi\'en se reciben los valores de $f_x'(t_0)$ y $f_y'(t_0)$ por par\'ametro. La funci\'on devuelve una tabla donde para cada valor de la variable $t_i$ se tienen cuatro valores correspondientes a los coeficientes del polinomio c\'ubico v\'alido para el segmento $[t_i, t_{i+1}]$.
		
		Se crearon funciones que dada la tabla y un valor cualquiera de $t$ se calcula el valor del polinomio correspondiente al segmento donde est\'a contenido $t$, y tambi\'en las derivadas primera, segunda y tercera.


