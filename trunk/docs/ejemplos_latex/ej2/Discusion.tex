\section{Discusi�n}

% PAUTAS: Se incluir� aqu� un an�lisis de los resultados obtenidos en la secci�n anterior (se
% analizara su validez, coherencia, etc). Deben analizarse como m�nimo los items pedidos en el
% enunciado. No es aceptable decir que "los resultados fueron los esperados", sin hacer clara 
% referencia a la te�ria a la cual se ajustan. Adem�s se deben mencionar los resultados
% interesantes y los casos "patol�gicos" encontrados.

Las pruebas con el ``algoritmo original'' se hicieron sobre la forma original del programa, que planteaba el sistema lineal de ecuaciones a resolver para un grafo de $n$ nodos y $m$ ejes con una matriz de $m$ columnas y $n+m-1$ filas.\\
El ``algoritmo optimizado'' tambi\'en asigna $m$ columnas, pero asigna $max(m, n+p)$ filas, donde $p$ es la cantidad de ejes en el backbone (como el backbone es lineal, se tienen $p+1$ nodos concentradores). Salvo casos particulares de grafos poco densos (Con centrales generadoras dispuestas en forma de estrella alrededor de las centrales concentradoras. Asi como sin conexiones como el caso de �rboles, ya que el backbone es lineal, o con muy pocas conexiones entre centrales generadoras. Grafos en los cuales la mayoria de los ejes son ejes pertenecientes al backbone.) donde la cantidad de ejes en el backbone m\'as la cantidad de nodos supere a la cantidad total de ejes, se tiene que $n+p \leq m$ y por lo tanto la matriz del sistema de ecuaciones es cuadrada.\\

Puede observarse claramente que el tiempo de ejecuci\'on del c\'alculo del flujo en la red es polinomial con grado $\geq 2$, lo cual es de esperar ya que conocemos que la eliminaci\'on gaussiana es $O(n^3)$ sobre un sistema de n inc\'ognitas y n ecuaciones.

Teniendo el tiempo de ejecuci\'on directamente relacionado con el tama\~no de la entrada, y siendo las entradas muy similares, nos sorprendi\'o que el n\'umero de condici\'on presentara ``oscilaciones'', aumentando y disminuyendo a medida que se utilizaban instancias m\'as grandes. Sin embargo este n\'umero depende de la complejidad del grafo, y el tama\~no y forma del backbone puede tener influencia en este valor.

Se hicieron algunas pruebas que no aparecen en este informe, como modificar estos grafos para asignar balances aleatoriamente a las Centrales Generadoras con balance cero, manteniendo la condici\'on que la suma de los balances se mantenga en $0$. En estas pruebas no se observ\'o ning\'un cambio en el n\'umero de condici\'on, lo cual era de esperarse ya que s\'olo cambia la matriz $b$ de valores y no la matriz $A$ de ecuaciones, y hubo un impacto despreciable en el tiempo de ejecuci\'on del algoritmo, que podr\'ia deberse simplemente a la carga adicional de tener que escribir \emph{strings} m\'as largos al \emph{log} del programa. Ya que no resultaron valiosos estos experimentos no se incluyeron los resultados en este informe.

Sin embargo probar el programa con balances asignados aleatoriamente sac\'o a la luz un problema subyacente de nuestra implementaci\'on de la eliminaci\'on gaussiana: en numerosos lugares se comparaba si un valor era exactamente cero (por ejemplo al buscar un elemento pivote para una columna), por ejemplo \emph{v == 0.0}. Pese a que utilizamos \textbf{long double} tanto en c\'alculos intermedios como en el almacenamiento de valores en la matriz, siempre sucede alg\'un error num\'erico al despejar y al comparar exactamente por cero se estaba agravando. El s\'intoma del problema se presentaba como una parada del programa al encontrar que el sistema de ecuaciones era incompatible, ya que se obten\'ia una fila en ceros en la matriz $A$ pero un valor muy cercano pero distinto de cero en la matriz de valores $b$.

Debido a este problema se cambiaron todas las comparaciones exactas por una comparaci\'on de cercan\'ia a cero. Tambi\'en se modific\'aron las sentencias relacionadas con despeje, comparando el valor despejado con cero: valores muy cercanos a cero ahora se cambian por el valor exacto 0.0.
