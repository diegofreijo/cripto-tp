\documentclass[11pt, a4paper]{article}
\usepackage{a4wide}
\usepackage{graphics}
\parindent = 0 pt
\parskip = 11 pt

\newcommand{\erf}{\hbox{erf}}
\newcommand{\real}{\hbox{\bf R}}

\begin{document}

\begin{centering}
\bf Laboratorio de M\'etodos Num\'ericos - Primer cuatrimestre 2007 \\
\bf Trabajo Pr\'actico N\'umero 2: Distribuci\'on el\'ectrica \\
\end{centering}

\vskip 25pt
\hrule
\vskip 11pt

Una \emph{red de Morley} de transporte de electricidad es un sistema
interconectado de \emph{centrales generadoras} y \emph{nodos concentradores},
unidos por l\'\i neas de alta tensi\'on. Por ejemplo, la Figura 1
muestra una red de Morley compuesta por 15 centrales generadoras
(c\'\i rculos blancos) y 4 nodos concentradores (c\'\i rculos negros). Las
l\'\i neas que conectan los nodos concentradores entre s\'\i\ se llaman
el \emph{backbone} de la red. Los nodos concentradores no realizan tareas
de generaci\'on de energ\'\i a, sino que se limitan a canalizar el flujo
el\'ectrico que proviene de las centrales generadoras. Este modelo debe
su nombre a la propuesta realizada por O.~Morley en 1907.

\begin{figure}[h]
\centering
\includegraphics{tp2.eps} \\
Figura 1: Ejemplo de una red de Morley.
\end{figure}

Los usuarios de la red el\'ectrica se encuentran conectados a la red
generadora m\'as cercana, que los abastece de energ\'\i a. Decimos que una
central generadora tiene \emph{balance positivo} si la energ\'\i a que
produce alcanza para abastecer a sus usuarios. La energ\'\i a sobrante es
entregada por la central a la red el\'ectrica, para abastecer a otras
centrales. Por el contrario, decimos que una central generadora tiene
\emph{balance negativo} si la energ\'\i a que genera no es suficiente para
cubrir la demanda de sus usuarios. En este caso, la central debe recibir
energ\'\i a desde otras centrales, a trav\'es de la red. Notar que las
centrales generadoras no s\'olo est\'an conectadas a los nodos
concentradores, sino que puede suceder que haya centrales generadoras
con conexiones directas entre s\'\i.

Sea $C=\{1,\dots,n\}$ el conjunto de centrales generadoras, y llamemos
$b_i\in\real$ al balance de la central $i\in C$ (notar que $b_i>0$ si la
central $i$ tiene balance positivo, y $b_i<0$ en caso contrario). El problema
de distribuci\'on el\'ectrica sobre la red consiste en determinar
qu\'e flujo de energ\'\i a el\'ectrica se debe enviar por cada l\'\i nea
de alta tensi\'on, de manera tal que:
\begin{itemize}
\item[(i)] Si $i\in C$ es una central con balance positivo, entonces el flujo
neto saliendo de la central $i$ (es decir, la suma de los flujos que salen
menos la suma de los flujos que entran a la central) debe ser igual a $b_i$.
\item[(ii)] Si $i\in C$ es una central con balance negativo, entonces el
flujo neto entrando a la central $i$ (es decir, la suma de los flujos que
entran menos la suma de los flujos que salen de la central) debe ser igual
a $-b_i$.
\item[(iii)] El flujo total que entra a un nodo concentrador debe ser igual
al flujo total que sale de ese nodo.
\item[(iv)] Todas las interconexiones del \emph{backbone} de la red deben
tener el mismo nivel de flujo.
\end{itemize}
Esta \'ultima condici\'on se conoce con el nombre de \emph{restricci\'on
de Morley}. Por \'ultimo, su\-po\-ne\-mos que no hay l\'\i mite al flujo que
puede transportar cada l\'\i nea de alta tensi\'on.

\textbf{Enunciado}

El objetivo del trabajo pr\'actico es implementar un programa que tome como
entrada las interconexiones de la red y el balance de cada central
generadora, y que determine si existe una distribuci\'on de energ\'\i a
el\'ectrica que cumpla las condiciones (i)-(iv). El programa implementado
debe tomar los datos de entrada de un archivo de texto, cuyo formato queda
a criterio del grupo.

Se pide plantear un sistema de ecuaciones lineales para este problema.
El informe debe justificar la validez del sistema propuesto, explicando por
qu\'e dicho sistema resuelve efectivamente el problema de distribuci\'on.
El informe debe contener tambi\'en una descripci\'on detallada del m\'etodo
que implementa el programa para la resoluci\'on de este sistema de ecuaciones.

Se pide realizar experimentos con redes de prueba para validar el
funcionamiento de la implementaci\'on. Medir el n\'umero de condici\'on de
las matrices de los sistemas de ecuaciones resultantes, para tener una idea
del error num\'erico cometido por la resoluci\'on. >Cu\'al es el tama\~no
de la mayor red el\'ectrica que el programa puede resolver en 10 minutos
o menos?

\vskip 15pt

\hrule

\vskip 11pt

Fecha de entrega: Lunes 30 de Abril

\end{document}
