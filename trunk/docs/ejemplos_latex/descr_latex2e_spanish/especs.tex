
\chapter{Especialidades}
\begin{intro}
  Si ya se siente lo sucifientemente seguro de s'i mismo, entonces
  ahora puede comenzar a escribir sus documentos en \LaTeX. El
  prop'osito de este cap'itulo es a~nadir algunas `especias'\ a sus
  conocimientos de \LaTeX. En el {\normalfont\manual{}} y {\normalfont
  \companion} podr'a encontrar una descripci'on m'as completa de las
  especialidades y de las posibles mejoras que puede realizar con
  \LaTeX.
\end{intro}

\section{Tipos y tama~nos}

\index{tipo}\index{tama~no del tipo} \LaTeX{} elige el tipo y el
tama~no de los tipos bas'andose en la estructura l'ogica del
documento (apartados, notas al pie\ldots). En algunos casos podr'iamos
desear cambiar directamente los tipos y los tama~nos. Para realizar
esto se pueden usar las instrucciones de las tablas~\ref{fonts}
y~\ref{sizes}. El tama~no real de cada tipo es cuesti'on de dise~no
y depende de la clase de documento y de sus opciones.

\begin{example}
{\small Los peque~nos y
\textbf{gordos} romanos dominaron}
{\Large toda la grande
\textit{Italia}.}
\end{example}

Una caracter'istica importante de \LaTeXe{} es que los atributos de
los tipos son independientes. Esto significa que se puede llamar a
instrucciones de cambio de tama~no o incluso de tipo y a'un as'i se
mantienen los atributos de negrita o inclinado que se establecieron
previamente. Si bien esto puede resultar evidente para alguien que
aprenda \LaTeX{} desde cero, esto no lo es tanto para quien haya
empleado \LaTeX{} 2.09.

En el \emph{modo matem'atico} se pueden emplear instrucciones de
cambio de tipos para salir temporalmente del \emph{modo matem'atico} e
introducir texto normal. Si para componer las ecuaciones Vd.{} desea
utilizar otro tipo existe un conjunto especial de instrucciones para
ello. V'ease la tabla~\ref{mathfonts}.

\begin{table}[!bp]
\caption{Tipos} \label{fonts}
\begin{lined}{12cm}
%
\begin{tabular}{@{}rl@{\qquad}rl@{}}
\ci{textrm}\verb|{...}|        &  \textrm{\wi{redonda}}&
\ci{textsf}\verb|{...}|        &  \textsf{\wi{sin l'inea de pie}}\\
\ci{texttt}\verb|{...}|        &  \texttt{de m'aquina}\\
                               &  \texttt{de escribir}&
                               &                         \\[6pt]
\ci{textmd}\verb|{...}|        &  \textmd{media}&
\ci{textbf}\verb|{...}|        &  \textbf{\wi{negrita}}\\[6pt]
\ci{textup}\verb|{...}|        &  \textup{\wi{vertical}}&
\ci{textit}\verb|{...}|        &  \textit{\wi{it'alica}}\\
\ci{textsl}\verb|{...}|        &  \textsl{\wi{inclinada}}&
\ci{textsc}\verb|{...}|        &  \textsc{\wi{versalita}}\\[6pt]
\ci{emph}\verb|{...}|          &  \emph{resaltada} &
\ci{textnormal}\verb|{...}|    &  tipo del\\
 & & & \textnormal{documento}
\end{tabular}

\bigskip
\end{lined}
\end{table}


\begin{table}[!bp]
\index{tama~nos del tipo}
\caption{Tama~nos de los tipos} \label{sizes}
\begin{lined}{12cm}
\begin{tabular}{@{}ll}
\ci{tiny}      & \tiny            letra diminuta \\
\ci{scriptsize}   & \scriptsize   letra muy peque~na\\
\ci{footnotesize} & \footnotesize letra bastante peque~na \\
\ci{small}        &  \small       letra peque~na \\
\ci{normalsize}   &  \normalsize  letra normal \\
\ci{large}        &  \large       letra grande
\end{tabular}%
\qquad\begin{tabular}{ll@{}}
\ci{Large}        &  \Large       letra mayor \\[5pt]
\ci{LARGE}        &  \LARGE       muy grande \\[5pt]
\ci{huge}         &  \huge        enorme \\[5pt]
\ci{Huge}         &  \Huge        la mayor
\end{tabular}

\bigskip
\end{lined}
\end{table}

\begin{table}[!bp]
\caption{Tipos matem'aticos} \label{mathfonts}
\begin{lined}{12.2cm}
\begin{tabular}{@{}lll@{}}
\textit{Orden}&\textit{Ejemplo}&    \textit{Resultado}\\[6pt]
\ci{mathcal}\verb|{...}|&    \verb|$\mathcal{B}=c$|&     $\mathcal{B}=c$\\
\ci{mathrm}\verb|{...}|&     \verb|$\mathrm{K}_2$|&      $\mathrm{K}_2$\\
\ci{mathbf}\verb|{...}|&     \verb|$\sum x=\mathbf{v}$|& $\sum x=\mathbf{v}$\\
\ci{mathsf}\verb|{...}|&     \verb|$\mathsf{G\times R}$|&        $\mathsf{G\times R}$\\
\ci{mathtt}\verb|{...}|&     \verb|$\mathtt{L}(b,c)$|&   $\mathtt{L}(b,c)$\\
\ci{mathnormal}\verb|{...}|& \verb|$\mathnormal{R_1}=R_1$|&      $\mathnormal{R_1}=R_1$\\
\ci{mathit}\verb|{...}|&     \verb|$eficaz\neq\mathit{eficaz}$|& $eficaz\neq\mathit{eficaz}$
\end{tabular}

\bigskip
\end{lined}
\end{table}

Conjuntamente con las instrucciones de los tama~nos de los tipos, las
\wi{llaves} juegan un papel significativo. Se utilizan para construir
agrupaciones o \emph{grupos}. Los \wi{grupo}s limitan el 'ambito de la
mayor'ia de las instrucciones de \LaTeX.

\begin{example}
A 'el le gustan las {\LARGE
letras grandes y las letras
{\small peque~nas}}.
\end{example}
 
Las instrucciones de tama~no del tipo tambi'en alteran el espaciado
entre renglones, pero s'olo si el p'arrafo termina dentro del 'ambito
de la orden de tama~no del tipo. Por ello, la llave de cierre \verb|}|
no deber'ia aparecer antes de lo indicado. Obs'ervese la posici'on de
la instrucci'on \verb|\par| en los dos ejemplos siguientes.

\begin{example}
{\Large !`No lea esto! No es
cierto. !`Cr'eame!\par}
\end{example}
\begin{example}
{\Large Esto no es cierto.
Pero recuerde que digo
mentiras.}\par
\end{example}

Para concluir este viaje al mundo de los tipos y los tama~nos de
tipos, tenga Vd.\ un peque~no consejo:
\nopagebreak
\begin{quote}
  \underline{\textbf{Recuerde\Huge!}} \textit{Cuanto}
  \textsf{M\textbf{\LARGE 'A} \textsl{S}} tipos \Huge utilice \tiny
  Vd. \footnotesize \textbf{en} un \small \texttt{documento}\Huge,
  \large \textit{m'as} \normalsize \textsc{legible} y
  \textsl{\textsf{agradable} resul\large t\Large a\LARGE r\huge
    'a}.\footnote{!`Ojo!, que se trata de una peque~na s'atira.
    !`Espero que se de cuenta!}
\end{quote}

\section{Separaciones}
 
\subsection{Separaciones entre renglones}

\index{separaciones entre renglones} Si quiere emplear mayores
separaciones entre renglones, puede cambiar su valor poniendo la orden
\begin{command}
\ci{linespread}\verb|{|\emph{factor}\verb|}|
\end{command}
\noindent en el pre'ambulo de su documento. Utilice
\verb|\linespread{1.3}| para textos a espacio y medio y
\verb|\linespread{1.6}| para textos a doble espacio. Normalmente los
renglones no se separan tanto, por lo que, a no ser que se indique
otra cosa, el factor de separaci'on entre renglones es~1.%
\index{doble espacio}
 
\subsection{Dise~no de los p'arrafos}\label{parsp}

En \LaTeX{} existen dos par'ametros que influyen sobre el formato de
los p'arrafos. Si se pone una definici'on como
\begin{code}
\ci{setlength}\verb|{|\ci{parindent}\verb|}{0pt}| \\
\verb|\setlength{|\ci{parskip}\verb|}{1ex plus 0.5ex minus 0.2ex}|
\end{code}
en el pre'ambulo del fichero de entrada\footnote{Entre las
  instrucciones $\backslash$\texttt{documentclass} y
  $\backslash$\texttt{begin$\mathtt{\{}$document$\mathtt{\}}$}.} se
puede cambiar el aspecto de los p'arrafos. Estas dos l'ineas pueden
aumentar el espacio entre dos p'arrafos y dejarlos sin sangr'ias. En
la Europa continental, a menudo se separan los p'arrafos con alg'un
espacio y no se le pone sangr'ia. Pero tenga cuidado, ya que esto
tambi'en tiene efecto en el 'indice general, haciendo que sus l'ineas
queden m'as separadas.

Si desea sangrar un p'arrafo que no tiene sangr'ia, entonces utilice
\begin{command}
\ci{indent}
\end{command}
\noindent al comienzo del p'arrafo\footnote{Para sangrar el primer
  p'arrafo despu'es de cada cabecera de apartado, util'icese el
 paquete \pai{indentfirst} del conjunto `tools'.}. Esto s'olo
 funcionar'a cuando \verb|\parindent| no est'e puesto a cero.

Para crear un p'arrafo sin sangr'ia use
\begin{command}
\ci{noindent}
\end{command}
\noindent como primera orden del p'arrafo. Esto podr'ia resultar 'util
cuando comience un documento con texto y sin ninguna instrucci'on de
seccionado.

\subsection{Separaciones horizontales}
 
\LaTeX{} determina autom'aticamente las separaciones entre palabras y
 oraciones. Para producir otras separaciones horizontales utilice:
 \index{espacio!horizontal}
\begin{command}
\ci{hspace}\verb|{|\emph{longitud}\verb|}|
\end{command}
Cuando se debe producir una separaci'on como 'esta, incluso si cae al
final o al comienzo de un rengl'on, utilice \verb|\hspace*| en vez de
\verb|\hspace|. La indicaci'on de la distancia consta, en el caso m'as
simple, de un n'umero m'as una unidad. En la tabla~\ref{units} se
muestran las unidades m'as importantes.
\index{unidades}\index{dimensiones}

\begin{example}
Este\hspace{1.5cm}es un espacio
de 1.5 cm.
\end{example}
\suppressfloats
\begin{table}[tbp]
\caption{Unidades de \TeX} \label{units}\index{unidades}
\begin{lined}{9.5cm} 
\begin{tabular}{@{}ll@{}}
\texttt{mm} &  mil'imetro $\approx 1/25$~pulgada \quad \demowidth{1mm} \\
\texttt{cm} & cent'imetro = 10~mm  \quad \demowidth{1cm}                    \\
\texttt{in} & pulgada $\approx$ 25~mm \quad \demowidth{1in}                 \\
\texttt{pt} & punto $\approx 1/72$~pulgada $\approx \frac{1}{3}$~mm  \quad\demowidth{1pt}\\
\texttt{em} & aprox.{} el ancho de una \texttt{m} en el tipo actual \quad \demowidth{1em}\\
\texttt{ex} & aprox.{} la altura de una \texttt{x} en el tipo actual \quad \demowidth{1ex}
\end{tabular}

\bigskip
\end{lined}
\end{table}
 
La instrucci'on
\begin{command}
\ci{stretch}\verb|{|\emph{n}\verb|}|
\end{command} 
\noindent produce una separaci'on especial el'astica. Se alarga hasta
que el espacio que resta en un rengl'on se llena. Si dos instrucciones
\verb|\hspace{\stretch{|\emph{n}\verb|}}| aparecen en el mismo
rengl'on, los espaciados crecen seg'un sus `factores de alargamiento'.
\begin{example}
x\hspace{\stretch{1}}
x\hspace{\stretch{3}}x
\end{example}

\subsection{Separaciones verticales especiales}

\LaTeX{} determina de modo autom'atico las separaciones entre dos
p'arrafos, apartados, subapartados\ldots\ En casos especiales se
pueden forzar separaciones adicionales \emph{entre dos p'arrafos} con
la orden
\begin{command}
\ci{vspace}\verb|{|\emph{longitud}\verb|}|
\end{command}

Esta orden se deber'ia indicar siempre entre dos renglones vac'ios.
Cuando esta separaci'on se debe introducir aunque vaya al principio o
al final de una p'agina, entonces en vez de \verb|\vspace| se debe
utilizar \verb|\vspace*|. \index{separaci'on vertical}

Se puede utilizar la orden \verb|\stretch| conjuntamente con
\verb|\pagebreak| para llevar texto al borde inferior de una
p'agina o para centrarlo verticalmente.

\begin{code}
\begin{verbatim}
Algo de texto \ldots

\vspace{\stretch{1}}
Esto va en el 'ultimo rengl'on de la p'agina.\pagebreak
\end{verbatim}
\end{code}

Las separaciones adicionales entre dos renglones \emph{del mismo
p'arrafo} o dentro de una tabla se consiguen con la orden
\begin{command}
\ci{\bs}\verb|[|\emph{longitud}\verb|]|
\end{command}

\section{Dise~no de la p'agina}

\begin{figure}[!hp]
\begin{center}
\makeatletter
\lay@layout
\makeatother
\end{center}
\vspace*{1.8cm}
\caption{Par'ametros del dise~no de la p'agina}
\label{fig:layout}
\end{figure}

\index{dise~no de la p'agina}
\LaTeXe{} le permite indicar el \wi{tama~no del papel} en la orden
\verb|\documentclass|. Entonces elige autom'aticamente los
\wi{m'argenes} del texto apropiados. Pero a veces puede que no se
encuentre conforme con los valores predefinidos. Naturalmente, los
puede cambiar.
% Ni idea de porqu'e hace falta esto aqu'i...
\thispagestyle{fancyplain}%
La figura~\ref{fig:layout} muestra todos los par'ametros que se pueden
cambiar. La figura se ha producido con el paquete \pai{layout} del
conjunto `tools'
\footnote{\texttt{CTAN:/tex-archive/macros/latex/packages/tools}.}.

\LaTeX{} proporciona dos instrucciones para cambiar estos
 par'ametros. Normalmente se utilizan en el pre'ambulo del documento.

La primera instrucci'on asigna un valor fijo para al par'ametro:
\begin{command}
\ci{setlength}\verb|{|\emph{par'ametro}\verb|}{|\emph{longitud}\verb|}|
\end{command}

La segunda instrucci'on le a~nade una longitud al par'ametro:
\begin{command}
\ci{addtolength}\verb|{|\emph{par'ametro}\verb|}{|\emph{longitud}\verb|}|
\end{command} 

De hecho, esta segunda instrucci'on es m'as 'util que la orden
\ci{setlength}, porque puede trabajar tomando como referencia las
dimensiones anteriormente definidas. Para a~nadir un cent'imetro al
ancho del texto, en el pre'ambulo del documento se pondr'ian las
siguientes instrucciones:
\begin{code}
\verb|\addtolength{\hoffset}{-0.5cm}|\\
\verb|\addtolength{\textwidth}{1cm}|
\end{code}

\section{Notas bibliogr'aficas}

Con el entorno \ei{thebibliography} se puede imprimir una
bibliograf'ia. Cada nota bibliogr'afica se introduce con
\begin{command}
\ci{bibitem}\verb|{|\emph{marcador}\verb|}|
\end{command}
El \emph{marcador} se utilizar'ia dentro del documento para indicar la
entrada en la bibliograf'ia (o sea, como una cita):
\begin{command}
\ci{cite}\verb|{|\emph{marcador}\verb|}|
\end{command}
La numeraci'on de las entradas se realiza autom'aticamente. El
par'ametro que se coloca tras la instrucci'on
\verb|\begin{thebibliography}| establece el ancho m'aximo del espacio
  destinado a estos n'umeros.

\begin{example}
Partl~\cite{pa} ha 
propuesto que\ldots
 
\begin{thebibliography}{99}
\bibitem{pa} H.~Partl: 
\emph{German \TeX},
TUGboat Vol.~9, No.~1 ('88)
\end{thebibliography}
\end{example}
\chaptermark{Especialidades}
\thispagestyle{fancyplain}

En ocasiones se puede emplear otra alternativa para introducir la
bibliograf'ia. 'Esta se basa en la utilizaci'on de la herramienta
\BibTeX.\index{bibtex@\BibTeX} El \BibTeX\ es un programa que recoge
los marcadores de las citas que se han introducido en el documento.
Esta lista de marcadores la deposita \LaTeX\ al procesar el documento
en un fichero especial. Este fichero tiene el mismo nombre que el
fichero original pero con una extensi'on diferente (\texttt{.aux}). En
realidad, en este fichero se deposita mucha m'as informaci'on que la
de los marcadores de estas referencias bibliogr'aficas, ya que incluso
este fichero es le'ido por \LaTeX\ en posteriores procesamientos. En
cualquier caso, \BibTeX\ identifica estos marcadores de entre toda la
informaci'on en este fichero especial y entonces intenta buscar la
informaci'on bibliogr'afica correspondiente a cada marcador en unos
ficheros con la extensi'on \texttt{.bib}. La informaci'on que resulta
de esta b'usqueda es almacenada en otro fichero especial, que esta vez
tiene la extensi'on \texttt{.bbl}. Para terminar de incluir esta
informaci'on en el texto final nuevamente se debe procesar el
documento con \LaTeX.

A partir de los marcadores \BibTeX\ decide qu'e referencias son las
que debe introducir en la bibliograf'ia del documento. Si por alguna
raz'on tambi'en se desea que \BibTeX\ introduzca una determinada
referencia en la bibliograf'ia pero sin introducirla en medio del
texto como con la orden \verb|\cite|, entonces se puede emplear

\begin{command}
\ci{nocite}\verb|{|\emph{marcador}\verb|}|
\end{command}

La funci'on de los ficheros con la extensi'on \texttt{.bib} es la de
servir como bases de datos de referencias bibliogr'aficas. Para
indicar el nombre concreto del fichero o ficheros donde se deben
buscar estas referencias bibliogr'aficas se emplea

\begin{command}
\ci{bibliography}\verb|{|\emph{fichero}\verb|,|\emph{fichero}%
  \verb|,|\ldots\verb|}|
\end{command}

La estructura de estos ficheros de bibliograf'ia se puede consultar en
\manual\ o en \companion. La principal utilidad de emplear este
sistema en vez del anterior es que de esta forma la misma informaci'on
sobre las diversas referencias bibliogr'aficas puede ser igualmente
accesible para otros documentos de \LaTeX.

Por otra parte, a la hora de disponer las referencias en el documento
las entradas pueden ir siguiendo un determinado estilo. Para elegir
este estilo se emplea

\begin{command}
\ci{bibliographystyle}\verb|{|\emph{estilo}\verb|}|
\end{command}

\noindent La tabla~\ref{bibstyle} muestra los estilos
predefinidos.

\begin{table}[!hbp]
\caption{Estilos de entradas bibliogr'aficas predefinidas en \LaTeX}
\label{bibstyle}
\begin{lined}{12cm}
\begin{description}
  
\item[\normalfont\texttt{plain}] coloca las entradas de la
  bibliotraf'ia por orden alfab'etico. A cada una se le asigna un
  n'umero entre corchetes que es el asignado como marcador. Este es el
  mismo que aparece en el lugar de la llamada a esta referencia en el
  texto cuando se introduce la orden \verb|\cite|.

\item[\normalfont\texttt{unsrt}] ordena las entradas por sus primeras
  referencias con las 'ordenes \verb|\cite| y \verb|\nocite|.

\item[\normalfont\texttt{alpha}] ordena las entradas igual que
  \texttt{plain} pero los marcadores se construyen con una abreviatura
  del autor o autores y el a~no de publicaci'on.

\item[\normalfont\texttt{abbrv}] ordena las entradas igual que
  \texttt{plain} y construye los marcadores de la misma forma, pero en
  la indicaci'on de la referencia se emplean abreviaturas para los
  nombres de pila, meses y, en ocasiones, los nombres de las revistas.

\end{description}
\end{lined}
\end{table}


\section{Indexado} \label{sec:indexing}
Una facilidad muy 'util para muchos libros es el 'indice de
materias.\index{indice de materias@'indice de materias} Con \LaTeX{} y
el programa de ayuda \texttt{makeindex}\footnote{En algunos sistemas
  que no permiten nombres de ficheros mayores de 8~caracteres, el
  programa puede que se llame \texttt{makeindx}.}, los 'indices de
materias se pueden crear de un modo razonablemente sencillo. En esta
descripci'on, s'olo se explicar'an las instrucciones b'asicas de
producci'on de 'indices de materias. Para una visi'on en mayor
profundidad por favor dir'ijase a \companion.%
\index{programa makeindex@programa \texttt{makeindex}}%
\index{paquete makeidx@paquete \texttt{makeidx}}

Para habilitar la facilidad de 'indice de materias de \LaTeX{} se
debe cargar en el pre'ambulo el paquete \pai{makeidx} con:
\begin{command}
\verb|\usepackage{makeidx}|
\end{command}
\noindent y las instrucciones especiales de indexado se deben
habilitar con la instrucci'on
\begin{command}
  \ci{makeindex}
\end{command}
\noindent en el pre'ambulo de los ficheros de entrada.

El contenido del 'indice se indica con instrucciones
\begin{command}
  \ci{index}\verb|{|\emph{clave}\verb|}|
\end{command}
\noindent donde \emph{clave} es la entrada en el 'indice. Se incluyen
las instrucciones de indexado en los lugares del texto a donde se
quiere apuntar. La tabla~\ref{index} muestra la sintaxis del argumento
\emph{clave} con varios ejemplos.

\begin{table}[!bp]
\caption{Ejemplos de sintaxis de llaves para 'indices de materias}
\label{index}
\begin{center}
\begin{tabular}{@{}lll@{}}
  \textbf{Ejemplo} &\textbf{Entrada} &\textbf{Comentario}\\\hline
  \rule{0pt}{1.05em}\verb|\index{hola}| &hola, 1 &Entrada simple\\ 
  \verb|\index{hola!Pedro}| &\hspace*{2ex}Pedro, 3 &Subentrada bajo
  `hola'\\
  \verb|\index{Juan@\textsl{Juan}}| &\textsl{Juan}, 2& Entrada con
  dise~no\\
  \verb|\index{Pepa@\textbf{Pepa}}| &\textbf{Pepa}, 7& Igual que antes\\
  \verb.\index{Loli|textbf}.  &Loli, \textbf{3}& N"o de p'agina
  con dise~no\\ 
  \verb.\index{Soraya|textit}.  &Soraya, \textit{5}& Igual que antes
\end{tabular}
\end{center}
\end{table}

Cuando se procesa el fichero de entrada con \LaTeX, cada instrucci'on
\verb|\index| escribir'a en un fichero especial la entrada en el
'indice con el n'umero de la p'agina actual. El fichero tiene el mismo
nombre que el fichero de entrada de \LaTeX{} pero con una extensi'on
distinta (\verb|.idx|). Despu'es se puede procesar este fichero
\texttt{.idx} con el programa \texttt{makeindex}.
\begin{command}
  \texttt{makeindex} \emph{fichero}
\end{command}
El programa \texttt{makeindex} produce un 'indice ordenado con la misma
base de nombre de fichero pero esta vez con la extensi'on
\texttt{.ind}. Si se procesa ahora el fichero de entrada \LaTeX{} de
nuevo, entonces este 'indice se incluye en el documento donde se
encuentra la instrucci'on
\begin{command}
  \ci{printindex}
\end{command}

El paquete \pai{showidx} que viene con \LaTeXe{} imprime todas las
entradas en el 'indice en el margen izquierdo del texto. Esto es
bastante 'util para las revisiones del documento y para verificar el
'indice.

% A~nadir alguna informaci'on sobre PROTECT ...  

\section{Inclusi'on de gr'aficos EPS}
Con los entornos \texttt{figure} y \texttt{table} \LaTeX{} proporciona
las facilidades b'asicas para trabajar con objetos flotantes, entre
los que se incluyen las im'agenes y los gr'aficos.

Tambi'en existen varias posibilidades para generar \wi{gr'aficos} con
el \LaTeX{} b'asico o un paquete de extensiones de \LaTeX. Por
desgracia, la mayor'ia de los usuarios los encuentran dif'iciles de
entender. Por esto, no se van a explicar en este manual. Para m'as
informaci'on sobre este particular cons'ultense \companion{} y el
\manual.

Un modo m'as sencillo de poner gr'aficos en un documento es
produci'endolos con un paquete de \emph{software}
especializado\footnote{Tales como XFig, CorelDraw!, Freehand, Gnuplot,
  Tgif, Paint Shop Pro, Gimp\ldots} e incluir los gr'aficos dentro del
documento. En este punto, tambi'en los paquetes de \LaTeX{} ofrecen
muchas alternativas. En esta descripci'on s'olo se mostrar'a el uso de
gr'aficos en \wi{PostScript Encapsulado} (EPS), ya que es un m'etodo
muy sencillo y ampliamente utilizado. Para utilizar dibujos en formato
EPS, debe disponer una impresora \wi{PostScript}\footnote{Otra
  posibilidad para imprimir PostScript es con el programa de GNU
  \textsc{\wi{GhostScript}}, que puede encontrar en
  \texttt{CTAN:/tex-archive/support/ghostscript}.} para imprimir.

Un buen conjunto de 'ordenes para la inclusi'on de gr'aficos se
proporciona con el paquete \pai{graphicx} de D.~P.~Carlisle. Forma
parte de todo un conjunto de paquetes que se llama el conjunto
``graphics''%
\footnote{\texttt{CTAN:/tex-archive/macros/latex/packages/graphics}.}.

Suponiendo que Vd.\ se halle trabajando con una impresora PostScript
para imprimir y con el paquete \textsf{graphicx}, puede seguir la
siguiente lista de pasos para incluir un dibujo dentro de su documento:

\begin{enumerate}
\item Exportar el dibujo desde su programa de gr'aficos en formato EPS.
\item Cargar el paquete \textsf{graphicx} en el pre'ambulo del fichero
  de entrada con
\begin{command}
\verb|\usepackage[|\emph{driver}\verb|]{graphicx}|
\end{command}
\emph{driver} es el nombre de su conversor ``de \emph{dvi} a
PostScript''\footnote{El programa m'as utilizado para esto se llama
  \texttt{dvips}.}. El paquete necesita esta informaci'on porque la
inclusi'on de los gr'aficos la realiza el \emph{driver} de la
impresora. Una vez que se conozca el \emph{driver}, el paquete
\textsf{graphicx} inserta las 'ordenes correctas en el
fichero~\texttt{.dvi} para incluir el gr'afico que se desea con el
\emph{driver} de impresora.
 
\item Utilice la orden
\begin{command}
\ci{includegraphics}\verb|[|\emph{clave}=\emph{valor},
 \ldots\verb|]{|\emph{fichero}\verb|}|
\end{command}
para incluir \emph{fichero} en su documento. El par'ametro opcional
acepta una lista de \emph{claves} separadas por comas y sus \emph{valores}
asociados. Las \emph{claves} se pueden emplear para modificar el ancho,
la altura y el giro del gr'afico incluido. La tabla~\ref{keyvals}
muestra las claves m'as importantes.
\end{enumerate}

\begin{table}[htb]
\caption{Nombres de las claves para el paquete \textsf{graphicx}}
\label{keyvals}
\begin{lined}{10.1cm}
\begin{tabular}{@{}ll}
\texttt{width}& escalado gr'afico al ancho indicado\\
\texttt{height}& escalado gr'afico a la altura indicada\\
\texttt{angle}& giro del gr'afico en el sentido de las agujas del reloj
\end{tabular}

\bigskip
\end{lined}
\end{table}

El siguiente ejemplo podr'a ayudar a clarificar algunas de estas ideas:
\begin{code}
\begin{verbatim}
\begin{figure}
\begin{center}
\includegraphics[angle=90, width=10cm]{test.eps}
\end{center}
\end{figure}
\end{verbatim}
\end{code}
Este c'odigo introduce el gr'afico que se encuentra en el fichero
\texttt{test.eps}. El gr'afico se gira \emph{primero} 90$^\circ$ y
\emph{despu'es} se escala hasta lograr los 10\,cm de ancho. La
relaci'on de aspecto es de 1.0 porque no se ha indicado ninguna altura
especial.


Para m'as informaci'on, por favor consulte~\cite{graphics}.  

\endinput
