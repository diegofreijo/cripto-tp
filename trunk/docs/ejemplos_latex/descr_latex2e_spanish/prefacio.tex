
\chapter{Prefacio}

\LaTeX~\cite{manual} es un sistema de composici'on de textos que est'a
orientado especialmente a la creaci'on de documentos cient'ificos que
contengan f'ormulas matem'aticas. Adem'as, tambi'en se pueden crear
otros tipos de documentos, que pueden ser desde cartas sencillas hasta
libros completos. \LaTeX\ est'a organizado sobre \TeX~\cite{texbook}.
 
El presente documento describe \LaTeX\ y deber'ia bastar para la
mayor'ia de las aplicaciones de \LaTeX. Existen diversos
manuales~\cite{manual,companion} donde se encuentra una descripci'on
completa de \LaTeX.

\LaTeX\ est'a disponible para la mayor�a de los miniordenadores y
microordenadores, desde IBM~PCs en adelante. En muchas redes
universitarias de ordenadores se encuentra instalado para utilizarse
al instante. En la \guia\ correspondiente se describe c'omo se accede
a la instalaci'on de \LaTeX, c'omo se opera con ella y de qu'e
complementos se dispone.

El prop'osito de este documento \emph{no} es indicar c'omo se instala
y se mantiene un sistema de \LaTeX, sino mostrar c'omo escribir
documentos para que se puedan procesar con \LaTeX.

\bigskip
\noindent Esta descripci'on se divide en cuatro cap'itulos:
\begin{description}
\item[El cap'itulo 1] muestra la estructura b'asica de los documentos
  de \LaTeXe. Tambi'en se ense~na un poco de la historia de
  \LaTeX. Tras leer este cap'itulo se deber'ia tener una visi'on muy
  escueta de \LaTeX. Esta visi'on consistir'a s'olo de un peque~no
  ``marco de trabajo'' en el que podr'a integrar la informaci'on que
  se proporciona en los cap'itulos posteriores y otras fuentes ---como
  los manuales~\cite{manual,companion}---.
\item[El cap'itulo 2] incide en los detalles sobre la composici'on de
  los documentos. Explica la mayor'ia de las instrucciones y los entornos
  b'asicos de \LaTeX. Una vez le'ido este cap'itulo ser'a capaz de
  escribir sus primeros documentos.
\item[El cap'itulo 3] explica c'omo componer f'ormulas matem'aticas
  con \LaTeX. Aqu'i se presentan varios ejemplos para ayudarle a
  entender una de las principales potencialidades de \LaTeX. Al final
  de este cap'itulo encontrar'a varias tablas con todos los s'imbolos
  matem'aticos disponibles en \LaTeX.
\item[El cap'itulo 4] indica otras posibilidades que se pueden obtener
  de \LaTeX{}, que, si bien no son esenciales, a veces pueden resultar
  muy 'utiles. Por ejemplo, se muestra c'omo incluir gr'aficos de
  PostScript encapsulado en sus documentos o c'omo a~nadir un 'indice
  de materias en su documento.
\end{description}
%
\bigskip
\noindent Es importante leer los cap'itulos en secuencia. Por favor,
lea cuidadosamente los ejemplos, ya que en los diversos ejemplos que
encontrar'a en esta descripci'on es donde se encuentra gran parte de
la informaci'on.
%
\bigskip
\noindent Si necesita cualquier material relacionado con \LaTeX,
examine cualquiera de los servidores de archivos de \texttt{CTAN}\@.
En la Rep'ublica Federal de Alemania es \texttt{ftp.dante.de} y en el
Reino Unido es \texttt{ftp.tex.ac.uk}. Tambi'en existen diversos
espejos. Si no se encuentra en uno de estos pa'ises, por favor elija
el servido m'as cercano.

\vspace{\stretch{1}}
\noindent Si tiene ideas sobre algo que deber'ia ser a~nadido o
alterado en este documento, por favor h'aganoslo saber. Estamos
especialmente interesados en los principiantes con \LaTeX.
\bigskip
\begin{verse}
\contrib{Tom'as Bautista}{bautista@cma.ulpgc.es}%
{Divisi'on de CAD,
Centro de Microelectr'onica Aplicada,
Universidad de Las Palmas de G.C.}
\end{verse}
\vspace{\stretch{1}}
\noindent La versi'on vigente de este documento estar'a disponible en:\\
\texttt{$<$ftp://ftp.cma.ulpgc.es/pub/tex/latex2e/doc/ldesc2e$>$}
\endinput
