\section{Conclusiones}

% PAUTAS: esta secci�n debe contener las conclusiones generales del trabajo. Se deben 
% mencionar las relaciones de la discuci�n sobre las que se tiene certeza, junto con 
% comentarios y observaciones generales aplicables a todo el proceso. Mencionar tambi�n 
% posibles extensiones a los m�todos, experimentos que hayan quedado pendientes, etc.

% TAREAS: conclusiones generales sobre certezas. Mejoras y cosas pendientes.

La incidencia del valor $\beta$ en los tiempos de ejecuci\'on de nuestras instancias de prueba es dificil de predecir con exactitud. Somos concientes de que los datos de prueba no fueron construidos con el fin de obtener algun resultado en particular.\\
Notamos que dada la naturaleza del problema nunca podemos incurrir en una instancia particular del mismo que sea no acotada ya que al menos tenemos una restricci\'on para cada uno de los productos debido a la presencia del factor de demanda m\'aximo.\\
Una ves resuelto el problema de tener la implementaci\'on del algoritmo resuelta nos hubiese gustado mucho disponer de otras instancias de pruebas para confirmar nuestros principios de hipotesis.\\

