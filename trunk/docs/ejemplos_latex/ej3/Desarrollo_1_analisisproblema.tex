\subsection{An\'alisis del problema}

\subsubsection{Modelo matem\'atico}

Como paso previo al dise\~no de una soluci\'on en computadora del problema, analicemos el problema y sus implicaciones.

La disciplina conocida como \emph{revenue management} se ocupa de establecer pol\'iticas de control sobre los recursos para maximizar los ingresos.

En este caso tenemos las ciudades conectadas por distintos tramos con sus capacidades y costos definidos por una Aerol\'inea que opera sobre una determinada cantidad de rutas.

Esto se puede modelar matem\'aticamente mediante un conjunto de ecuaciones que representan las restricciones en cuanto a la cantidad de pasajeros por tramo. Y la funci\'on a maximizar va a ser la ganancia que se va a obtener dependiendo del costo del pasaje y su cupo.
Adem\'as de las restricciones que pone la aerol\'inea para un determinado producto, las esperanzas de las demandas, y su costo, la suma de todos los pasajeros que van a viajar en cada tramo no puede superar a la capacidad del avi\'on para dicho tramo.

En resumen, el problema consiste en determinar la cantidad \'optima de pasajes a vender para cada producto teniendo en cuenta de conseguir la mayor ganancia posible.

\subsubsection{Planteo como sistema de ecuaciones lineales}

Primero debimos resolver un planteo del problema apropiado para resolver mediante un sistema de ecuaciones lineales. 
El modelo que planteamos para resolver el problema fue el siguiente:
$$\mbox{maximizar}  \hspace{1cm}   cx$$

$$\mbox{sujeto a:}  \hspace{1cm}   Ax \leq b$$

Nuestra matriz $A$ esta formada por $n$ columnas y $m$ filas. Donde: 
$$n = Cantidad de Productos$$
$$m = Cantidad de Productos + Cantidad de Tramos$$

Las filas $1,2,...,n$ van a estar representando a los productos. O sea, cada $a_{kl} = 1$ si $k = l$ y $0$ si no. (Con $k = 1,2,...,n$ y $l = 1,2,...,n$).

Las filas $n+1,n+2,...,m$ van a estar representando los tramos que abarca cada producto. O sea, cada $a_{ij} = 1$ si el tramo $k$ pertenece al producto $l$ y $0$ si no. (Con $i = n+1,n+2,...,m$ y $j = 1,2,...,n$).

Nuestro vector $b$ va a tener la misma cantidad de filas que $A$ y va a ser de una sola columna. En las filas $1,2,...,n$ vamos a tener la demanda estimada por producto.
Y en las filas $n+1,n+2,...,m$ vamos a tener la cantidad de asientos disponibles por tramos.

Nuestro vector $c$ a maximizar va a tener 1 fila y n columnas. En cada posici\'on va a tener el precio de cada producto.

COMPLETAR!!! Explicar mejor mas...

COMPLETAR!!! El informe debe justificar la validez del modelo propuesto, explicando por qu\'e dicho sistema resuelve efectivamente el problema de asignaci\'on de asientos


%------------------------------------------------------------------------------%
