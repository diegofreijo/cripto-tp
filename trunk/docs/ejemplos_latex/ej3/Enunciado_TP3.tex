\begin{centering}
\bf Laboratorio de M\'etodos Num\'ericos - Primer cuatrimestre 2007 \\
\bf Trabajo Pr\'actico N\'umero 3: D\'onde est\'a el piloto? \\
\end{centering}

\vskip 25pt
\hrule
\vskip 11pt

El problema general de programaci\'on lineal puede escribirse de la siguiente
forma:
\begin{eqnarray}
\hbox{maximizar} \qquad cx & & \nonumber \\
\hbox{sujeto a:} \qquad Ax & \le & b \nonumber \\
x & \in & \hbox{\bf R}^n_+ \nonumber
\end{eqnarray}
Es decir, se busca optimizar una funci\'on lineal (la \emph{funci\'on
objetivo}) sobre el conjunto de vectores $x\in\hbox{\bf R}^n_+$ que cumplen
simult\'aneamente un cierto n\'umero de desigualdades lineales (las
\emph{restricciones del problema}). Un m\'etodo general para resolver este
tipo de problemas es el \emph{m\'etodo simplex}, descubierto por G.~Dantzig
en 1947.

Cuando $b\ge 0$, decimos que el problema es \emph{homog\'eneo positivo}.
Es importante notar que un problema homog\'eneo positivo siempre es factible,
puesto que $x=0$ es una soluci\'on factible. El objetivo del trabajo
pr\'actico es implementar el m\'etodo simplex para resolver problemas de
programaci\'on lineal homog\'eneos positivos y realizar experimentos sobre
un problema particular.

\textbf{El problema}

La disciplina conocida como \emph{revenue management} se ocupa, entre otras
cosas, de establecer pol\'{\i}ticas de control sobre los recursos de una
organizaci\'on con el objetivo de maximizar sus ingresos. Se utilizan
actualmente t\'ecnicas de revenue management en diferentes \'areas de
servicios, como aerol\'{\i}neas, hoteles, y ferrocarriles. En el caso
particular de las aerol\'{\i}neas, el problema consiste en manejar las
capacidades de un conjunto de vuelos en una red que conecta varias ciudades.

\begin{figure}[h]
\begin{center}

\caption{Ejemplo de una red de aerol\'{\i}neas.}
\label{fig:Network}
\end{center}
\vspace*{-3mm}
\end{figure}

Por ejemplo, la Figura~\ref{fig:Network} muestra una posible red con tres
ciudades y tres vuelos que las conectan.
Llamamos \emph{tramo} a un vuelo directo entre dos ciudades (tambi\'en
llamado vuelo punto a punto). Cada tramo posee una cantidad determinada de
asientos disponibles. Un \emph{producto} sobre la red se define como una
combinaci\'on entre un \emph{itinerario} (ciudades de origen-destino y,
eventualmente, paradas intermedias) y una tarifa.

En el ejemplo, un posible
producto corresponde a un vuelo directo desde A hasta C, con una tarifa de
\$1200. Otro producto podr\'{\i}a ser un vuelo directo desde A hasta C pero,
a diferencia del caso anterior, con una tarifa de \$800. Tambi\'en es
posible tener productos distintos con igual origen-destino y tarifa, pero
con distintos itinerarios. Por ejemplo, un tercer producto podr\'\i a ser
el que va desde A hasta C haciendo un \emph{stop} en B, y con una tarifa de
\$500. El Cuadro~\ref{table:Products} muestra una posible definici\'on de
productos sobre esta red.

\begin{table}[!h]
{\small
\begin{center}
\begin{tabular}{|c|l|c|}
\hline
Producto & Origen-Destino & Tarifa\\
\hline
1 & A $\rightarrow$ C & 1200\\
2 & A $\rightarrow$ B $\rightarrow$ C & 800\\
3 & A $\rightarrow$ B & 500\\
4 & B $\rightarrow$ C & 500\\
5 & A $\rightarrow$ C & 800\\
6 & A $\rightarrow$ B $\rightarrow$ C & 500\\
7 & A $\rightarrow$ B & 300\\
8 & B $\rightarrow$ C & 300\\
\hline
\end{tabular}
\caption{Definici\'on de productos.}
\label{table:Products}
\end{center}
}
\vspace*{-4mm}
\end{table}

En esta tabla podemos ver que distintos productos pueden compartir uno o m\'as
tramos. Adem\'as, cada producto tiene asociada una esperanza
(determin\'{\i}stica) de la demanda que recibir\'a. Es importante destacar
que, en general, la demanda sobre productos con tarifas bajas suele
ser mayor que para productos con tarifas m\'as caras.

Llamamos $m$ a la cantidad de tramos de la red y $n$ a la cantidad de
productos. Denotamos por $c = (c_1, \dots, c_m)$ al vector de capacidades
de los tramos, de modo tal que el tramo $i$ tiene una capacidad de $c_i$
asientos. Denotamos por $r = (r_1,\dots,r_n)$ al vector de tarifas de los
productos definidos sobre la red, de modo tal que el ingreso obtenido por
reservar un asiento para el producto $j$ es $r_j$. Por \'ultimo,
llamamos $\mu_j$ a la esperanza de la demanda sobre el producto $j$,
para $j=1,\dots,n$.

El problema que la aerol\'{\i}nea debe resolver consiste en determinar
cu\'antos asientos debe reservar para cada producto, de manera tal de
maximizar los ingresos generados y cumpliendo al mismo tiempo las siguientes
condiciones:
\vspace*{-5mm}
\begin{itemize}
\item[(i)] Para cada tramo $i=1,\dots,m$, la cantidad de asientos
reservados para los productos que utilizan el tramo no debe exceder la
capacidad inicial $c_i$.

\item[(ii)] Para cada producto $j=1,\dots,n$, la cantidad de asientos
reservados no debe exceder la esperanza $\mu_j$ de la demanda.
\end{itemize}

\textbf{Enunciado}

El objetivo del trabajo pr\'actico es implementar un programa que tome como
entrada la definici\'on de la red, las capacidades iniciales, las
definiciones de los productos y las esperanzas de las demandas, y que
determine una asignaci\'on \'optima de asientos para los productos
considerando las condiciones (i) y (ii). El programa implementado debe tomar
los datos de entrada de un archivo de texto, cuyo formato queda a criterio
del grupo.

Se pide plantear un modelo de programaci\'on lineal \emph{homog\'eneo
positivo} para este problema. El informe debe justificar la validez del
modelo propuesto, explicando por qu\'e dicho sistema resuelve efectivamente
el problema de asignaci\'on de asientos. Es importante observar que al
plantear un modelo de programaci\'on lineal para el problema, en realidad
estamos proponiendo una \emph{relajaci\'on} del problema original, dado que
se permite que para un producto se reserve una cantidad fraccionaria de
asientos. Para la resoluci\'on de este modelo se debe implementar cualquiera
de las variantes vistas en clase del m\'etodo simplex.

Se pide tambi\'en realizar experimentos con distintas instancias para validar
el funcionamiento de la implementaci\'on. Para cada una de las instancias
propuestas, escalar el vector de capacidades $c$ multiplic\'andolo por
$\beta =$ 0.6, 0.8, 1.0, 1.2, 1.4 y resolver estas nuevas instancias
escaladas. ?`Se observa alguna relaci\'on entre las capacidades y el
tiempo de ejecuci\'on del algoritmo?.

\vskip 15pt

\hrule

\vskip 11pt

Fecha de entrega: Lunes 4 de Junio


