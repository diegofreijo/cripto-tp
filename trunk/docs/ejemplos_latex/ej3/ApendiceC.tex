\section{Instancias de pruebas para el M\'etodo Simplex Revisado}

\subsection{Casos de prueba del M\'etodo Simplex Revisado}

Estos casos de prueba fueron utilizados para probar el funcionamiento de del algoritmo independientemente del modelo utilizado para representar el problema.

Para ver los ejemplos con los cuales fue probado el M\'etodo Simplex Revisado ver el archivo testSimplexCHVATAL.cpp

\subsection{Unica Soluci\'on (acotado)}

\subsection{M\'as de una soluci\'on (no acotado)}

Tener en cuenta que dado que el vector b es siempre positivo para los casos 
de estudio de nuestro TP tenemos solo problemas factibles. Con una o varias
soluciones:
              
Todos los ejemplos que utilizamos son factibles con una unica soluci�n.
               
Aqui hay un ejemplo para probar que tiene mas de una solucion:
 
$maximizar   x1 + 3x2 - x3      $
$sujeto a   2x1 + 2x2 - x3 <= 10$
$           3x1 - 2x2 + x3 <= 10$
$            x1 - 3x2 + x3 <= 10$
                    
              $  x1,x2,x3 >= 0  $
 
Soluci�n: probar con $ x1 = 0 ; x2 = 5+0.5t ; x3 = t.$