\section{Discusi�n}

% PAUTAS: Se incluir� aqu� un an�lisis de los resultados obtenidos en la secci�n anterior (se
% analizara su validez, coherencia, etc). Deben analizarse como m�nimo los items pedidos en el
% enunciado. No es aceptable decir que "los recultados fueron los esperados", sin hacer clara 
% referencia a la te�ria a la cual se ajustan. Adem�s se deben mencionar los resultados
% interesantes y los casos "patol�gicos" encontrados.

Las instancias de prueba que ejecutamos no son lo sufientemente representativas como para poder extraer alguna idea del comportamiento del tiempo de ejecuci\'on en funci\'on del factor de demanda.
La complejidad de la version del algoritmo que elejimos para implementar nos acoto considerablemente los tiempos de prueba de otras instancias que nos pudieran proveer de alguna hipotesis.\\
A priori nuestra idea era que cuanto mayor sea $\beta$ mayor ser\'ia el tiempo de ejecuci\'on ya que las restricciones para los productos ser\'ian menores. Entiendase por esto al hecho de que ser\'ian menos restrictivas. Los valores de los factores de demanda multiplicados por un valor mayor a 1 har\'ia que los terminos independientes sean m\'as grandes y por ende el algoritmo tendr\'ia mayor libertad para acercentar los valores de las incognitas.\\
En el caso de que $\beta$ fuera menor a 1 el algoritmo debiera tener un comportamiento an\'alogo. Todo esto asumiendo que los valores esperados de la demanda son mayores a 1, algo que nos parece totalmente l\'ogico ya que ninguna empresa pretende vender solo un pasaje o menos de esto.\\
Las soluciones obtenidas presentan muy marcadamente dos grupos de valores. Aquellos que valen $0$ y aquellos otros que alcanzan el valor m\'aximo permitido por sus restricciones.\\

