\section{Introducci\'on}

\textit{Mat\'ias: yo opino que ac\'a deber\'ian estar al menos mencionados cualquier concepto te\'orico que aparezca en el informe - salvo los triviales. No vamos a explicar la suma, por ejemplo.}

\subsection{Aritm\'etica finita y An\'alisis de error}

\textit{Opini\'on: Creo que habr\'ia que hacer que esta secci\'on tenga m\'as sentido, coherencia reagrupando los p\'arrafos por ejemplo}

Cuando se usa una calculadora o computadora digital para realizar c\'alculos num\'ericos, se debe considerar un error inevitable, el llamado error de redondeo o truncamiento.

Truncamiento: Este consiste en usar solamente las n primeras cifras de un n\'umero para representarlo.

Redondeo: La representaci\'on del n\'umero tendr\'a una cantidad fija de d\'igitos, solo que ahora se utilizar\'a el n\'umero de n d\'igitos mas cercano al valor real.

Este error se origina porque la aritm\'etica realizada en una m\'aquina involucra n\'umeros con solo un n\'umero finito de d\'igitos, con el resultado de que muchos c\'alculos se realizan con representaciones aproximadas de los n\'umeros verdaderos. En una computadora, s\'olo un subconjunto relativamente peque�o del sistema de los n\'umeros reales se usa para representar a todos los n\'umeros reales. Este subconjunto contiene s\'olo n\'umeros racionales, positivos y negativos, y almacena una parte fraccionaria, llamada mantisa, junto con una parte exponencial, llamado exponente.
Aunque muchas veces la representaci\'on de un numero en la m\'aquina es exacta, esta despu\'es de haber sufrido alguna operaci\'on aritm\'etica pierde su precision total, es decir que hay un error de representaci\'on.

El objeto de este trabajo es realizar un an\'alisis emp\'irico del comportamiento de la funci\'on $G(x)$ para distintos valores de x, calculada con aritm\'etica de punto flotante de precision finita.

\subsection{Errores de representaci\'on y operaciones matem\'aticas}

Los n\'umeros reales representados en punto flotante son los m\'as utilizados en computaci\'on cient\'ifica. Como hemos observado al normalizar un n\'umero real se comete un error. Incluso n\'umeros con una representaci\'on finita en decimal, como 0.1, o 27.9, tienen una representaci\'on binaria infinita, tienen decimales binarios peri\'odicos. Por ello, cuando estos n\'umeros se almacenan en punto flotante se debe ''cortar'' este n\'umero a una cantidad finita de bits, y se incurre en un error de representaci\'on flotante. Estos errores, de truncado o redondeo, son inevitables. Adem\'as, cuando realizamos operaciones elementales como sumar o multiplicar n\'umeros tambi\'en se incurre en un error adicional, que es similar al de
normalizaci\'on.

\subsection{Ejemplo: Cancelaci\'on catastr\'ofica}

Hemos observado en este trabajo pr\'actico que al sumar (restar) n\'umeros de diferente (igual) signo, y de m\'odulo(magnitud creo que es mas adecuado que modulo!!!) muy parecido, se produce una p\'erdida de d\'igitos significativos en el resultado, incluso cuando \'este es exacto. Este fen\'omeno se denomina cancelaci\'on. En principio, la cancelaci\'on no es algo malo. Sin embargo, si el resultado de \'esta se utiliza en operaciones sucesivas, la p\'erdida de d\'igitos significativos puede generar la propagaci\'on de errores que pueden reducir fuertemente la exactitud del resultado final, y se produce una cancelaci\'on catastr\'ofica.

%--------------------------------------------------------------------------------------------------------
