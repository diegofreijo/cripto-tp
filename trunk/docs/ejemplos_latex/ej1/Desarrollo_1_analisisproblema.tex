\section{An\'alisis del problema}

Como etapa previa al desarrollo en computadora de los c\'alculos pedidos en el enunciado,
analicemos las propiedades de las funciones a calcular.

Recordemos la definici\'on de estas funciones:

\begin{eqnarray}
T(x) & = & \left\{ \begin{array}{cl}
                      1 & \hbox{si } x = 0 \\
                      \frac{e^x-1}{x} & \hbox{en caso contrario}
                    \end{array} \right. \nonumber \\
Q(x) & = & \Big|x - \sqrt{x^2+1} \Big| - \frac{1}{x+\sqrt{x^2+1}} \nonumber \\
G(x) & = & T( Q(x)^2 ) \nonumber
\end{eqnarray}

Pese a su aparente dificultad, se prueba que $Q(x) = 0$ $\forall x \in \real$ (ver m\'as abajo).
Por lo tanto, 
$$ G(x) = T( Q(x)^2 ) = T( 0^2 ) = T( 0 ) = 1 $$
por definici�n de $T(x)$.

Notemos que $T(x)$ es una funci\'on partida, definida en 1 para $x = 0$. Esta definici\'on
permite que $T(x)$ sea continua en $x = 0$ y en todo $\real$, como se demuestra m\'as abajo. \\

?`Que implica esto? Interpretamos que el esp\'iritu del problema no es obtener el valor
m\'as cercano posible al te\'orico, porque \'este es constante. Se trata, en cambio, de
observar las caracter\'isticas de distintas operaciones de punto flotante de la m\'aquina
y los errores de c\'alculo que producen.

Esto nos permiti\'o tomar distintas decisiones de implementaci\'on, privilegiando facilitad
de desarrollo por encima de la minimizaci\'on de errores en los c\'alculos. \\

%----------------------------------------------------------------------------------------------%

\subsection{Demostraci\'on de $Q(x) = 0$ $\forall x \in \real$}

Queremos probar que, matem\'aticamente, $Q(x) = 0$ para cualquier valor $x \in \real$
\begin{eqnarray}
Q(x) & = & \Big|x - \sqrt{x^2+1} \Big| - \frac{1}{x+\sqrt{x^2+1}} \nonumber
\end{eqnarray}
Para esto separaremos la demostraci\'on para distintos valores de $x$.

\begin{itemize}
\item Para $x < 0$:
\begin{eqnarray}
\Big|x - \sqrt{x^2+1}\Big| & = & \Big| -|x| - \sqrt{x^2+1} \Big| \nonumber \\
& = & \sqrt{x^2+1} + |x| \nonumber
\end{eqnarray}

Entonces:
\begin{eqnarray}
Q(x) & = & \Big|x - \sqrt{x^2+1}\Big| - \frac{1}{x+\sqrt{x^2+1}} \nonumber \\
\nonumber \\
& = & \Big(\sqrt{x^2+1} + |x|\Big) - \frac{1}{-|x|+\sqrt{x^2+1}} \nonumber \\
\nonumber \\
& = & \frac{\Big(\sqrt{x^2+1} + |x|\Big)\Big(\sqrt{x^2+1} - |x|\Big) - 1}{\Big(\sqrt{x^2+1} - |x|\Big)} \nonumber \\
& = & \frac{\sqrt{x^2+1}^2 - |x|^2 - 1}{\Big(\sqrt{x^2+1} - |x|\Big)} \nonumber \\
\nonumber \\
& = & \frac{x^2 + 1 - x^2 - 1}{\Big(\sqrt{x^2+1} - |x|\Big)} \nonumber \\
\nonumber \\
& = & 0 \nonumber
\end{eqnarray}

\item Para $x \geq 0$, veamos que $x - \sqrt{x^2+1} \leq 0$:

\begin{itemize}
\item Para $0 \leq x \leq 1$:
\begin{eqnarray}
0 & \leq & x \nonumber \\
0 & \leq & x^2 \nonumber \\
1 & \leq & x^2 + 1 \nonumber \\
1 & \leq & \sqrt{x^2 + 1} \nonumber \\
x & \leq & \sqrt{x^2 + 1} \nonumber \\
x - \sqrt{x^2 + 1} & \leq & 0 \nonumber
\end{eqnarray}

\item Y para $1 \leq x$:
\begin{eqnarray}
1 & \leq & x \nonumber \\
x & \leq & x^2 \nonumber \\
x^2 & < & x^2 + 1 \nonumber \\
x & < & \sqrt{x^2 + 1} \nonumber \\
x - \sqrt{x^2 + 1} & < & 0 \nonumber
\end{eqnarray}

Por lo tanto para $x \geq 0$ tenemos:
\begin{eqnarray}
\Big|x - \sqrt{x^2+1}\Big| & = & \sqrt{x^2+1} - x \nonumber
\end{eqnarray}

Entonces:
\begin{eqnarray}
Q(x) & = & \Big|x - \sqrt{x^2+1}\Big| - \frac{1}{x+\sqrt{x^2+1}} \nonumber \\
\nonumber \\
& = & \Big(\sqrt{x^2+1} - x\Big) - \frac{1}{\sqrt{x^2+1} + x} \nonumber \\
\nonumber \\
& = & \frac{\Big(\sqrt{x^2+1} - x\Big)\Big(\sqrt{x^2+1} + x\Big) - 1}{\Big(\sqrt{x^2+1} + x\Big)} \nonumber \\
\nonumber \\
& = & \frac{\sqrt{x^2+1}^2 - x^2 - 1}{\Big(\sqrt{x^2+1} + x\Big)} \nonumber \\
\nonumber \\
& = & 0 \nonumber
\end{eqnarray}
\end{itemize}
\end{itemize}

%----------------------------------------------------------------------------------------------%

\subsection{Demostraci�n de continuidad de $T(x)$}

Queremos ver que $T(x)$ es continua en $x = 0$. Para eso tenemos que probar:

\begin{enumerate}
\item $ \lim_{x\rightarrow 0^+} T(x)= \lim_{x\rightarrow 0^-} T(x) = L   (L\in \mathbb{R}) $,

\item $ T(0) = L$
\end{enumerate}

En conclusi\'on $$ \lim_{x\rightarrow 0} T(x) = T(0) $$


Si calculamos el $$ \lim_{x\rightarrow 0} \left( \frac{e^x - 1}{x} \right)$$ nos quedaria una indefinici\'on del tipo $( \frac{0}{0} )$.


Seg\'un la Regla de L'Hopital si tenemos dos funciones $f(x)$ y $g(x) \ne 0$, son derivables y $\lim_{x\rightarrow c} \frac{f(x)}{g(x)} = \frac{0}{0}$ o $ \frac{\infty}{\infty} $, entonces cuando $\lim_{x\rightarrow c} \frac{f'(x)}{g'(x)}$ existe o es $\infty$.


En este caso la funci\'on $f(x) = e^x -1$ y $g(x) = x$. Estas dos funciones son derivables y tanto el $\lim_{x\rightarrow 0^+} \frac{f(x)}{g(x)} = \frac{0}{0}$ como el $\lim_{x\rightarrow 0^-} \frac{f(x)}{g(x)} = \frac{0}{0}$.
\\

Entonces estamos dentro de las condiciones de la Regla de L'Hopital y por lo tanto:\\

\begin{enumerate}
\item $$ \lim_{x\rightarrow 0^+} \frac{f(x)}{g(x)} = \lim_{x\rightarrow 0^+} \frac{f'(x)}{g'(x)} = \lim_{x\rightarrow 0^+} \frac{(e^x -1)'}{(x)'} =  \lim_{x\rightarrow 0^+} \frac{e^x}{1} = 1$$

\item $$ \lim_{x\rightarrow 0^-} \frac{f(x)}{g(x)} = \lim_{x\rightarrow 0^-} \frac{f'(x)}{g'(x)} = \lim_{x\rightarrow 0^-} \frac{(e^x -1)'}{(x)'} =  \lim_{x\rightarrow 0^-} \frac{e^x}{1} = 1$$
\end{enumerate}

Por otro lado sabemos, por definici\'on, que $f(0) = 1$.


Entonces $T(x)$ es continua en $x = 0$ ya que :

\begin {enumerate}
\item $ \lim_{x\rightarrow 0^+} T(x) = \lim_{x\rightarrow 0^-} T(x) = 1  (1\in \mathbb{R})$, y

\item $ T(0) = 1$.
\end{enumerate}

Fuera de $x = 0$, T(x) es continua en todo $\real$ por ser cociente de funciones continuas.

%----------------------------------------------------------------------------------------------%
