\section{Resultados}

% Pautas: deben incluir los resultados de los experimentos. utilizando el formato m�s adecuado
% para su presentaci�n. Deberan especificar claramente a que experiencia corresponde cada
% resultado. No se incluiran aqui corridas de m�quina.

% TAREAS: presentar bien los datos medidos y  aclarar a que experimento corresponde cada grafico

\subsection{C\'alculo de G(x) en distintas precisiones, en un rango amplio de valores de $\Rset$}

Se grafic\'o el exponente del resultado en punto flotante del c\'alculo de $G(x)$ en la m\'aquina.
Esto permite observar y comparar la magnitud de resultados muy grandes y muy
peque\~nos dentro del mismo gr\'afico, sacrificando poder distinguir
el signo del resultado.

\begin{center}
	\includegraphics[width=0.96\textwidth]{punto2_mantisa64_global.jpg}
	\textbf{Gr\'afico 1}: Exponente para G(x), con mantisa de 64 bits
	\label{fig:G_global_64}
\end{center}

\begin{center}
	\includegraphics[width=0.96\textwidth]{punto2_mantisa56_global.jpg}
	\textbf{Gr\'afico 2}: Exponente para G(x), con mantisa de 56 bits
	\label{fig:G_global_56}
\end{center}

\begin{center}
	\includegraphics[width=0.96\textwidth]{punto2_mantisa48_global.jpg}
	\textbf{Gr\'afico 3}: Exponente para G(x), con mantisa de 48 bits
	\label{fig:G_global_48}
\end{center}

\begin{center}
	\includegraphics[width=0.96\textwidth]{punto2_mantisa40_global.jpg}
	\textbf{Gr\'afico 4}: Exponente para G(x), con mantisa de 40 bits
	\label{fig:G_global_40}
\end{center}

\begin{center}
	\includegraphics[width=0.96\textwidth]{punto2_mantisa32_global.jpg}
	\textbf{Gr\'afico 5}: Exponente para G(x), con mantisa de 32 bits
	\label{fig:G_global_32}
\end{center}

\begin{center}
	\includegraphics[width=0.96\textwidth]{punto2_mantisa24_global.jpg}
	\textbf{Gr\'afico 6}: Exponente para G(x), con mantisa de 24 bits
	\label{fig:G_global_24}
\end{center}

\begin{center}
	\includegraphics[width=0.96\textwidth]{punto2_mantisa16_global.jpg}
	\textbf{Gr\'afico 7}: Exponente para G(x), con mantisa de 16 bits
	\label{fig:G_global_16}
\end{center}

\begin{center}
	\includegraphics[width=0.96\textwidth]{punto2_mantisa8_global.jpg}
	\textbf{Gr\'afico 8}: Exponente para G(x), con mantisa de 8 bits
	\label{fig:G_global_8}
\end{center}

\subsection{}

%----------------------------------------------------------------------------------------------%
\newpage

