\section{Ap�ndice A}

\subsection{Enunciado}

\begin{centering}
\bf Laboratorio de M\'etodos Num\'ericos - Primer cuatrimestre 2007 \\
\bf Trabajo Pr\'actico N\'umero 1: Sin margen de error (num\'erico) \\
\end{centering}

\vskip 25pt
\hrule
\vskip 11pt

El objetivo del trabajo pr\'actico es analizar el comportamiento de la
aritm\'etica de punto flotante para el c\'alculo de una funci\'on sobre
el conjunto de los n\'umeros reales. Consideremos la funci\'on
$G:\real\to\real$ definida por:
\begin{eqnarray}
T(x) & = & \left\{ \begin{array}{cl}
                      1 & \hbox{si } x = 0 \\
                      \frac{e^x-1}{x} & \hbox{en caso contrario}
                    \end{array} \right. \nonumber \\
Q(x) & = & \Big|x - \sqrt{x^2+1} \Big| - \frac{1}{x+\sqrt{x^2+1}} \nonumber \\
G(x) & = & T( Q(x)^2 ) \nonumber
\end{eqnarray}
En este trabajo pr\'actico se pide realizar un an\'alisis emp\'\i rico
del comportamiento de la funci\'on $G(x)$ para distintos valores de $x$,
calculada con aritm\'etica de punto flotante. Para esto, se pide implementar
el c\'alculo de esta sucesi\'on en aritm\'etica de $t$ d\'\i gitos binarios
de precisi\'on y, sobre la base de la implementaci\'on, realizar las
siguientes mediciones:
\begin{enumerate}
\item Graficar el valor de $G(x)$ en funci\'on de $x$ para diferentes
valores de $t$. >Se observa alguna regularidad?
\item Medir el valor de $G(x)$ en funci\'on de la cantidad $t$ de d\'\i gitos
binarios de precisi\'on en la aritm\'etica de punto flotante, para diferentes
valores de $x$. >Se observa alguna influencia de la cantidad de d\'\i gitos
de precisi\'on en el c\'alculo de $G(x)$?
\end{enumerate}

>Es posible obtener el valor de $G(x)$ anal\'\i ticamente? En caso afirmativo,
>c\'omo se compara este valor exacto con la aproximaci\'on lograda con
aritm\'etica de punto flotante? En caso de que se observen diferencias
importantes, analizar las causas que generan este comportamiento.

El informe debe contener una descripci\'on detallada de las distintas
alternativas que el grupo haya considerado para la implementaci\'on de la
aritm\'etica de punto flotante de $t$ d\'\i gitos binarios de precisi\'on, junto
con una discusi\'on de estas alternativas que justifique la opci\'on
implementada. Por otra parte, se debe incluir en la secci\'on correspondiente
el c\'odigo que implementa esta aritm\'etica, junto con todos los comentarios
y decisiones relevantes acerca de esta implementaci\'on.

\vskip 15pt

\hrule

\vskip 11pt

Fecha de entrega: Lunes 9 de Abril

