\subsection{Transcurso del juego}
\subsubsection{Envio y recepci'on}
Durante el transcurso del juego se deber'an cumplir las siguientes reglas:

\begin{itemize}
\item Es mano el que pidi'o conexion.

\item Supongo que B quiere jugar $B_2$. Entonces debe enviar la firma que posee:

$$	k(B_2)^{e^2_a} [mod\ p] $$
	
Cuando A lo recibe, le aplica su desencripci'on:

$$	d^2_a(k(B_2)^{e^2_a} [mod\ p]) = 
	k(B_2)^{e^2_a * d^2_a} [mod\ p] = 
	k(B_2) $$

Y con aplicar $k$, obtiene $B_2$.

\item Cada vez que se canta o se juega una carta, se debe adjuntar un timestamp del momento del canto/juego. Luego, el paquete debe ser firmado (con $d^3_a$ / $d^3_b$ seg'un corresponda) para evitar el repudio del emisor m'as adelante (``...no no, yo no te cante, entendiste mal...'') y la reutilizaci'on del canto como prueba falsa m'as adelante (``...s'i s'i, vos cantaste truco; hace 5 horas, pero cantaste, mir'a...'').

\item El que finalize el juego (el que se vaya al mazo, el que mate la 'ultima carta del contrincante, el que gane en el falta envido) debe adem'as enviar un timestamp firmado, marc'andolo como el final oficial del juego. El oponente, aunque est'e caliente, debe confirmarle si la hora es v'alida.

\end{itemize}


\subsubsection{Formato de las jugadas}
Suponiendo que se env'ian los datos en limpio, el formato de paquete ser'ia

\begin{verbatim}
------------------------------------------------------------------
|   ts   |   comando   |   carta(opcional)   |  tanto(opcional)  |
------------------------------------------------------------------
   32 b        4 b            1024 b                  6 b
\end{verbatim}


\negrita{Timestamp / Nro de secuencia}

N'umero de secuenciamiento de paquetes. Quien comienza el juego setea el valor inicial. Se debe respetar la correlatividad, sino se aborta la conexi'on.


\negrita{Comandos}
\begin{verbatim}
00: Quiero
00: Quiero
01: No quiero
02: Envido
03: Real envido
04: Falta envido
05: Truco
06: Quiero retruco
07: Quiero vale cuatro
08: Juego carta
09: Canto tanto
10: Son buenas
11: Me voy al mazo
\end{verbatim}
