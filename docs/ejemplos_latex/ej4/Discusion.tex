\section{Discusi\'on}

% PAUTAS: Se incluir� aqu� un an�lisis de los resultados obtenidos en la secci�n anterior (se
% analizara su validez, coherencia, etc). Deben analizarse como m�nimo los items pedidos en el
% enunciado. No es aceptable decir que "los recultados fueron los esperados", sin hacer clara 
% referencia a la te�ria a la cual se ajustan. Adem�s se deben mencionar los resultados
% interesantes y los casos "patol�gicos" encontrados.


No era necesario hacer un Spline general, hubiera sido m\'as eficiente en el contexto del problema que los valores de la variable independiente sean equidistantes.

Debimos hacer un Spline de frontera sujeta en ambos extremos para poder acelerar hasta el final de la pista, en lugar de ir desacelerando hasta llegar a $0$ en ese punto.
