\section{Ap�ndice B}

\subsection{Funciones Relevantes}

\begin{scriptsize} 
Matriz spline(Matriz tablaValores, long double derivPrimerPunto) {
    // Planteo
    // 1) crear sistema de ecuaciones para obtener los valores c_j
    // 2) resolver (siempre puede resolverse en la teor�a, o hubo error de datos
    //    de entrada)
    // 3) con los valores despejados de c_j y los a_j obtenidos al principio
    //    despejar b_j y d_j

    // 1) crear sistema para obtener c_j
    // El sistema a plantear tiene esta forma:
    // - una fila 2*h_0*c_0 + h_0*c_1 = val(0) donde val(0) es la expresi�n
    //   val(0) = 3(a_1 - a_0)/h_0 - 3*derivPrimerPunto
    // - ecuaciones h_{j-1}*c_{j-1} + 2(h_{j-1}+h_{j})*c_j + h_{j}*c_{j+1} = val(j)
    // para j entre 1 y n-1 donde val(j) es 3(a_{j+1}-a_j)/h_j - 3(a_j - a_{j-1})/h_{j-1}
    // - una fila 0 0 ... 0 1 con valor de t�rmino independiente 0
    // Esta matriz es diagonal dominante - el sistema siempre tiene soluci�n
    // Este spline es de frontera sujeta en el primer segmento, y de frontera
    // libre en el �ltimo.

    assert(tablaValores.getCol() == 2);
    assert(tablaValores.getFil() >= 3); // al menos 3 filas, o no puede hacerse un spline!

    // Definici�n: n es la cantidad de segmentos entre puntos, y la cantidad de
    // ecuaciones de grado 3 que deben generarse. La tabla de valores de entrada
    // proporciona entonces n+1 puntos
    unsigned int n = tablaValores.getFil() - 1;

    // Crear las matrices para el sistema de ecuaciones
    Matriz A(n+1, n+1); // coeficientes de las variables
    Matriz b(n+1, 1); // t�rminos independientes

    // Completar las matrices del sistema lineal

	// Asignar la primera ecuaci�n
	// Frontera libre
	// A.sub(1, 1) = 1.0; // c_0 == 0 (frontera libre)
	// Frontera sujeta en el primer segmento del spline
	// - calcular h0 para la ecuaci�n en la primera fila

    // Obtener h_0 = x_{0+1} - x_0
    long double hjmas1 = tablaValores.sub(2, 1) - tablaValores.sub(1, 1); // x_1 - x_0
    assert(hjmas1 > 0); // pedimos que los x_j est�n en orden ascendente y no se repitan (!)
    // Obtener a_j - a_{j-1} con j=1  ---------> obtener a_1 - a_0
    long double Dajmas1 = tablaValores.sub(2, 2) - tablaValores.sub(1, 2);
    // Obtener 3(a_j - a_{j-1}) / h_{j-1} con j=1  ---------> obtener 3(a_1 - a_0) / h_0
    long double t3Dajmas1Dhjmas1 = (3.0*Dajmas1)/hjmas1;

	A.sub(1, 1) = 2.0*hjmas1;
	A.sub(1, 2) = hjmas1;
	b.sub(1, 1) = t3Dajmas1Dhjmas1 - 3.0*derivPrimerPunto;

	// Obtener 3(a_{j+1} - a_j) / h_j - 3(a_j - a_{j-1}) / h_{j-1} para j>=1
    for (int j = 1; j<=n-1; j++) {
        // Guardar los valores previos para esta iteraci�n
        long double prevhj = hjmas1;
        long double prevDaj = Dajmas1;
        long double prev3DajDhj = t3Dajmas1Dhjmas1;
        // Calcular nuevos valores
        hjmas1 = tablaValores.sub(j+1+1, 1) - tablaValores.sub(j+1, 1); // x_{j+1} - x_j
        assert(hjmas1 > 0); // pedimos que los x_j est�n en orden ascendente y no se repitan (!)
        Dajmas1 = tablaValores.sub(j+1+1, 2) - tablaValores.sub(j+1, 2);
        t3Dajmas1Dhjmas1 = (3.0*Dajmas1)/hjmas1;
        // Guardarlo en el vector de t�rminos independientes
        b.sub(j+1, 1) = t3Dajmas1Dhjmas1 - prev3DajDhj;
        // Guardar en la matriz asociada del sistema
        A.sub(j+1, j) = prevhj; // h_j
        A.sub(j+1, j+1) = 2.0*(prevhj+hjmas1); // 2(h_j + h_{j+1})
        A.sub(j+1, j+2) = hjmas1; // h_{j+1}
    }

	// Asignar la �ltima ecuaci�n
	// Tenemos un spline con frontera sujeta en el �ltimo segmento
	A.sub(n+1, n+1) = 1.0; // c_{n+1} == 0 (idem)
	b.sub(n+1, 1) = 0.0;
	// Resolver el sistema
	assert(A.triangular(b));
	A.despejar(b);
	// No necesito m�s la matriz A - la convierto en la matriz resultado
    // Crear la matriz con los coeficientes de cada polinomio
	A.resize(n, 5); // Devolver una matriz de n filas por 5 columnas

    for (int jmas1 = 1; jmas1 <= n; jmas1++) {
		// Asignar los valores de x_j
		A.sub(jmas1, 1) = tablaValores.sub(jmas1, 1); // definici�n de x_j
		// Asignar los valores de a_j
		// Estos valores son sencillamente f(x_j) para j entre 0 y n-1
		A.sub(jmas1, 2) = tablaValores.sub(jmas1, 2); // definici�n de a_j
		// Asignar los valores de b_j
		// Estos valores son:
		// b_j = (a_{j+1} - a_j)/h_j - h_j*((2/3)*c_j + c_{j+1}/3) para j entre 0 y n-1
		long double hj = tablaValores.sub(jmas1+1, 1) - tablaValores.sub(jmas1, 1);
		A.sub(jmas1, 3) = (tablaValores.sub(jmas1+1, 2) - tablaValores.sub(jmas1, 2))/hj;
		A.sub(jmas1, 3) -= hj*(2.0*b.sub(jmas1, 1) + b.sub(jmas1+1, 1))/3.0;
		// Asignar los valores de c_j
		// Estos valores son las inc�gnitas del sistema resuelto, y est�n en b
		A.sub(jmas1, 4) = b.sub(jmas1, 1);
		// Asignar los valores de d_j
		// d_j = (c_{j+1} - c_j) / (3.0*h_j)
		A.sub(jmas1, 5) = (b.sub(jmas1+1, 1) - b.sub(jmas1, 1)) / (3.0*hj);
	}
return A;
}
\end{scriptsize} 
 \newpage 
\begin{scriptsize} 
 bool chequearPositivoBisec(long double (*func)(long double), long double (*deriv)(long double), long double a, long double b, const long double& tolBiseccion, long double& valorNegativo, std::ostream* pOutput) {
        if (a >= b)
            return true; // intervalo incorrecto - informar que la funci�n se mantiene positiva

        long double valF = func(a);
        if (valF < 0) {
            valorNegativo = a;
            return false; // encontramos un negativo
        }

        valF = func(b);
        if (valF < 0) {
            valorNegativo = b;
            return false; // encontramos un negativo
        }

        long double valDerivA = deriv(a);
        long double valDerivB = deriv(b);
        int signDerivA = sign(valDerivA);
        int signDerivB = sign(valDerivB);
        if (signDerivA == 0) {
            a = salirDelCero(deriv, a, tolBiseccion); // omitir el punto cr�tico en la pr�xima iteraci�n
        }
        if (signDerivB == 0) {
            b = salirDelCero(deriv, b, -tolBiseccion); // omitir el punto cr�tico en la pr�xima iteraci�n

        }
        // Iterar sobre un intervalo que excluya a el o los puntos cr�ticos en los
        // bordes anteriores del intervalo
        if (signDerivA == 0 || signDerivB == 0)
            return chequearPositivoBisec(func, deriv, a, b, tolBiseccion, valorNegativo, pOutput);

        if (signDerivA != signDerivB) {
            // ra�z de la derivada en el medio! Usar bisecci�n para encontrarla
            long double nuevoa = a;
            long double nuevob = b;
            long double aproxRaiz = biseccion(deriv, nuevoa, nuevob, tolBiseccion);
            cout << "Encontre una raiz de la derivada en " << aproxRaiz << endl;
            *pOutput << "Encontre una raiz de la derivada en " << aproxRaiz << endl;
            // chequear si la funcion se hace negativa
            long double valF = func(aproxRaiz);
            if (valF < 0) {
                cout << "Encontr� un valor negativo de la funci�n en " << aproxRaiz << endl;
								*pOutput << "Encontr� un valor negativo de la funci�n en " << aproxRaiz << endl;
                valorNegativo = aproxRaiz;
                //return false; // encontramos el negativo
            }
            // buscar recursivamente en [a, nuevoa] y [nuevob, b]
            bool rv = chequearPositivoBisec(func, deriv, a, nuevoa, tolBiseccion,valorNegativo,pOutput);
            if (rv) {
                rv = chequearPositivoBisec(func, deriv, nuevob, b, tolBiseccion, valorNegativo,pOutput);
            }
            return rv;
        }
        else {
            // subdividir el intervalo [a,b]
            long double ancho = (b - a)/2.0;
            if (ancho < tolBiseccion)
                return true; // el intervalo ya era demasiado chico para subdividirlo
            long double medio = a + ancho;
            // buscar recursivamente en [a, medio] y [medio, b]
            bool rv = chequearPositivoBisec(func, deriv, a, medio, tolBiseccion, valorNegativo,pOutput);
            if (rv) {
                rv = chequearPositivoBisec(func, deriv, medio, b, tolBiseccion, valorNegativo, pOutput);
            }
            return rv;
        }
    }

\end{scriptsize} 
 \newpage 
\begin{scriptsize} 

long double biseccion(long double (*func)(long double), long double& a, long double& b, long double tolBiseccion) {
        long double valA = func(a);
        int signA = sign(valA);
        if (signA == 0) {
            // Encontr� la ra�z: es a
            return a;
        }

        while (true) {
            long double mitad = (b - a)/2.0;
            if (mitad < tolBiseccion) {
                return a + mitad;
            }
            long double p = a + mitad;
            long double valP = func(p);
            int signP = sign(valP);
            if (signA == signP)
                a = p;
            else
                b = p;
        }
    }
\end{scriptsize} 


\begin{scriptsize} 

 long double evalH(long double t) {
        return (datos.masa*G + evalV2(t)*datos.coef_carga_dinam)*datos.coef_rozamiento - datos.masa*evalA(t);
    }

\end{scriptsize} 

\begin{scriptsize} 
long double evalDiffH(long double t) {
        return (2.0*evalDiffV2(t)*datos.coef_carga_dinam*datos.coef_rozamiento - datos.masa*evalDiffA(t));
    }
\end{scriptsize} 
 \newpage 
\begin{scriptsize} 

	void calcular(ofstream& salida) {
		datos.spline_x = spline(datos.matriz_x, datos.dir_inicial_x);
		datos.spline_y = spline(datos.matriz_y, datos.dir_inicial_y);
		pOut = &salida;
    // Buscar ceros de la derivada y evaluar la funci�n en los puntos cr�ticos
		// Tambi�n verificar el signo de la funci�n H en los extremos del intervalo
    long double tolBiseccion = 0.0001;
    long double a, b, valorNegativo = 0.0;
  	bool esPositivo = true;
  
    b = datos.matriz_x.sub(1, 1);
    for (int i = 2; i <= datos.matriz_x.getFil(); i++) {
			a = b;
			b = datos.matriz_x.sub(i, 1);
			esPositivo = calc::chequearPositivoBisec(evalH, evalDiffH, a, b, tolBiseccion, valorNegativo, pOut);
			if (!esPositivo)
		    	break;
		}
		if (!esPositivo) {
			cout << "SE PUDRE TODO en t = " << valorNegativo << endl;
			cout << "Ese punto cae dentro del segmento numero " << valorNegativo / datos.omega + 1.0 << endl;
			salida << "SE PUDRE TODO en t = " << valorNegativo << endl;
			salida << "Ese punto cae dentro del segmento numero " << valorNegativo / datos.omega + 1.0 << endl;
			// Sugerir un punto
			DatosTP4 orig = datos;
			int problema = (int) ( 1.1 + valorNegativo / datos.omega );

			datos.cant_muestras = problema + 1;
			// Definir una matriz de problema + 1 puntos, los primeros problema - 1 son iguales
			Matriz mX(problema + 1, 2);
			Matriz mY(problema + 1, 2);
			for (int i = 1; i <= problema + 1; i++) {
				mX.sub(i, 1) = mY.sub(i, 1) = datos.omega * (i-1);
				if (i < problema) {
					mX.sub(i, 2) = datos.matriz_x.sub(i, 2);
					mY.sub(i, 2) = datos.matriz_y.sub(i, 2);
				}
			}
			datos.matriz_x = mX;
			datos.matriz_y = mY;
			// Hay que calcular los puntos problema y problema + 1
			// 1era condici�n: ambos sobre la trayectoria original
			// 2da condici�n: el punto problema debe hacer que la funci�n H no reviente
			// 3ra condici�n: el punto problema + 1 debe hacer que el vector aceleraci�n en
			// el punto problema no tenga mucho valor absoluto
			datos = orig;
		}
        mostrarGraficos();
 }
\end{scriptsize}
