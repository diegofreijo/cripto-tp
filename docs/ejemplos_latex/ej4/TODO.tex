\section{Cosas por hacer}

A medida que las vamos terminando las podemos ir borrando de aqui o comentando %asi%

\subsection{TP 4}

\begin{itemize}

\item Optimizci�n de trayectorias. Desarrollar y probar. Existen dos alternativas:

	 A. Ir cambiando el valor de omega y ver en que puno muere y ajustar los puntos en funci�n 
	    del error cometido en el derrape.
	
	 B. Fijar un omega e ir moviendo los puntos.

	Se fij� un omega y se fue cambiando la ubicaci�n de los puntos para mantener la
	forma de la trayectoria, pero haciendo que la distancia recorrida sea mayor
	o menor en cada punto.
	

\item Completar la parte de desarrollo. Nuestro Splines y nuestro calculo del m�nimo. 

\item Recomendaciones validas para analisis de instancias.

	 A. Graficar la velocidad y la aceleracion en funcion del tiempo juntas.

\item Analizar la instancia provista por al catedra.

\end{itemize}


\subsection{Recuperatorio TP 4}

\begin{itemize}

\item Criterio de selecion del epsilon que representa la longitud minima de los intervalos
dentro de los cuales se analiza la existencia de raices del polinomio proveniente de g(t).

\item Recomendaciones validas para analisis de instancias.

	 B. Mostrar una animacion en pantalla de la trayectoria del auto, junto con su vector  
	    velocidad y fuerza de frenado, para ver como evoluciona a medida que recorre la pista.
 
\end{itemize}