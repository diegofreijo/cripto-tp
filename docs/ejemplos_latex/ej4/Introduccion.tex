\section{Introducci\'on}


\subsection{Interpolaci\'on}

La interpolaci\'on consiste en obtener una funci�n que corresponda a una serie de datos conocidos.
Una de las clases de funciones m\'as \'utiles y mejor conocidas es la de los polinomios algebraicos, es decir el conjunto de funciones de la forma:

$$P(x) = a_n x^n + a_{n-1}x^{n-1} + ... + a_1 x + a_0$$

donde n es un entero no negativo y $a_n, a_{n-1}, ..., a_1,a_0$ son constantes reales.
Una de las razones importantes por la cual se debe considerar esta clase de polinomios en la interpolaci\'on de funciones, es que la derivada y la integral de un polinomio son f\'aciles de determinar y tambi\'en son polinomios. Por estas razones resulta conveniente en muchos casos aproximar una funci\'on mediante un polinomio que coincida con la funci\'on en un conjunto finito de puntos $x_0$ ... $x_n$.

\subsection{Interpolaci\'on Polin\'omica Segmentaria}

Un procedimiento alterno consiste en dividir el intervalo en una serie de subintervalos y en cada uno construir un polinomio (generalmente) diferente. A esta forma se le conoce como aproximaci\'on polin\'omica segmentaria.
La aproximaci\'on polin\'omica segmentaria m\'as simple es la interpolaci\'on lineal segmentaria que consiste en unir la serie de puntos dados por medio de una poligonal.

\subsection{Splines C\'ubicos}

La aproximaci\'on polin\'omica segmentaria m\'as com\'un utiliza polinomios de grado tres entre cada par de puntos consecutivos y recibe el nombre de interpolaci\'on de trazadores c\'ubicos o Splines c\'ubicos.

Definici\'on: Dada una funci\'on f definida en $[a,b]$ y un conjunto de puntos 
$a = x_0 < x_1 < ... < x_n = b$ un interpolante de trazador c\'ubico S para f es una funci\'on que cumple con las siguientes condiciones:

\begin{enumerate}
\item $S(x)$ es un polinomio c�bico denotado por $S_j(x)$ en el intervalo 
$[x_j, x_{j+1}]$ para cada $j = 0, 1, 2, ..., n-1$
\item $S(x_j) = f(x_j)$ para cada $j = 0, 1, 2, ..., n$
\item $S_j(x_{j+1}) = S_{j+1}$ para cada $j = 0, 1, 2,..., n-2$ (lo que asegura la continuidad)
\item $S'_j(x_{j+1}) = S'_{j+1}(x_{j+1})$ para cada $j = 0, 1, 2,..., n-2$ (lo que asegura diferenciabilidad en los puntos)
\item $S''_j(x_{j+1}) = S''_{j+1}(x_{j+1})$ para cada $j = 0, 1, 2,..., n-2$ (lo que asegura que no hay cambios de concavidad en los nodos o puntos)
\end{enumerate}

Tambi\'en deben satisfacerse uno de las siguientes conjuntos de condiciones en la frontera:

\begin{enumerate}
\item $S''(x_0) = S''(x_n) = 0$ (frontera libre o natural)
\item	$S'(x_0) = f'(x_0) y S'(x_n) = f'(x_n)$ (frontera sujeta)
\end{enumerate}

En t\'erminos generales, con las condiciones de frontera sujeta se logran aproximaciones m\'as exactas, ya que abarca mayor informaci\'on acerca de la funci\'on. Pero para que se cumpla este tipo de condici\'on, se requiere tener los valores de la derivada en los extremos o bien una aproximaci\'on precisa de ellos. Si se desea construir el conjunto de polinomios de la interpolaci\'on de trazador c\'ubico de una determinada funci\'on $f$, se van aplicando cada una de las condiciones de la definici\'on a un polinomio c\'ubico general.

\subsection{Ra\'ices de una Funci\'on}

Para una funci\'on continua sobre un intervalo $[a, b]$, si $f(a)$ y $f(b)$ tienen signos opuestos, sabemos por el Teorema de Valor Medio la funci\'on alcanza el valor $0$ dentro del intervalo $(a, b)$.
Esta propiedad es utilizada por los algoritmos de b\'usqueda de ra\'ices en funciones, como en el caso de estos algoritmos cl\'asicos:
\begin{enumerate}
\item M\'etodo de la Bisecci\'on
\item	M\'etodo de Newton
\end{enumerate}	

\subsection{M\'etodo de la Bisecci\'on}

Dado un intervalo cerrado $[a, b]$ y una funci\'on continua $f$ que cambia de signo en los extremos del intervalo, este m\'etodo acota la ra\'iz en un intervalo peque\~no (seg\'un se desee).
El algoritmo comienza con el intervalo conocido; lo divide a la mitad y eval\'ua el valor central. Si no se encontr\'o la ra\'iz, el valor tiene el mismo signo que uno de los extremos; quedarse con el subintervalo donde existe el cambio de signo (por lo tanto, contiene la ra\'iz) y repetir hasta que el subintervalo sea lo suficientemente peque\~no.

\subsection{M\'etodo de Newton}

Requiere un punto $x$ que se sabe cercano a una ra\'iz de la funci\'on continua.
Iniciar en el mismo punto $x$ del eje $x$. Luego ajustar una l\'inea tangente a $f(x)$ en $x$ obtener la soluci\'on para el punto y en la intersecci\'on de la tangente con el eje $x$ repetir el proceso hasta que $|x-y| < epsilon$, y devolver $y$.

%-------------------------------------------------------------------------------------------%