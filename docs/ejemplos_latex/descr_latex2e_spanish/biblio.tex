
\begin{thebibliography}{99}
\addcontentsline{toc}{chapter}{\numberline{}\bibname}
\ifx\pdfoutput\undefined % No se est'a ejecutando pdftex
\else
\pdfbookmark{\bibname}{bibliografia}
\fi

\bibitem{manual} Leslie Lamport.  \newblock \emph{{\LaTeX:} A Document
    Preparation System}.  \newblock Addison-Wesley, Reading,
  Massachusetts, segunda edici'on, 1994, ISBN~0-201-52983-1.

\bibitem{texbook} Donald~E. Knuth.  \newblock \textit{The \TeX{}book,}
  Tomo~A de \textit{Computers and Typesetting}, Addison-Wesley
  Publishing Company (1984), ISBN~0-201-13448-9.

\bibitem{companion} Michel Goossens, Frank Mittelbach and Alexander
  Samarin.  \newblock \emph{The {\LaTeX} Companion}.  \newblock
  Addison-Wesley, Reading, Massachusetts, 1994, ISBN~0-201-54199-8.
 
\bibitem{local} Cada instalaci'on de  \LaTeX{} deber'ia proporcionar
  la llamada \emph{Gu'ia Local de \LaTeX}, que explica las cosas que
  son particulares del sistema local. Deber'ia residir en un fichero
  llamado \texttt{local.tex}. Por desgracia, en algunos sitios no se
  halla dicha gu'ia. En este caso, p'idale ayuda a un experto de
  \LaTeX.
 
\bibitem{usrguide} \LaTeX3 Project Team.  \newblock \emph{\LaTeXe~for
    authors}.  \newblock Viene con la distribuci'on de \LaTeXe{} como
  \texttt{usrguide.tex}.

\bibitem{clsguide} \LaTeX3 Project Team.  \newblock \emph{\LaTeXe~for
    Class and Package writers}.  \newblock Viene con la distribuci'on
    de \LaTeXe{} como \texttt{clsguide.tex}.

\bibitem{fntguide} \LaTeX3 Project Team.  \newblock \emph{\LaTeXe~Font
    selection}.  \newblock Se incluye en la distribuci'on de \LaTeXe{}
    como \texttt{fntguide.tex}.

\bibitem{graphics} D.~P.~Carlisle.  \newblock \emph{Packages in the
    `graphics'\ bundle}.  \newblock Se incluye en el conjunto `graphics'\
  como \texttt{grfguide.tex}, disponible en el mismo sitio de donde se
  ha tomado la distribuci'on de \LaTeX.
\end{thebibliography}
