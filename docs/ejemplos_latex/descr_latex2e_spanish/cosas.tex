\chapter{Lo que necesita saber}
\begin{intro}
En la primera parte de este cap'itulo tendr'a una visi'on general de
la filosof'ia e historia de \textrm{\LaTeXe}. La segunda parte incide
en las estructuras b'asicas de un documento de \LaTeX. Tras
leer este cap'itulo, tendr'a un conocimiento b'asico del modo de
funcionamiento de \LaTeX. Cuando contin'ue leyendo, la informaci'on
del presente cap'itulo le ayudar'a a integrar toda la informaci'on
adicional que pueda obtener sobre \LaTeX, tanto en cap'itulos
posteriores como de otros sitios.
\end{intro}

\section{El nombre del juego}

\subsection{\TeX}
 
\TeX\ es un programa de ordenador de Donald E.~Knuth~\cite{texbook}.
Est'a orientado a la composici'on e impresi'on textos y f'ormulas
matem'aticas.

\TeX{} se pronuncia ``Tech'', con una ``ch'' como en la palabra
alemana ``Buch'' o en la escocesa ``Loch''. Este es el sonido de una
`h'\ aspirada, como en la onomatopeya ``argh''. En un entorno
\texttt{ASCII} \TeX{} se escribe \texttt{TeX}.

% ``Buch'', --libro--, casi seguro que se conoce m'as que ``Ach''.
% En las Canarias, Andaluc'ia (?) y en toda Latinoam'erica este sonido
% es m'as utilizado que la cl'asica `j' castellana, que es bastante
% m'as fuerte.

\subsection{\LaTeX}
 
\LaTeX\ es un paquete de macros que le permite al autor de un texto
componer e imprimir su documento con la mayor calidad tipogr'afica,
empleando para ello patrones previamente definidos. Originalmente,
\LaTeX{} fue escrito por \index{Lamport, Leslie}Leslie
Lamport~\cite{manual}. Utiliza el cajista \TeX{} como su elemento de
composici'on.

Desde diciembre de 1994, el paquete \LaTeX{} est'a siendo actualizado
por el equipo \index{LaTeX3@\LaTeX 3}\LaTeX 3, que dirige por
\index{Mittelbach, Frank}Frank Mittelbach, para incluir algunas de las
mejoras que se hab'ian solicitado desde hace tiempo, y para reunificar
todas las versiones retocadas
% Ya, ya s'e que entre nosotros decimos ``parcheadas''
que han surgido desde que apareciera \index{LaTeX 2.09@\LaTeX{}
  2.09}\LaTeX{} 2.09 hace ya algunos a~nos. Para distinnguir la nueva
versi'on de la vieja se le llama \index{LaTeX 2e@\LaTeXe}\LaTeXe. Este
documento trata sobre \LaTeXe.

\LaTeX{} se pronuncia ``Lei-tegh'', aunque entre los
hispanohablantes se ha aceptado ``La-tegh''. Para referirnos a
\LaTeX{} en un entorno \texttt{ASCII} escribiremos \texttt{LaTeX}.
\LaTeXe{} se pronuncia ``Lei-tegh tu 'ii'' ---aunque muchos nos
empe~namos en leer ``Lategh dos e''--- y se puede escribir
\texttt{LaTeX2e}.

\subsection{Conceptos b'asicos}
 
\subsubsection{Autor, dise~nador y cajista}
 
Normalmente, para una publicaci'on el autor le entrega a una editorial
un escrito a m'aquina. El dise~nador de libros de la editorial
decide entonces sobre el formato del documento (longitud de los
renglones, tipo de letra, espacios antes y despu'es de cada cap'itulo,
etc.)\ y le da estas instrucciones al cajista para producir este
formato.

Un dise~nador de libros humano intenta averiguar las intenciones del
autor mientras ha realizado el escrito. Entonces decide sobre el modo
de presentar los t'itulos de cap'itulos, citas, ejemplos, f'ormulas,
etc.,\ bas'andose en su saber profesional y sobre el contenido del
escrito.

En un entorno de \LaTeX, \LaTeX{} realiza el papel del dise~nador de
libros y emplea a \TeX{} como cajista. Pero \LaTeX{} \emph{s'olo} es
un programa y, por tanto, necesita m'as ayuda para sus decisiones que
un dise~nador humano de libros. El autor tiene que proporcionar
informaci'on adicional que describa la estructura l'ogica del texto.
Esta informaci'on se indica dentro del texto a trav'es de las
\emph{instrucciones} u \emph{'ordenes} de \LaTeX.

Esto es bastante diferente del enfoque \wi{WYSIWYG}\footnote{Siglas
  que significan \emph{What you see is what you get,} lo que ve es lo
  que obtendr'a.} de la mayor'ia de los procesadores de textos tales
como \emph{Microsoft Word} o \emph{WordPerfect}. Con estas
aplicaciones, el autor establece el formato del texto con la entrada
interactiva al introducirlo en el ordenador. En cada momento, el autor
ver� en pantalla el aspecto que tendr� el trabajo final cuando lo
imprima.
% En algunos casos no tan ``exactamente'', pero bueno...

Por regla general, al emplear \LaTeX{} el autor no ve, al introducir
el texto, c'omo va a resultar la composici'on final que resultar'a.
Sin embargo, existen herramientas que permiten mostrar en pantalla lo
que finalmente se obtiene de haber procesado sus ficheros con \LaTeX.
Con ellas se pueden realizar correcciones antes de enviar el documento
a la impresora.

\subsubsection{Dise~no del formato}
 
El dise~no tipogr'afico es una artesan'ia que se debe aprender.  Los
autores inexpertos con frecuencia cometen graves errores de dise'no.
Muchos profanos creen err'oneamente que el dise'no tipogr'afico es,
ante todo, una cuesti'on de est'etica: si el documento presenta un
buen aspecto desde el punto de vista art'istico, entonces est'a bien
``dise~nado''. Sin embargo, ya que los documentos se van a leer y no a
colgarse en un museo, es m'as importante una mayor legibilidad y una
comprensi'on mejor que un aspecto m'as agradable.

Por ejemplo:
\begin{itemize}
\item Se debe elegir el tama~no de las letras y la numeraci'on de los
  t'itulos de modo que la estructura de los cap'itulos y las secciones
  sea f'acilmente reconocible.
\item Se debe elegir la longitud de los renglones de modo que se evite
  el movimiento fatigoso de los ojos del lector y no para que
  rellenen, a ser posible, las p'aginas con un aspecto est'eticamente
  bueno.
\end{itemize}

Con los sistemas \wi{WYSIWYG} los autores producen, en general,
documentos est'eticamente bonitos pero con una estructura muy escasa o
inconsistente. \LaTeX{} evita estos errores de formato, ya que con
\LaTeX{} el autor est'a obligado a indicar la estructura
\emph{l'ogica} del texto. Entonces \LaTeX{} elige el formato m'as
apropiado para 'este.


\subsubsection{Ventajas e inconvenientes}

Una cuesti'on que se discute a menudo cuando la gente del mundo
\mbox{\wi{WYSIWYG}} se encuentra con la gente que utiliza \LaTeX{} es
sobre ``las \wi{ventajas de \LaTeX} sobre un procesador de textos
normal'' o al rev'es. Cuando comienza una discusi'on como 'esta, lo
mejor que se puede hacer es mantener una postura de asentimiento, ya
que las cosas se suelen salir de control. Pero a veces no se puede
huir\ldots

\medskip\noindent Las principales vetajas de \LaTeX\ sobre los
procesadores de texto normales son las siguientes:

\begin{itemize}
\item Existe mayor cantidad de dise'nos de texto profesionales a
  disposici'on, con los que realmente se pueden crear documentos
  como si fueran ``de imprenta''.

\item Se facilita la composici'on de f'ormulas con un cuidado
  especial.

\item El usuario s'olo necesita introducir instrucciones sencillas de
  entender con las que se indica la estructura del documento.  Casi
  nunca hace falta preocuparse por los detalles de creaci'on con
  t'ecnicas de impresi'on.
  
\item Tambi'en las estructuras complejas como notas a pie de p'agina,
  bibliograf'ia, 'indices, tablas y muchas otras se pueden producir
  sin gran esfuerzo.
  
\item Existen paquetes adicionales sin coste alguno para muchas tareas
  tipogr'aficas que no se facilitan directamente por el \LaTeX{}
  b'asico. Por ejemplo, existen paquetes para incluir gr'aficos en
  formato \mbox{\textsc{PostScript}} o para componer bibliograf'ias
  conforme a determinadas normas. Muchos de estos paquetes se
  describen en \companion.

\item \LaTeX{} hace que los autores tiendan a escribir textos bien
  estructurados porque as'i es como trabaja \LaTeX, o sea,
  indicando su estructura.

\item \TeX, la m'aquina de composici'on de \LaTeXe, es altamente
  portable y gratis. Por esto, el sistema funciona pr'acticamente en
  cualquier en cualquier plataforma.

\end{itemize}

\LaTeX\ tiene, naturalmente, tambi'en inconvenientes:

\begin{itemize}

\item Para hacer funcionar un sistema de \LaTeX, se necesitan m'as
  recursos (memoria, espacio de disco y potencia de procesamiento, y
  espacio de almacenamiento) que para un procesador de texto simple.
  Pero las cosas van siendo cada vez mejores, y \emph{Word for Windows
    6.0} necesita cada vez m'as espacio de disco que un sistema de
  \LaTeX{} normal. Cuando analizamos el uso del procesador, podemos
  ver que \LaTeX{} supera en prestaciones cualquier sistema
  \wi{WYSIWYG} ya que necesita mucha cantidad de CPU pero 'unicamente
  cuando el documento se procesa, mientras que los paquetes
  \wi{WYSIWYG} tienen ocupada la CPU continuamente.

\item Si bien se pueden ajustar algunos par'ametros de un dise~no de
  documento predefinido, la creaci'on de un dise~no entero es dif'icil
  y lleva mucho tiempo\footnote{Los rumores dicen que este es uno de
    los puntos claves sobre el que se har'a hincapi'e en el pr'oximo
    sistema LaTeX 3.}.\index{LaTeX3@\LaTeX 3}

\end{itemize}

\section{Ficheros de entrada de \LaTeX}

La entrada para \LaTeX{} es un fichero de texto en formato
\texttt{ASCII}. Se puede crear con cualquier editor de textos.
Contiene tanto el texto que se debe imprimir como las
``instrucciones'', con las cuales \LaTeX\ interpreta c'omo debe
disponer el texto.

\subsection{Signos de espacio}

Los caracteres ``invisibles'', como el espacio en blanco, el tabulador
y el final de l'inea, son tratados por \LaTeX\ como signos de
\wi{espacio} propiamente dichos. \emph{Varios} espacios seguidos se
tratan como \emph{un} \wi{espacio en blanco}. Generalmente, un espacio
en blanco al comienzo de una l'inea se ignora, y \emph{varios}
renglones en blanco se tratan como un rengl'on en blanco.
\index{espacio en blanco!al comienzo de una l'inea}

Un rengl'on en blanco entre dos l'ineas de texto definen el final de
un p'arrafo. \emph{Varias} l'ineas en blanco se tratan como \emph{una
  sola} l'inea en blanco. El texto que mostramos a continuaci'on es un
ejemplo. A la derecha se encuentra el texto del fichero de entrada y a
la izquierda la salida formateada.

\begin{example}
No importa si introduce
varios        espacios tras
una palabra.

Con una l'inea vac'ia se empieza un
nuevo p'arrafo.
\end{example}
 
\subsection{Caracteres especiales}

Los s'imbolos siguientes son \wi{caracteres reservados} que tienen un
significado especial para \LaTeX{} o que no est'an disponibles en
todos los tipos. Si los introduce en su fichero directamente es muy
probable que no se impriman o que fuercen a \LaTeX{} a hacer cosas que
Vd.\ no desea.
%
\begin{code}
\verb.$ & % # _ { }  ~  ^  \ . 
\end{code}

Como puede ver, estos caracteres se pueden incluir en sus documentos
anteponiendo el car'acter \verb|\| (\emph{barra invertida}):
%
\begin{example}
\$ \& \% \# \_ \{ \}
\end{example}

Los restantes s'imbolos y otros muchos caracteres especiales se pueden
imprimir en f'ormulas matem'aticas o como acentos con 'ordenes
espec'ificas.

\subsection{Las 'ordenes de \LaTeX{}}

En las 'ordenes\index{ordenes@'ordenes} de \LaTeX{} se distinguen las
letras may'usculas y las min'usculas. Toman uno de los dos formatos
siguientes:

\begin{itemize}
\item Comienzan con una \emph{barra invertida}\index{barra invertida}
  \verb|\|\cih{\bs} y tienen un nombre compuesto s'olo por letras. Los
    nombres de las 'ordenes acaban con uno o m'as espacios en blanco,
    un car'acter especial o una cifra.
\item Se compone de una \emph{barra invertida} y un car'acter
  especial.
\end{itemize}

%
% No se tiene en cuenta \\*
%

%
% ?`Puede \3 ser una instrucci'on v'alida? (jacoboni)
% En ``LaTeX: eine Einf"uhrung'', de Helmut Kopka, 1988 existe
% en la p'agina 13 una posible definici'on de este comando.
% Esto es para LaTeX 2.09, pero funcionar'ia con LaTeX2e?
% (Tom'as Bautista)
%


\LaTeX{} ignora los espacios en blanco que van tras las 'ordenes. Si
se desea introducir \index{espacio en blanco!tras instrucci'on} un
espacio en blanco tras una instrucci'on, se debe poner o bien
\verb|{}| y un espacio, o bien una instrucci'on de espaciado despu'es
de la orden. Con \verb|{}| se fuerza a \LaTeX{} a dejar de ignorar el
resto de espacios que se encuentren despu'es de la instrucci'on.


\begin{example}
He le'ido que Knuth distingue a la
gente que trabaja con \TeX{} en
\TeX{}nicos y \TeX pertos.\\
Hoy es \today.
\end{example}

Algunas instrucciones necesitan un \wi{par'ametro} que se debe poner
entre \wi{llaves} \verb|{ }| tras la instrucci'on. Otras 'ordenes
pueden llevar \wi{par'ametros opcionales} que se a~naden a la
instrucci'on entre \wi{corchetes}~\verb|[ ]| o no. El siguiente
ejemplo usa algunas 'ordenes de \LaTeX{} que explicaremos m'as
adelante.

\begin{example}
!`Te puedes \textsl{apoyar} en m'i!
\end{example}
\begin{example}
!`Por favor, comienza una nueva
l'inea justamente aqu'i!%
\linebreak[3] Gracias.
\end{example}

\subsection{Comentarios}
\index{comentarios}

Cuando \LaTeX{} encuentra un car'acter \verb|%| mientras procesa un
fichero de entrada, ignora el resto de la l'inea. Esto suele ser 'util
para introducir notas en el fichero de entrada que no se mostrar'an en
la versi'on impresa.

\begin{example}
Esto es un % tonto
% Mejor: instructivo <----
ejemplo.
\end{example}

Esto a veces puede resultar 'util cuando nos encontramos con l'ineas
demasiado largas en el fichero fuente. Si no quisi'esemos introducir
un espacio entre dos palabras, y perferimos tener dos renglones,
entonces el signo \verb|%| debe ir justo al final del rengl'on pero
pegado al 'ultimo car'acter. De este modo comentamos el car'acter de
``salto de l'inea'', que de otro modo se hubiese tratado como un
espacio en blanco.

\begin{example}
Este es otro ejem% y
% ahora el resto
plo.
\end{example}

\section{Estructura de un fichero de entrada}

Cuando \LaTeXe{} procesa un fichero de entrada, espera de 'el que siga
una determinada \wi{estructura}. Todo fichero de entrada debe comenzar
con la orden
\begin{code}
\verb|\documentclass{...}|
\end{code}
Esto indica qu'e tipo de documento es el que se pretende crear. Tras
esto, se pueden incluir 'ordenes que influir'an sobre el estilo del
documento entero, o puede cargar \wi{paquete}s que a~nadir'an nuevas
propiedades al sistema de \LaTeX. Para cargar uno de estos paquetes se
usar'a la instrucci'on
\begin{code}
\verb|\usepackage{...}|
\end{code}

Cuando todo el trabajo de configuraci'on est'e realizado\footnote{El
  'area entre \texttt{\bs documentclass} y \texttt{\bs
    begin$\mathtt{\{}$document$\mathtt{\}}$} se llama
  \emph{\wi{pre'ambulo}}.} entonces comienza el cuerpo del texto con
la instrucci'on
\begin{code}
\verb|\begin{document}|
\end{code}

A partir de entonces se introducir'a el texto mezclado con algunas
instrucciones 'utiles de \LaTeX. Al finalizar el documento debe
ponerse la orden
\begin{code}
\verb|\end{document}|
\end{code}
LaTeX{} ingorar'a cualquier cosa que se ponga tras esta instrucci'on.

La figura~\ref{mini} muestra el contenido m'inimo de un fichero de
\LaTeXe. En la figura~\ref{document} se expone un \wi{fichero de
  entrada} algo m'as complejo.

\begin{figure}[!bp]
\begin{lined}{6cm}
\begin{verbatim}
\documentclass{article}
\begin{document}
Lo peque~no es bello.
\end{document}
\end{verbatim}
\end{lined}
\caption{Un fichero m'inimo de \LaTeX} \label{mini}
\end{figure}
 
\begin{figure}[!bp]
\begin{lined}{10cm}
\begin{verbatim}
\documentclass[a4paper,11pt]{article}
\usepackage{latexsym}
\usepackage[activeacute,spanish]{babel}
\author{H.~Partl}
\title{Minimizando}
\frenchspacing
\begin{document}
\maketitle
\tableofcontents
\section{Inicio}
Bien\ldots{} y aqu'i comienza mi art'iculo tan
estupendo.
\section{Fin}
\ldots{} y aqu'i acaba.
\end{document}
\end{verbatim}
\end{lined}
\caption{Ejemplo para un art'iculo cient'ifico en espa~nol.} \label{document}
\end{figure}
 
\section{El formato del documento}
 
\subsection{Clases de documentos}\label{sec:documentclass}

Cuando procesa un fichero de entrada, lo primero que necesita saber
\LaTeX{} es el tipo de documento que el autor quiere crear. Esto se
indica con la instrucci'on \ci{documentclass}.
\begin{command}
\ci{documentclass}\verb|[|\emph{opciones}\verb|]{|\emph{clase}\verb|}|
\end{command}
\noindent En este caso, la \emph{clase} indica el tipo de documento
que se crear'a. En la tabla~\ref{documentclasses} se muestran las
clases de documento que se explican en esta introducci'on. La
distribuci'on de \LaTeXe{} proporciona m'as clases para otros
documentos, como cartas y transparencias. El par'ametro de
\emph{\wi{opciones}} personaliza el comportamiento de la clase de
documento elegida. Las opciones se deben separar con comas. En la
tabla~\ref{options} se indican las opciones m'as comunes de las clases
de documento est'andares.


\begin{table}[!bp]
\caption{Clases de documentos} \label{documentclasses}
\begin{lined}{12cm}
\begin{description}
 
\item [\normalfont\texttt{article}] para art'iculos de revistas
  especializadas, ponencias, trabajos de pr'acticas de formaci'on,
  trabajos de seminarios, informes peque'nos, solicitudes,
  dict'amenes, descripciones de programas, invitaciones y muchos
  otros.\index{articulo@art'iculo}%
\index{clase \texttt{article}@clase article}
\item [\normalfont\texttt{report}] para informes mayores que constan
  de m'as de un cap'itulo, proyectos fin de carrera, tesis doctorales,
  libros peque'nos, disertaciones, guiones y
  similares.\index{informe}\index{clase \texttt{report}@clase report}
\item [\normalfont\texttt{book}] para libros de
  verdad\index{libro}\index{clase \texttt{book}@clase book}
\item [\normalfont\texttt{slide}] para transparencias. Esta clase emplea
  tipos grandes \textsf{sans serif}.
  \index{transparencias}\index{clase \texttt{slide}@clase slide}
\end{description}
\end{lined}
\end{table}

\begin{table}[!bp]
\caption{Opciones de clases de documento} \label{options}
\begin{lined}{12cm}
\begin{flushleft}
\begin{description}
\item[\normalfont\texttt{10pt}, \texttt{11pt}, \texttt{12pt}] \quad
  Establecen el tama~no (cuerpo) para los tipos. Si no se especifica
  ninguna opci'on, se toma \texttt{10pt}.\index{tama~no de los
    tipos!del documento}
\item[\normalfont\texttt{a4paper}, \texttt{letterpaper}, \ldots] \quad
  Define el tama~no del papel. Si no se indica nada, se toma
  \texttt{letterpaper}. Aparte de 'este se puede elegir
  \texttt{a5paper}, \texttt{b5paper}, \texttt{executivepaper} y
  \texttt{legalpaper}. \index{papel legal} \index{tama~no del
    papel}\index{papel DIN-A4} \index{papel de carta} \index{papel DIN-A5}\
  \index{papel DIN-B5} \index{papel ejecutivo}

\item[\normalfont\texttt{fleqn}] \quad Dispone las ecuaciones hacia la
  izquierda en vez de centradas.

\item[\normalfont\texttt{leqno}] \quad Coloca el n'umero de las
  ecuaciones a la izquierda en vez de a la derecha.

\item[\normalfont\texttt{titlepage}, \texttt{notitlepage}] \quad
  Indica si se debe comenzar una p'agina nueva tras el \wi{t'itulo del
    documento} o no. Si no se indica otra cosa, la clase
  \texttt{article} no comienza una p'agina nueva, mientras que \texttt{report}
  y \texttt{book} s'i.\index{titlepage@\texttt{titlepage}}

\item[\normalfont\texttt{twocolumn}] \quad Le dice a \LaTeX{} que
  componga el documento en \wi{dos columnas}.
  
\item[\normalfont\texttt{twoside, oneside}] \quad Especifica si se
  debe generar el documento a una o a dos caras.  En caso de no
  indicarse otra cosa, las clases \texttt{article} y \texttt{report}
  son a una cara y la clase \texttt{book} es a dos.
  
\item[\normalfont\texttt{openright, openany}] \quad Hace que los
  cap'itulos comienzen o bien s'olo en p'aginas a la derecha, o bien
  en la pr'oxima que est'e disponible. Esto no funciona con la clase
  \texttt{article}, ya que en esta clase no existen cap'itulos. De
  modo predeterminado, la clase \texttt{report} comienza los
  cap'itulos en la pr'oxima p'agina disponible y la clase
  \texttt{book} las comienza en las p'aginas a la derecha.
\end{description}
\end{flushleft}
\end{lined}
\end{table}

Por ejemplo: un fichero de entrada para un documento de \LaTeX{}
podr'ia comenzar con
\begin{code}
\ci{documentclass}\verb|[11pt,twoside,a4paper]{article}|
\end{code}
Esto le indica a \LaTeX{} que componga el documento como un
\emph{art'iculo} utilizando tipos del cuerpo 11, y que produzca un
formato para impresi'on a \emph{doble cara} en \emph{papel DIN-A4}.

\pagebreak[2]
\subsection{Paquetes}
\index{paquete} Mientras escribe su documento, probablemente se
encontrar'a en situaciones donde el \LaTeX{} b'asico no basta para
solucionar su problema. Si desea incluir \wi{gr'aficos}, \wi{texto en
  color} o el c'odigo fuente de un fichero, necesita mejorar las
capacidades de \LaTeX. Tales mejoras se realizan con ayuda de los llamados
\emph{paquetes.} Los paquetes se activan con la orden
\begin{command}
\ci{usepackage}\verb|[|\emph{opciones}\verb|]{|\emph{paquete}\verb|}|
\end{command}
\noindent donde \emph{paquete} es el nombre del paquete y
\emph{opciones} es una lista palabras clave que activan funciones
especiales del paquete, a las que \LaTeX{} les a~nade las opciones que
previamente se hayan indicado en la orden \verb|\documentclass|.
Algunos paquetes vienen con la distribuci'on b'asica de \LaTeXe{}
(v'ease la tabla~\ref{packages}). Otros se proporcionan por separado.
En la \guia{} puede encontrar m'as informaci'on sobre los paquetes
disponibles en su instalaci'on local. La fuente principal de
informaci'on sobre \LaTeX{} es \companion. Contiene descripciones de
cientos de paquetes, as'i como informaci'on sobre c'omo escribir sus
propias extensiones a \LaTeXe.

\begin{table}[!hbp]
\caption{Algunos paquetes distribuidos con \LaTeX} \label{packages}
\begin{lined}{11cm}
\begin{description}
\item[\normalfont\pai{doc}] Permite la documentaci'on de paquetes y
  otros ficheros de \LaTeX.\\ Se describe en \texttt{doc.dtx} y en
  \companion.

\item[\normalfont\pai{exscale}] Proporciona versiones escaladas de
  los tipos adicionales para matem'aticas.\\ 
 Descrito en \texttt{ltexscale.dtx}.

\item[\normalfont\pai{fontenc}] Especifica qu'e \wi{codificaci'on de
    tipo} debe usar \LaTeX.\\ Descrito en \texttt{ltoutenc.dtx}.

\item[\normalfont\pai{ifthen}] Proporciona instrucciones de la forma\\
  `si\ldots{} entonces\ldots{} si no\ldots'\\ Descrito en
  \texttt{ifthen.dtx} y en \companion.

\item[\normalfont\pai{latexsym}] Para que \LaTeX{} acceda al tipo
  de s'imbolos, se debe usar el paquete \texttt{latexsym}.\\ Descrito
  en \texttt{latexsym.dtx} y en \companion.
 
\item[\normalfont\pai{makeidx}] Proporciona instrucciones para
  producir 'indices de materias.\\ Descrito en el
  apartado~\ref{sec:indexing} y en \companion.

\item[\normalfont\pai{syntonly}] Procesa un documento sin
  componerlo.\\ Se describe en \texttt{syntonly.dtx} y en \companion.
  Es 'util para la verificaci'on r'apida de errores.
  
\item[\normalfont\pai{inputenc}] Permite la especificaci'on de una
  codificaci'on de entrada como ASCII (con la opci'on \pai{ascii}),
  ISO Latin-1 (con la opci'on \pai{latin1}), ISO Latin-2 (con la
  opci'on \pai{latin2}), p'aginas de c'odigo de 437/850 IBM (con las
  opciones \pai{cp437} y \pai{cp580}, respectivamente), Apple
  Macintosh (con la opci'on \pai{applemac}), Next (con la opci'on
  \pai{next}), ANSI-Windows (con la opci'on \pai{ansinew}) o una
  definida por el usuario.  Descrito en \texttt{inputenc.dtx}.

\end{description}
\end{lined}
\end{table}

\clearpage
%
% Puntero a informaci'on de los paquetes
%

\subsection{Estilo de p'agina}

Con \LaTeX{} existen tres combinaciones predefinidas de \wi{cabeceras}
y \wi{pies de p'agina}, a las que se llaman estilos de
p'agina.\index{estilo de pagina@estilo de p'agina} El par'ametro
\emph{estilo} de la instrucci'on
\index{estilo de pagina@estilo de p'agina!plain@\texttt{plain}}%
\index{plain@\texttt{plain}}%
\index{estilo de pagina@estilo de p'agina!headings@\texttt{headings}}%
\index{headings@\texttt{headings}}%
\index{estilo de pagina@estilo de p'agina!empty@\texttt{empty}}%
\index{empty@\texttt{empty}}%
\begin{command}
\ci{pagestyle}\verb|{|\emph{estilo}\verb|}|
\end{command}
\noindent define cu'al emplearse. La tabla~\ref{pagestyle} muestra los
estilos de p'agina predefinidos.

\begin{table}[!hbp]
\caption{Estilos de p'agina predefinidos en \LaTeX} \label{pagestyle}
\begin{lined}{12cm}
\begin{description}

\item[\normalfont\texttt{plain}] imprime los n'umeros de p'agina en el
  centro del pie de las p'aginas. Este es el estilo de p'agina que se
  toma si no se indica ning'un otro.

\item[\normalfont\texttt{headings}] en la cabecera de cada p'agina
  imprime el cap'itulo que se est'a procesando y el n'umero de
  p'agina, mientras que el pie est'a vac'io. (Este estilo es similar
  al empleado en este documento).

\item[\normalfont\texttt{empty}] deja tanto la cabecera como el pie
  de las p'aginas vac'ios.

\end{description}
\end{lined}
\end{table}

Es posible cambiar el estilo de p'agina de la p'agina actual con la
instrucci'on
\begin{command}
\ci{thispagestyle}\verb|{|\emph{estilo}\verb|}|
\end{command}
En \companion{} hay una descripci'on de c'omo crear sus propias
cabeceras y pies de p'agina.
%
% Puntero a la descripci'on del paquete fancyhdr
%
% Informaci'on sobre la numeraci'on de p'aginas, ...
% \pagenumbering

\section{Proyectos grandes}

Cuando trabaje con documentos grandes, podr'ia, si lo desea, dividir
el fichero de entrada en varias partes. \LaTeX{} tiene dos
instrucciones que le ayudan a realizar esto.

\begin{command}
\ci{include}\verb|{|\emph{fichero}\verb|}|
\end{command}
\noindent se puede utilizar en el cuerpo del documento para introducir el
contenido de otro fichero. En este caso, \LaTeX{} comenzar'a una p'agina
nueva antes de procesar el texto del \emph{fichero}.

La segunda instrucci'on s'olo puede ser empleada en el pre'ambulo.
Permite indicarle a \LaTeX{} que s'olo tome la entrada de algunos
ficheros de los indicados con \verb|\include|.

\begin{command}
\ci{includeonly}\verb|{|\emph{fichero}\verb|,|\emph{fichero}%
\verb|,|\ldots\verb|}|
\end{command}

Una vez que esta instrucci'on se ejecute en el pre'ambulo del
documento, s'olo se procesar'an las instrucciones \ci{include} con los
ficheros indicados en el argumento de la orden \ci{includeonly}.
Observe que no hay espacios entre los nombres de los ficheros y las
comas.

\endinput
