\section{Ap�ndice B}

\subsection{Funciones Relevantes}

\begin{center} $Q(x)$ \end{center}

\begin{verbatim}
  pflotante calcularQ(pflotante& paramX) {
		pflotante x, raiz, modulo, division;
		// calcular fl(x)
		x = paramX;
		truncar(x);
		// calcular x*x
		raiz = x*x;
		truncar(raiz);
		// calcular x*x+1
		raiz = raiz + 1;
		truncar(raiz);
		// calcular sqrt(x*x+1)
		raiz = sqrt(raiz);
		truncar(raiz);
		// calcular x - sqrt(x*x+1)
		modulo = x - raiz;
		truncar(modulo);
		// calcular abs(x - sqrt(x*x+1))
		modulo = abs(modulo);
		truncar(modulo);
		// calcular x + sqrt(x*x+1)
		division = x + raiz;
		truncar(division);
		// calcular 1 / (x + sqrt(x*x+1))
		division = 1 / division;
		truncar(division);
		// valor final
		modulo = modulo - division;
		truncar(modulo);
		// devolver
		return modulo;
	}
\end{verbatim}

...explicaci�n de la implementaci�n del m�todo...

\begin{center} $T(x)$ \end{center}

\begin{verbatim}
  pflotante calcularT(pflotante& paramX) {
		pflotante x, cero, resultado, ealax;
		// calcular fl(x)
		x = paramX;
		truncar(x);
		// ver si x == 0
		cero = 0.0;
		truncar(cero);
		if (x == cero) {
			resultado = 1.0;
		}
		else {
			// calcular e^x
			ealax = exp(x);
			truncar(ealax);
			// calcular e^x - 1
			ealax = ealax - 1;
			truncar (ealax);
			// calcular (e^x - 1)/x
			resultado = ealax/x;
		}
		truncar(resultado);
		return resultado;
	}
\end{verbatim}

...explicaci�n de la implementaci�n del m�todo...

\begin{center} $G(x)$ \end{center}

\begin{verbatim}
  pflotante calcularG(pflotante& paramX) {
		pflotante q2, resultado;
		// calcular Q(x)^2;
		q2 = calcularQ(paramX);
		q2 = q2*q2;
		truncar(q2);
		// calcular T(Q(x)^2)
		resultado = calcularT(q2);
		truncar(resultado);
		return resultado;
	}
\end{verbatim}

...explicaci�n de la implementaci�n del m�todo...
