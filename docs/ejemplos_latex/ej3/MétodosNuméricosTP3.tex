\documentclass[11pt,letterpaper]{article}
\usepackage{graphicx}
\usepackage[activeacute, spanish]{babel}
\usepackage[latin1]{inputenc}
\usepackage{amssymb,amsmath}

% \let\mathnumsetfont\mathbb
\newcommand\Nset{\mathbb N}       % set of positive integer numbers
\newcommand\Zset{\mathbb Z}       % set of integer numbers
\newcommand\Rset{\mathbb R}       % set of real numbers
\newcommand\real{\mathbb R}       % set of real numbers

\author{ Guillermo T\'unez, Mat\'ias Albanesi, Mercedes Bianchi }

\title{ \Huge
        \textbf{Trabajo Pr�ctico N� 3}\\[2mm]
        \LARGE
        \textbf{D\'onde est\'a el piloto?}\\[6mm]
        \large
        Universidad de Buenos Aires\\
        Facultad de Ciencias Exactas y Naturales\\
        Departamento de Computaci�n\\
        M�todos N�mericos\\[10mm]
        %\includegraphics[width=3cm]{uba.png}\\[6mm]
        Primer Cuatrimestre - 2007\\ }

\begin{document}
\maketitle
    \begin{center}
        \begin{tabular}{|c|c|c|}
            \hline
                Nombre & LU & E-mail \\
                \hline
      					Guillermo T\'unez & 790/00 & guillermotunez@yahoo.com.ar\\
                Mat\'ias Albanesi C\'aceres & 522/00 & mac@sion.com \\
      					Mar\'ia Mercedes Bianchi & 685/97 & mbianchi9@arnet.com.ar \\      			
                \hline
        \end{tabular}
\end{center}


\addvspace{1.5cm}

\textbf{Resumen:} 
Este trabajo se centra en plantear un modelo de programaci\'on lineal homog\'eneo positivo. Para la resoluci\'on de este modelo se debera implementar cualquiera de las variantes del M\'etodo Simplex. En este caso de opto por la version conocida con el nombre de M\'etodo Simplex Revisado. El problema que atacamos con esta herramienta tan potente se conoce como Revenue Managment, el cual tiene una aplicaci\'on muy importante en diferentes tipos de negocios. En este trabajo pr\'actico utilizamos instancias provenientes de empresas de transporte aereo comercial.\\

\textbf{Palabras Clave:} 
M\'etodo Simplex, Programaci\'on Lineal, Revenue Management.\\

\tableofcontents

\newpage
\section{Introducci\'on}

\subsection{Programaci\'on Lineal}

La programaci\'on lineal es el estudio de modelos matem\'aticos concernientes a la asignaci\'on eficiente de los recursos limitados en las actividades conocidas, con el objetivo de satisfacer las metas deseadas (tal como maximizar beneficios o minimizar costos).

En general, si $c_1, c_2, ..., c_n$ son numeros reales, entonces la funci\'on f de variables reales $x_1, x_2, ..., x_n$ definida por 

$$ f(x_1, x_2, ..., x_n) = c_1x_1 + c_2x_2 + ... + c_nx_n = \sum_{j=1}^n c_jx_j$$

es llamada \emph{funcion lineal}. Si $f$ es una funci\'on lineal y si $b$ es un n\'umero real, entonces la ecuaci\'on

$$f(x_1, x_2, ..., x_n) = b$$

es llamada una \emph{ecuaci\'on lineal} y las desigualdades 

$$f(x_1, x_2, ..., x_n)  \leq b$$
$$f(x_1, x_2, ..., x_n)  \geq b$$

son llamadas \emph{desigualdades lineales}. Tanto las ecuaciones lineales como las desigualdades lineales son llamadas \emph{restricciones lineales}. Finalmente, el problema de la programaci\'on lineal consiste en un problema de maximizaci\'on (o minimizaci\'on) de una funci\'on lineal sujeta a un finito n\'umero de restricciones lineales. Generalmente uniremos diversos sub\'indices $i$ a diversas restricciones  y diversos sub\'indices $j$ a diferentes variables. Esto lo haremos de la siguiente forma:

$$\mbox{maximizar}  \hspace{1cm}   \sum_{j=1}^n c_jx_j $$

$$\mbox{sujeto  a} \hspace{1cm} \sum_{j=1}^n a_{ij}x_j \leq b_i \hspace{1cm} (i = 1,2,...,m)$$

$$x_j \geq 0  \hspace{1cm} (j = 1,2,...,n).$$


La funci\'on lineal a maximizar o minimizar es llamada como la \emph{funci\'on objetivo} en los problemas LP. Finalmente, a la posible soluci\'on que maximiza la funci\'on objetivo (o que la minimiza) se la denomina \emph{soluci\'on \'optima}; el correspondiente valor de la funci\'on objetivo es llamado como el \emph{valor \'optimo} del problema.

No todo problema LP tiene una \'unica soluci\'on \'optima, algunos problemas tienen varias soluciones \'optimas y otros no tienen ninguna soluci\'on \'optima. El tipo de problema que no tiene una soluci\'on posible se lo llama insatisfactible. Otro tipo de problema sucede cuando tenemos varias posibles soluciones pero ninguna es la \'optima. Este tipo de problema es llamado como no acotado. Los problemas de programaci\'on lineal pertenecen  a una de estas tres categor\'ias: los que tienen una soluci\'on \'optima, los insatisfactible o los no acotados.

El m\'etodo m\'as conocido y habitual para resolver problemas de PL es el m\'etodo Simplex. 

\subsection{M\'etodo Simplex}


El m\'etodo Simplex es un procedimiento iterativo que permite ir mejorando la soluci\'on a cada paso. El proceso concluye cuando no es posible seguir mejorando m\'as dicha soluci\'on.

Partiendo del valor de la funci\'on objetivo en un v\'ertice cualquiera, el m\'etodo consiste en buscar sucesivamente otro v\'ertice que mejore al anterior. La b\'usqueda se hace siempre a trav\'es de los lados del pol\'igono (o de las aristas del poliedro, si el n\'umero de variables es mayor). C\'omo el n\'umero de v\'ertices (y de aristas) es finito, siempre se podr\'a encontrar la soluci\'on.


El m\'etodo Simplex se basa en la siguiente propiedad: si la funci\'on objetivo, f, no toma su valor m\'aximo en el v\'ertice A, entonces hay una arista que parte de A, a lo largo de la cual f aumenta.


Deber\'a tenerse en cuenta que este m\'etodo s\'olo trabaja para restricciones que tengan un tipo de desigualdad $<=$ y coeficientes independientes mayores o iguales a 0, y habr\'a que estandarizar las mismas para el algoritmo. En caso de que despu\'es de \'este proceso, aparezcan (o no var\'ien) restricciones del tipo $\geq$ o $=$ habr\'a que emplear otros m\'etodos.


\newpage
\subsection{Forma estandar del m\'odelo}


				
				Funci\'on objetivo: 
									
										$$c1 x1 + c2 x2 + ... + cn xn $$
				
				Sujeto a: 	
				
										$$a11 x1 + a12 x2 + ... + a1n xn \leq b1$$
				
										$$a21 x1 + a22 x2 + ... + a2n xn \leq b2$$
										
										$$...$$
										
										$$am1 x1 + am2 x2 + ... + amn xn \leq bm$$
										
										$$x1,..., xn \geq 0 $$
				
Para ello se deben cumplir las siguientes condiciones:

\begin{itemize}
\item El objetivo es de la forma de maximizaci\'on.
\item Todas las restricciones son desigualdades del tipo $\leq$.
\item Todas las variables son no negativas.
\item Las constantes a la derecha de las restricciones son no negativas.
\end{itemize}

\newpage
\subsection{Conversi\'on de signo de los t\'erminos independientes}

Deberemos preparar nuestro modelo de forma que los t\'erminos independientes de las restricciones sean mayores o iguales a 0, sino no se puede emplear el m\'etodo Simplex. Lo \'unico que habr\'ia que hacer es multiplicar por $-1$ las restricciones donde los t\'erminos independientes sean menores que 0.



\newpage
\subsection{An\'alisis del problema}

\subsubsection{Modelo matem\'atico}

Como paso previo al dise\~no de una soluci\'on en computadora del problema, analicemos el problema y sus implicaciones.

Una \emph{red de Morley} es un sistema interconectado de centrales generadoras de electricidad y nodos concentradores, dise\~nado para transportar electricidad y abastecer la demanda de todos los usuarios de la red.

Puede modelarse matem\'aticamente con un grafo, donde los nodos son de dos tipos distintos: concentradores y generadores. Hay ejes entre nodos del mismo tipo o de tipo distinto. Los ejes entre nodos concentradores forman el \emph{backbone} de la red.

Las centrales generadoras producen electricidad, pero tambi\'en abastecen a sus usuarios, cuyo consumo puede superar (o ser inferior, o igual) a la producci\'on de la central. La diferencia entre la producci\'on y la demanda es el balance, que puede ser positivo, nulo o negativo. Desde el punto de vista matem\'atico, balance es el \'unico dato relevante de una central generadora y nos abstraemos de su producci\'on real y la demanda de sus usuarios directos.

Una demanda que supera la producci\'on implica un balance negativo para una central generadora, y debe cubrirse con energ\'ia suministrada por otras centrales que tengan balance positivo.

Hay algunas condiciones adicionales importantes:

\begin{itemize}

\item Las centrales generadoras que tienen balance positivo entregan todo su excedente a la red.

\item Las centrales generadoras que tienen balance negativo cubren completamente su d\'eficit con energ\'ia suministrada por centrales con balance positivo a la red el\'ectrica.

\item Los nodos concentradores transmiten toda la energ\'ia que reciben.

\item \emph{(Restricci\'on de Morley):} El flujo de energ\'ia es id\'entico en cada eje del \emph{backbone}.

\end{itemize}

En resumen, el problema consiste en determinar cuanta energ\'ia debe enviarse a trav\'es de las interconexiones (ejes del grafo) entre nodos (concentradores o generadores) para cubrir  totalmente la demanda.

Notar que estas condiciones implican que no hay exceso de producci\'on de energ\'ia globalmente; o sea

\begin{eqnarray}
\sum_{i=1}^{n}{b_{i}} & = & 0 \nonumber
\end{eqnarray}

siendo $b_{i}$ el balance del nodo i-\'esimo del grafo, y definiendo para los nodos concentradores un balance cero (porque transmiten toda la energ\'ia que reciben y no la generan).


\subsubsection{Planteo como sistema de ecuaciones lineales}

Primero debemos encontrar un planteo apropiado para resolver el problema mediante un sistema de ecuaciones lineales. Nuestras inc�gnitas son los flujos de energ�a a asignar a cada eje del grafo.

Las condiciones (i) y (ii) nos proporcionan las ecuaciones:

\begin{eqnarray}
\sum_{\begin{tabular}{c}$e_{j}$:\\\small{eje saliente}\\\small{del nodo i}\end{tabular}}{flujo(e_{j})}
-
\sum_{\begin{tabular}{c}$e_{k}$:\\\small{eje entrante}\\\small{al nodo i}\end{tabular}}{flujo(e_{k})} & = & b_{i} \nonumber
\end{eqnarray}

Estas ecuaciones equivalen a:
\begin{eqnarray}
+x_{j_1} -x_{j_2} + x_{j_3} - x_{j_4} + ... & = & b_i \nonumber
\end{eqnarray}

donde $x_{j_k}$ es el flujo de un eje incidente al nodo i-\'esimo, y $b_i$ es el balance del nodo i-\'esimo. Como puede verse, se coloca un signo positivo si el eje es ``saliente'' del nodo y signo negativo si el eje es ``entrante''.

Resolver un sistema formado por esas ecuaciones cumplir\'ia con las condiciones (i), (ii) y tambi\'en (iii) definiendo para los nodos concentradores un balance cero.

La condici\'on (iv) puede expresarse como:
\begin{eqnarray}
| x_{j_p} | & = & | x_{j_q} | \nonumber
\end{eqnarray}

para cada par de ejes $j_p$ y $j_q$, $p \neq q$, que formen parte del backbone. Notar que al igualar los valores absolutos de los flujos de los ejes del backbone, no podemos incorporar estas condiciones directamente a un sistema de ecuaciones lineales.

\subsubsection{Problemas encontrados}

De la secci\'on anterior surgen estos problemas:

\begin{itemize}

\item Necesitamos asignar a los ejes una direcci\'on. Saber si son ``salientes'' o ``entrantes''.
\item Las ecuaciones que cumplen con la condici\'on (iv) del enunciado no pueden incorporarse a un sistema de ecuaciones lineales directamente.

\end{itemize}

Para resolver el primer punto pensamos inicialmente establecer dos ejes entre cada nodo, uno entrante y otro saliente, y poner a ambos como inc\'ognitas.

Pero eso no es necesario: puede modelarse con s\'olo un eje orientado entre nodos. Si al resolver el sistema de ecuaciones el valor de flujo para un eje resultara negativo, querr\'ia decir que ese eje deber\'ia tener el sentido opuesto. Si fuera cero o positivo el eje conservar\'ia su sentido inicial. Entonces el sentido inicial puede ser cualquiera, y se corrige al tener la soluci\'on del sistema.

El segundo punto es un obst\'aculo m\'as importante. Finalmente decidimos asumir que nuestra implementaci\'on s\'olo puede resolver el problema con un backbone donde todos los ejes tienen el mismo sentido. Esto puede definirse sin problemas s\'olo en el caso de que el backbone sea lineal, por lo tanto tambi\'en pedimos esta condici\'on en el grafo de entrada.

De no asumirse esas condiciones sobre el backbone, habr\'ia que resolver $2^{n}$ sistemas, cada uno compuesto por las ecuaciones derivadas de las condiciones (i) a (iii), m\'as la ecuaci\'on

\begin{eqnarray}
x_{j_p} & = & x_{j_q} \nonumber
\end{eqnarray}
o
\begin{eqnarray}
x_{j_p} & = & - x_{j_q} \nonumber
\end{eqnarray}

para cada par de ejes del backbone.

%------------------------------------------------------------------------------%


\newpage
\section{An\'alisis previo al desarrollo en computadora}

La primer decisi\'on a tomar consiste en plataforma y compilador a utilizar. Como el desarrollo debe cumplir con el requisito de poder compilarse y ejecutarse con el software disponible en los laboratorios del Departamento de Computaci\'on, optamos por computadoras con procesadores compatibles con la arquitectura Intel x86, y el compilador g++ de la Free Software Foundation.

Luego analizamos las distintas alternativas posibles para implementar la aritm\'etica de punto flotante de precisi\'on variable, que consisten en:

\begin{itemize}
\item Desarrollar un formato propio, junto con una implementaci\'on de las operaciones sobre dicho formato.
\item Utilizar la aritm\'etica de punto flotante que provee la computadora y realizar ajustes para reducir la precisi\'on de los resultados.
\end{itemize}

Nos decidimos r\'apidamente por la segunda opci\'on ya que desarrollar un nuevo formato e implementaci\'on insumir\'ia m\'as tiempo de desarrollo y prueba, y el problema al cual se aplica no amerita el esfuerzo. Se trata de encontrar errores m\'as que de evitarlos con n\'umeros de precisiones mayores.

Al usar la aritm\'etica de punto flotante que provee el hardware, podemos elegir entre cualquiera de los tres formatos: \textbf{float}, \textbf{double} o \textbf{long double}. Optamos por este \'ultimo ya que pose\'ia mayor cantidad de bits en la mantisa y tendr\'iamos un rango de precisiones mayor para elegir; tambi\'en encontramos que puede accederse f\'acilmente a la mantisa por separado del exponente y signo lo que simplifica la implementaci\'on.

Al operar con formatos nativos del equipo, se realizar\'an las operaciones y luego de cada una deber\'a ajustarse el resultado por alguno de los siguiente m\'etodos:
\begin{itemize}
\item redondeo
\item truncamiento
\end{itemize}

Pese a que el redondeo es preferible ya que el error relativo de representaci\'on es menor (la mitad) que el error cometido al realizar truncamiento, optamos por truncar el resultado ya que la implementaci\'on ser\'ia muy simple. Y debido a que estamos buscando resultados inexactos, al operar con aritm\'etica de punto flotante y truncar tendremos errores m\'as evidentes. Cabe mencionar que al realizar un truncamiento luego de cada operaci\'on de c\'alculo estamos agregando error a cada paso y obtendremos un resultado final con elevado alto grado de error.

%----------------------------------------------------------------------------------------------%

\subsection{El formato en punto flotante IEEE-754}

Un formato de punto flotante es una estructura de datos que especifica los campos que abarcan un n\'umero de punto flotante, la disposici\'on de esos campos, y su interpretaci\'on aritm\'etica. Un formato de almacenamiento de punto flotante especifica c\'omo un n\'umero de punto flotante se almacena en memoria. El est\'andar de IEEE define los formatos, pero deja a los implementadores la opci\'on de las formas de almacenamiento.

Existen varios formatos para la representaci\'on de n\'umeros flotantes en la computadora, aunque
el est\'andar es el formato ANSI/IEEE standard 754-1985, que llamaremos IEEE-754 para abreviar.

Existen 2 formatos b\'asicos de representaci\'on de punto flotantes: single y double.
Tambi\'en presenta dos clases extendidas de representaci\'on de n\'umeros flotantes: single extendida y double extendida.
El est\'andar no prescribe la precisi\'on y el tama�o exactos de estos formatos, sino que especifica la precisi\'on y el tama\~no m\'inimos. 

Una implementaci\'on concreta de este est\'andar es el formato de punto flotante extendido de la arquitectura Intel x86, de 80 bits de longitud, que dedica 64 bits para la mantisa - este formato se denomina \textbf{long double} en el lenguaje C/C++.

%----------------------------------------------------------------------------------------------%

\subsection{El Formato de precisi�n doble extendido de Intel x86}

El formato \emph{double} extendido de la arquitectura Intel x86, denominado \textbf{long double} en el lenguaje C/C++, cumple con las condiciones del formato doble extendido IEEE-754, y es utilizado internamente por el procesador al realizar c\'alculos. El procesador convirte cualquier valor, ya sea en precisi\'on simple o doble, a este formato al realizar operaciones, y vuelve a convertir a la precisi\'on de salida deseada al almacenar el resultado en una ubicaci\'on de memoria.

Ocupando un total de 80 bits, est\'a compuesto por los siguientes campos:
\begin{itemize}
\item Una mantisa de 64 bits, con la particularidad de tener todos sus bits expl\'icitos (a diferencia del formato de precisi�n simple o doble, donde se asume un primer bit con valor 1).
\item Un exponente de 15 bits, con desplazamiento 16383.
\item Un indicador de signo de 1 bit.
\end{itemize}

Un est\'andar en entornos Unix es hacer el pasaje de valores long double por stack ocupando tres palabras consecutivas de 32 bits (dwords), ocupando un total de 96 bits. Los 16 bits superiores de la dword con la direcci\'on en memoria m\'as alta se dejan en cero. Como resultado, en estas implementaciones el operador \textbf{sizeof(long double)} de C/C++ devuelve 12 en lugar de 10.

\begin{center}
% Definimos el estilo de la tabla (3 columnas separadas por
% lineas verticales)
\begin{tabular}{|l|l|l|}
\hline
Signo & Exponente & Mantisa \\
\hline
$\pm$ & $e_{14} e_{13}...e_1 e_0$ & $m_{63} m_{62}...m_2 m_1 m_0$ \\
\hline
\end{tabular} \\
% Definimos el titulo y la etiqueta de la tabla
Representaci\'on de un n\'umero flotante. Un bit para el signo, 15 bits para el exponente y 64 bits para la mantisa.
\end{center}

Teniendo $E = e_{14} e_{13}...e_1 e_0$, con este formato pueden representarse los valores a continuaci\'on:

\begin{center}
\begin{tabular}{|l|l|l|}
\hline
E & Mantisa & Valor representado \\
\hline
$0$ & $0$ & $\pm 0$ \\
\hline
$0$ & $\neq 0$ & n\'umero denormalizado \\
\hline
$0 < E < 32767$ & cualquiera & n\'umero en punto flotante normalizado \\
\hline
$32767$ & $0$ & $\pm$ infinito \\
\hline
$32767$ & $\neq 0$ & Not A Number \\
\hline
\end{tabular}
\end{center}

Un n\'umero en punto flotante normalizado tiene el bit m\'as significativo de la mantisa $m_{63}$ con el valor $1$, y representa al valor real $\pm 1 . m_{62} m_{61}... m_2 m_1 m_0 \times 2^{E - 16383}$.

%----------------------------------------------------------------------------------------------%


\newpage
\section{Triste intento de c\'alculo de error relativo de G(x)}

\begin{eqnarray}
\epsilon(\frac{e^{Q(x)^2}-1}{Q(x)^2}) & = & \epsilon(\frac{a}{b}) \nonumber \\
\nonumber \\
& \leq & \Big|\epsilon(a)\Big| + \Big|\epsilon(b)\Big| + \Big|\epsilon_{(\%)}\Big| \nonumber \\
\nonumber
\end{eqnarray}

\begin{eqnarray}
\epsilon(a) & = & \epsilon(e^{Q(x)^2} - 1) \nonumber \\
\nonumber \\
& = & \epsilon(a' - b') \nonumber \\
\nonumber \\
& \leq & \Big| \frac{1}{a' - b'} a' \epsilon(a') \Big| + \Big| \frac{b'}{a'-b'} \epsilon(b') \Big| + \epsilon_{(-)} \nonumber \\
\nonumber
& = & \Big| \frac{1}{a' - 1} a' \epsilon(a') \Big| + \Big| \frac{1}{a'-1} \epsilon(1) \Big| + \epsilon_{(-)} \nonumber \\
& = & \Big| \frac{1}{a' - 1} a' \epsilon(a') \Big| + \epsilon_{(-)} \nonumber \\
\nonumber
\end{eqnarray}

\begin{eqnarray}
\epsilon(a') & = & \epsilon( e^{Q(x)^2} ) \nonumber \\
\nonumber \\
& = & \epsilon(e^{a''}) \nonumber \\
\nonumber \\
& \leq & \Big| \frac{e^{a''}}{e^{a''}} \epsilon(a'') \Big| + \epsilon_{(exp)} \nonumber \\
\nonumber \\
& = & \epsilon(a'') + \epsilon_{(exp)} \nonumber \\
\nonumber
\end{eqnarray}

\begin{eqnarray}
\epsilon(a'') & = & \epsilon(Q(x)^2) \nonumber \\
\nonumber \\
& = & \epsilon(a'''.a''') \nonumber \\
\nonumber \\
& \leq & \Big| \frac{a''' + a'''}{a'''^2} a''' \epsilon(a''') \Big| + \epsilon_{(x)} \nonumber \\
\nonumber \\
& = & \Big| 2 \epsilon(a''') \Big| + \epsilon_{(x)} \nonumber \\
\nonumber
\end{eqnarray}

\begin{eqnarray}
\epsilon(b) & = & \epsilon(Q(x)^2) \nonumber \\
\nonumber \\
& = & \epsilon(a''') \nonumber \\
\nonumber \\
& = & \Big| 2 \epsilon(a''') \Big| + \epsilon_{(x)} \nonumber \\
\nonumber
\end{eqnarray}

\begin{eqnarray}
\epsilon(a''') & = & \epsilon(Q(x)^2) \nonumber \\
\nonumber \\
& = & \epsilon(q_1(x) - q_2(x)) \nonumber \\
\nonumber \\
& \leq & \Big| \frac{1}{q_1(x) - q_2(x)} q_1(x) \epsilon(q_1(x)) \Big| +
\Big| \frac{-1}{q_1(x) - q_2(x)} q_2(x) \epsilon(q_2(x)) \Big| + \epsilon_{(-)} \nonumber \\
\nonumber
\end{eqnarray}

Que se me indefine, porque tengo en el denominador a $Q(x)$, que es matem'aticamente igual a cero
para cualquier valor de x.
Esto no s'e muy bien como salvarlo.




\newpage
\section{Resultados}

% Pautas: deben incluir los resultados de los experimentos. utilizando el formato m�s adecuado
% para su presentaci�n. Deberan especificar claramente a que experiencia corresponde cada
% resultado. No se incluiran aqui corridas de m�quina.

% TAREAS: presentar bien los datos medidos y aclarar a que experimento corresponde cada grafico

\subsection{Archivo de prueba 1}

\begin{scriptsize}
Red Aerolineas con 7 tramos y 22 productos.\newline
\newline
Init Tramos \newline
\newline
Tr 1 Cant. 100 BandaH. M \newline
Tr 2 Cant. 150 BandaH. M\newline
Tr 3 Cant. 150 BandaH. T\newline
Tr 4 Cant. 150 BandaH. M\newline
Tr 5 Cant. 150 BandaH. T\newline
Tr 6 Cant.  80 BandaH. M\newline
Tr 7 Cant.  80 BandaH. T\newline
\newline
End Tramos\newline
\newline
Init Productos\newline   
\newline
Pr  1 CantTr 1 { 1 }	Tarifa 1000	Cant. 20\newline
Pr  2 CantTr 1 { 2 }	Tarifa 400	Cant. 20\newline
Pr  3 CantTr 1 { 3 }	Tarifa 400	Cant. 16\newline
Pr  4 CantTr 1 { 4 }	Tarifa 300	Cant. 21\newline
Pr  5 CantTr 1 { 5 }	Tarifa 300	Cant. 20\newline
Pr  6 CantTr 1 { 6 }	Tarifa 500	Cant. 19\newline
Pr  7 CantTr 1 { 7 }	Tarifa 500	Cant. 21\newline
Pr  8 CantTr 2 { 2 4 }	Tarifa 600	Cant. 15\newline
Pr  9 CantTr 2 { 3 5 }	Tarifa 600	Cant. 15\newline
Pr 10 CantTr 2 { 2 6 }	Tarifa 700	Cant. 18\newline
Pr 11 CantTr 2 { 3 7 }	Tarifa 700	Cant. 16\newline
Pr 12 CantTr 1 { 1 }	Tarifa 500	Cant. 75\newline
Pr 13 CantTr 1 { 2 }	Tarifa 200	Cant. 45\newline
Pr 14 CantTr 1 { 3 }	Tarifa 200	Cant. 50\newline
Pr 15 CantTr 1 { 4 }	Tarifa 150	Cant. 45\newline
Pr 16 CantTr 1 { 5 }	Tarifa 150	Cant. 41\newline
Pr 17 CantTr 1 { 6 }	Tarifa 250	Cant. 40\newline
Pr 18 CantTr 1 { 7 }	Tarifa 250	Cant. 37\newline
Pr 19 CantTr 2 { 2 4 }	Tarifa 300	Cant. 35\newline
Pr 20 CantTr 2 { 3 5 }	Tarifa 300	Cant. 43\newline
Pr 21 CantTr 2 { 2 6 }	Tarifa 350	Cant. 39\newline
Pr 22 CantTr 2 { 3 7 }	Tarifa 350	Cant. 35\newline
End Productos\newline

\end{scriptsize}

\begin{center}
	\begin{tabular}{|c|c|}
	\hline
	Cantidad de tramos & 7 \\
	\hline
	Cantidad de productos & 22 \\
	\hline
	\end{tabular}
\end{center}
\newpage
%------------------------------------------------------------------------------------%

\subsection{Archivo de prueba 2}

\begin{scriptsize}
Red Aerolineas con 11 tramos y 108 productos.\newline
\newline
Init Tramos\newline 
\newline
Tr 1  Cant. 200 BandaH. M \newline
Tr 2  Cant. 150 BandaH. M\newline
Tr 3  Cant. 100 BandaH. T\newline
Tr 4  Cant. 300 BandaH. M\newline
Tr 5  Cant. 200 BandaH. T\newline
Tr 6  Cant. 180 BandaH. M\newline
Tr 7  Cant. 150 BandaH. T\newline
Tr 8  Cant. 100 BandaH. T\newline
Tr 9  Cant.  90 BandaH. T\newline
Tr 10 Cant. 200 BandaH. T\newline
Tr 11 Cant. 100 BandaH. T\newline
\newline
End Tramos\newline
\newline
Init Productos\newline   
\newline
Pr   1	CantTr 1 { 1 }			Tarifa  200 	Cant. 20\newline
Pr   2	CantTr 1 { 2 }			Tarifa  600 	Cant. 23\newline
Pr   3	CantTr 1 { 3 }			Tarifa 1100	Cant. 15\newline
Pr   4	CantTr 1 { 4 }			Tarifa  150 	Cant. 18\newline
Pr   5	CantTr 1 { 5 }			Tarifa  200 	Cant. 19\newline
Pr   6	CantTr 1 { 6 }			Tarifa  800 	Cant. 20\newline
Pr   7	CantTr 1 { 7 }			Tarifa  400 	Cant. 15\newline
Pr   8	CantTr 1 { 8 }			Tarifa  250 	Cant. 25\newline
Pr   9	CantTr 1 { 9 }			Tarifa   50 	Cant. 10\newline
Pr  10	CantTr 1 { 10 }			Tarifa  600 	Cant. 33\newline
Pr  11	CantTr 1 { 11 }			Tarifa  300 	Cant. 41\newline
Pr  12	CantTr 2 { 1 4 }		Tarifa  250 	Cant. 13\newline
Pr  13	CantTr 2 { 1 6 }		Tarifa  850 	Cant. 45\newline
Pr  14	CantTr 2 { 2 5 }		Tarifa  700 	Cant. 27\newline
Pr  15	CantTr 3 { 1 4 5 }		Tarifa  500 	Cant. 35\newline
Pr  16	CantTr 3 { 1 4 7 }		Tarifa  750 	Cant. 52\newline
Pr  17	CantTr 4 { 1 4 5 11 }		Tarifa  950 	Cant. 11\newline
Pr  18	CantTr 5 { 1 4 7 8 10 }		Tarifa 1200 	Cant.  5\newline
Pr  19	CantTr 4 { 1 6 8 10 }		Tarifa 1000 	Cant.  7\newline
Pr  20	CantTr 6 { 1 4 7 8 9 11 }	Tarifa  950 	Cant. 47\newline
Pr  21	CantTr 5 { 1 6 8 9 11 }		Tarifa 1100	Cant. 32\newline
Pr  22	CantTr 2 { 3 11 }		Tarifa 1500	Cant. 29\newline
Pr  23	CantTr 4 { 2 7 8 10 }		Tarifa  900 	Cant. 20\newline
Pr  24	CantTr 5 { 2 7 8 9 11 }		Tarifa 1300 	Cant. 55\newline
Pr  25	CantTr 2 { 8 9 }		Tarifa  200 	Cant. 15\newline
Pr  26	CantTr 2 { 8 10 }		Tarifa  550 	Cant. 16\newline
Pr  27	CantTr 3 { 8 9 11 }		Tarifa  450 	Cant. 19\newline
Pr  28	CantTr 2 { 9 11 }		Tarifa  320 	Cant. 12\newline
Pr  29	CantTr 3 { 7 8 9 }		Tarifa  150 	Cant.  5\newline
Pr  30	CantTr 3 { 7 8 10 }		Tarifa  900 	Cant. 21\newline
Pr  31	CantTr 4 { 7 8 9 11 }		Tarifa  800	Cant. 23\newline
Pr  32	CantTr 2 { 7 8 }		Tarifa  500	Cant. 43\newline
Pr  33	CantTr 2 { 6 8 }		Tarifa  500	Cant. 45\newline
Pr  34	CantTr 3 { 6 8 9 }		Tarifa  600	Cant. 29\newline
Pr  35	CantTr 4 { 6 8 9 11 }		Tarifa  700	Cant. 21\newline
Pr  36	CantTr 3 { 6 8 10 }		Tarifa  490	Cant. 30\newline
Pr  37	CantTr 2 { 3 11 }		Tarifa 2150 	Cant. 50\newline
Pr  38	CantTr 3 { 2 5 11 }		Tarifa  400 	Cant. 50\newline
Pr  39	CantTr 2 { 2 7 }		Tarifa  600 	Cant. 41\newline
Pr  40	CantTr 3 { 2 7 8 }		Tarifa  700 	Cant. 39\newline
Pr  41	CantTr 4 { 2 7 8 10 }		Tarifa  800 	Cant. 40\newline
Pr  42	CantTr 4 { 2 7 8 9 }		Tarifa  390 	Cant. 45\newline
Pr  43	CantTr 5 { 2 7 8 9 11 }		Tarifa  750 	Cant. 75\newline
Pr  44	CantTr 2 { 4 5 }		Tarifa  300 	Cant. 35\newline
Pr  45	CantTr 3 { 4 5 11 }		Tarifa  600 	Cant. 16\newline
Pr  46	CantTr 2 { 4 7 }		Tarifa  350 	Cant. 20\newline
Pr  47	CantTr 3 { 4 7 8 }		Tarifa  390 	Cant. 21\newline
Pr  48	CantTr 4 { 4 7 8 9 }		Tarifa  220 	Cant. 15\newline
Pr  49	CantTr 5 { 4 7 8 9 11 }		Tarifa  850 	Cant. 18\newline
Pr  50	CantTr 4 { 4 7 8 10 }  		Tarifa  950 	Cant. 17\newline	
Pr  51	CantTr 2 { 5 11 } 		Tarifa  350 	Cant. 19\newline
Pr  52	CantTr 2 { 2 7 } 		Tarifa  450 	Cant. 22\newline
Pr  53	CantTr 3 { 2 7 8 }		Tarifa  550 	Cant. 16\newline
Pr  54	CantTr 4 { 2 7 8 9 }		Tarifa 	930 	Cant. 13\newline
Pr  55	CantTr 1 { 1 }			Tarifa  250 	Cant. 50\newline
Pr  56	CantTr 1 { 2 }			Tarifa  680 	Cant. 53\newline
Pr  57	CantTr 1 { 3 }			Tarifa 1300	Cant. 55\newline
Pr  58	CantTr 1 { 4 }			Tarifa  220 	Cant. 48\newline
Pr  59	CantTr 1 { 5 }			Tarifa  250 	Cant. 49\newline
Pr  60	CantTr 1 { 6 }			Tarifa  950 	Cant. 50\newline
Pr  61	CantTr 1 { 7 }			Tarifa  500 	Cant. 45\newline
Pr  62	CantTr 1 { 8 }			Tarifa  300 	Cant. 45\newline
Pr  63	CantTr 1 { 9 }			Tarifa  100 	Cant. 50\newline
Pr  64	CantTr 1 { 10 }			Tarifa  750 	Cant. 63\newline
Pr  65	CantTr 1 { 11 }			Tarifa  400 	Cant. 51\newline
Pr  66	CantTr 2 { 1 4 }		Tarifa  360 	Cant. 33\newline
Pr  67	CantTr 2 { 1 6 }		Tarifa 1050 	Cant. 65 \newline
Pr  68	CantTr 2 { 2 5 }		Tarifa 1000 	Cant. 47\newline
Pr  69	CantTr 3 { 1 4 5 }		Tarifa  700 	Cant. 65\newline
Pr  70	CantTr 3 { 1 4 7 }		Tarifa  900 	Cant. 72\newline
Pr  71	CantTr 4 { 1 4 5 11 }		Tarifa 1300 	Cant. 31\newline
Pr  72	CantTr 5 { 1 4 7 8 10 }		Tarifa 1400 	Cant. 15\newline
Pr  73	CantTr 4 { 1 6 8 10 }		Tarifa 1500 	Cant. 17\newline
Pr  74	CantTr 6 { 1 4 7 8 9 11 }	Tarifa 1250 	Cant. 57\newline
Pr  75	CantTr 5 { 1 6 8 9 11 }		Tarifa 1300	Cant. 42\newline
Pr  76	CantTr 2 { 3 11 }		Tarifa 1850	Cant. 39\newline
Pr  77	CantTr 4 { 2 7 8 10 }		Tarifa 1200 	Cant. 30\newline
Pr  78	CantTr 5 { 2 7 8 9 11 }		Tarifa 2000 	Cant. 65\newline
Pr  79	CantTr 2 { 8 9 }		Tarifa  300 	Cant. 25\newline
Pr  80	CantTr 2 { 8 10 }		Tarifa  700 	Cant. 26\newline
Pr  81	CantTr 3 { 8 9 11 }		Tarifa  600 	Cant. 29\newline
Pr  82	CantTr 2 { 9 11 }		Tarifa  520 	Cant. 22\newline
Pr  83	CantTr 3 { 7 8 9 }		Tarifa  250 	Cant. 15\newline
Pr  84	CantTr 3 { 7 8 10 }		Tarifa 1350 	Cant. 31\newline
Pr  85	CantTr 4 { 7 8 9 11 }		Tarifa 1200	Cant. 33\newline
Pr  86	CantTr 2 { 7 8 }		Tarifa  800	Cant. 63\newline
Pr  87	CantTr 2 { 6 8 }		Tarifa  700	Cant. 65\newline
Pr  88	CantTr 3 { 6 8 9 }		Tarifa  750	Cant. 59\newline
Pr  89	CantTr 4 { 6 8 9 11 }		Tarifa  900	Cant. 51\newline
Pr  90	CantTr 3 { 6 8 10 }		Tarifa  620	Cant. 50\newline
Pr  91	CantTr 2 { 3 11 }		Tarifa 2600 	Cant. 60\newline
Pr  92	CantTr 3 { 2 5 11 }		Tarifa  500 	Cant. 55\newline
Pr  93	CantTr 2 { 2 7 }		Tarifa  800 	Cant. 61\newline
Pr  94	CantTr 3 { 2 7 8 }		Tarifa 1050 	Cant. 59\newline
Pr  95	CantTr 4 { 2 7 8 10 }		Tarifa 1200 	Cant. 60\newline
Pr  96	CantTr 4 { 2 7 8 9 }		Tarifa  520 	Cant. 65\newline
Pr  97	CantTr 5 { 2 7 8 9 11 }		Tarifa 1050 	Cant. 85\newline
Pr  98	CantTr 2 { 4 5 }		Tarifa  500 	Cant. 55\newline
Pr  99	CantTr 3 { 4 5 11 }		Tarifa  920 	Cant. 36\newline
Pr 100	CantTr 2 { 4 7 }		Tarifa  530 	Cant. 40\newline
Pr 101	CantTr 3 { 4 7 8 }		Tarifa  760 	Cant. 41\newline
Pr 102	CantTr 4 { 4 7 8 9 }		Tarifa  300 	Cant. 15\newline
Pr 103	CantTr 5 { 4 7 8 9 11 }		Tarifa  940 	Cant. 18\newline
Pr 104	CantTr 4 { 4 7 8 10 }  		Tarifa 1140 	Cant. 17	\newline
Pr 105	CantTr 2 { 5 11 } 		Tarifa  510 	Cant. 19\newline
Pr 106	CantTr 2 { 2 7 } 		Tarifa  650 	Cant. 32\newline
Pr 107	CantTr 3 { 2 7 8 }		Tarifa  750 	Cant. 16\newline
Pr 108	CantTr 4 { 2 7 8 9 }		Tarifa 1150 	Cant. 13\newline
\newline
End Productos\newline

\end{scriptsize}

\begin{center}
	\begin{tabular}{|c|c|}
	\hline
	Cantidad de tramos & 11 \\
	\hline
	Cantidad de productos & 108 \\
	\hline
	\end{tabular}
\end{center}
\newpage
%------------------------------------------------------------------------------------%

\subsection{Archivo de prueba 3}


\begin{scriptsize}
Red Aerolineas con 45 tramos y 166 productos.\newline
\newline
Init Tramos \newline
\newline
Tr 1	Cant.	200	BandaH. M\newline
Tr 2	Cant.	150	BandaH. M\newline
Tr 3	Cant.	180	BandaH. M\newline
Tr 4	Cant.	200	BandaH. M\newline
Tr 5	Cant.	200	BandaH. M\newline
Tr 6	Cant.	120	BandaH. M\newline
Tr 7	Cant.	160	BandaH. M\newline
Tr 8	Cant.	220	BandaH. M\newline
Tr 9	Cant.	180	BandaH. M\newline
Tr 10	Cant.	180	BandaH. M\newline
Tr 11	Cant.	200	BandaH. M\newline
Tr 12	Cant.	100	BandaH. M\newline
Tr 13	Cant.	150	BandaH. M\newline
Tr 14	Cant.	150	BandaH. M\newline
Tr 15	Cant.	150	BandaH. M\newline
Tr 16	Cant.	180	BandaH. M\newline
Tr 17	Cant.	160	BandaH. M\newline
Tr 18	Cant.	180	BandaH. M\newline
Tr 19	Cant.	200	BandaH. M\newline
Tr 20	Cant.	150	BandaH. M\newline
Tr 21	Cant.	200	BandaH. M\newline
Tr 22	Cant.	180	BandaH. M\newline
Tr 23	Cant.	160	BandaH. M\newline
Tr 24	Cant.	130	BandaH. M\newline
Tr 25	Cant.	150	BandaH. M\newline
Tr 26	Cant.	180	BandaH. M\newline
Tr 27	Cant.	180	BandaH. M\newline
Tr 28	Cant.	190	BandaH. M\newline
Tr 29	Cant.	200	BandaH. M\newline
Tr 30	Cant.	200	BandaH. M\newline
Tr 31	Cant.	200	BandaH. M\newline
Tr 32	Cant.	150	BandaH. M\newline
Tr 33	Cant.	200	BandaH. M\newline
Tr 34	Cant.	180	BandaH. M\newline
Tr 35	Cant.	200	BandaH. M\newline
Tr 36	Cant.	180	BandaH. M\newline
Tr 37	Cant.	180	BandaH. M\newline
Tr 38	Cant.	180	BandaH. M\newline
Tr 39	Cant.	150	BandaH. M\newline
Tr 40	Cant.	200	BandaH. M\newline
Tr 41	Cant.	150	BandaH. M\newline
Tr 42	Cant.	200	BandaH. M\newline
Tr 43	Cant.	200	BandaH. M\newline
Tr 44	Cant.	180	BandaH. M\newline
Tr 45	Cant.	190	BandaH. M\newline	
\newline
End Tramos\newline
\newline
Init Productos\newline
\newline
Pr   1	CantTr 1 { 1 }		Tarifa	288	Cant.	20\newline
Pr   2	CantTr 1 { 1 }		Tarifa	360	Cant.	20\newline
Pr   3	CantTr 1 { 1 }		Tarifa	555	Cant.	50\newline
Pr   4	CantTr 1 { 2 }		Tarifa	308	Cant.	70\newline
Pr   5	CantTr 1 { 3 }		Tarifa	597	Cant.	15\newline
Pr   6	CantTr 1 { 3 }		Tarifa	950	Cant.	15\newline
Pr   7	CantTr 1 { 3 }		Tarifa	1145	Cant.	85\newline
Pr   8	CantTr 2 { 1 3 }	Tarifa	885	Cant.	21\newline
Pr   9	CantTr 2 { 1 3 }	Tarifa	1310	Cant.	28\newline
Pr  10	CantTr 2 { 1 3 }	Tarifa	1700	Cant.	45\newline
Pr  11	CantTr 1 { 4 }		Tarifa	864	Cant.	103\newline
Pr  12	CantTr 2 { 2 4 }	Tarifa	1172	Cant.	54\newline
Pr  13	CantTr 1 { 5 }		Tarifa	321	Cant.	20\newline
Pr  14	CantTr 1 { 5 }		Tarifa	356	Cant.	23\newline
Pr  15	CantTr 1 { 5 }		Tarifa	440	Cant.	35\newline
Pr  16	CantTr 1 { 5 }		Tarifa	678	Cant.	67\newline
Pr  17	CantTr 1 { 7 }		Tarifa	289	Cant.	20\newline
Pr  18	CantTr 1 { 7 }		Tarifa	556	Cant.	30\newline
Pr  19	CantTr 1 { 6 }		Tarifa	329	Cant.	20\newline
Pr  20	CantTr 1 { 6 }		Tarifa	410	Cant.	20\newline
Pr  21	CantTr 1 { 6 }		Tarifa	633	Cant.	20\newline
Pr  22	CantTr 1 { 8 }		Tarifa	941	Cant.	62\newline
Pr  23	CantTr 2 { 2 8 }	Tarifa	1249	Cant.	35\newline
Pr  24	CantTr 1 { 9 }		Tarifa	197	Cant.	90\newline
Pr  25	CantTr 1 { 10 }		Tarifa	303	Cant.	15\newline
Pr  26	CantTr 1 { 10 }		Tarifa	377	Cant.	19\newline
Pr  27	CantTr 1 { 10 }		Tarifa	581	Cant.	33\newline
Pr  28	CantTr 1 { 11 }		Tarifa	551	Cant.	10\newline
Pr  29	CantTr 1 { 11 }		Tarifa	857	Cant.	15\newline
Pr  30	CantTr 1 { 11 }		Tarifa	1061	Cant.	68\newline
Pr  31	CantTr 1 { 12 }		Tarifa	358	Cant.	30\newline
Pr  32	CantTr 1 { 12 }		Tarifa	685	Cant.	40\newline
Pr  33	CantTr 1 { 13 }		Tarifa	859	Cant.	85\newline
Pr  34	CantTr 1 { 14 }		Tarifa	937	Cant.	10\newline
Pr  35	CantTr 1 { 14 }		Tarifa	985	Cant.	12\newline
Pr  36	CantTr 1 { 14 }		Tarifa	1555	Cant.	21\newline
Pr  37	CantTr 1 { 14 }		Tarifa	1888	Cant.	58\newline
Pr  38	CantTr 1 { 14 }		Tarifa	1437	Cant.	35\newline
Pr  39	CantTr 2 { 13 14 }	Tarifa	1769	Cant.	20\newline
Pr  40	CantTr 2 { 13 14 }	Tarifa	1844	Cant.	30\newline
Pr  41	CantTr 2 { 13 14 }	Tarifa	2414	Cant.	40\newline
Pr  42	CantTr 2 { 13 14 }	Tarifa	2747	Cant.	50\newline
Pr  43	CantTr 2 { 13 14 }	Tarifa	2296	Cant.	35\newline
Pr  44	CantTr 1 { 15 }		Tarifa	492	Cant.	20\newline
Pr  45	CantTr 1 { 15 }		Tarifa	944	Cant.	36\newline
Pr  46	CantTr 1 { 16 }		Tarifa	397	Cant.	30\newline
Pr  47	CantTr 1 { 16 }		Tarifa	493	Cant.	32\newline
Pr  48	CantTr 1 { 16 }		Tarifa	760	Cant.	33\newline
Pr  49	CantTr 1 { 17 }		Tarifa	353	Cant.	20\newline
Pr  50	CantTr 1 { 17 }		Tarifa	438	Cant.	40\newline
Pr  51	CantTr 1 { 17 }		Tarifa	675	Cant.	60\newline
Pr  52	CantTr 1 { 18 }		Tarifa	72	Cant.	10\newline
Pr  53	CantTr 1 { 18 }		Tarifa	89	Cant.	20\newline
Pr  54	CantTr 1 { 18 }		Tarifa	135	Cant.	30\newline
Pr  55	CantTr 2 { 17 18 }	Tarifa	469	Cant.	15\newline
Pr  56	CantTr 2 { 17 18 }	Tarifa	527	Cant.	26\newline
Pr  57	CantTr 2 { 17 18 }	Tarifa	810	Cant.	34\newline
Pr  58	CantTr 1 { 19 }		Tarifa	443	Cant.	20\newline
Pr  59	CantTr 1 { 19 }		Tarifa	480	Cant.	21\newline
Pr  60	CantTr 1 { 19 }		Tarifa	852	Cant.	21\newline
Pr  61	CantTr 1 { 20 }		Tarifa	392	Cant.	13\newline
Pr  62	CantTr 1 { 20 }		Tarifa	433	Cant.	15\newline
Pr  63	CantTr 1 { 20 }		Tarifa	538	Cant.	17\newline
Pr  64	CantTr 1 { 20 }		Tarifa	830	Cant.	19\newline
Pr  65	CantTr 1 { 21 }		Tarifa	342	Cant.	20\newline
Pr  66	CantTr 1 { 21 }		Tarifa	655	Cant.	29\newline
Pr  67	CantTr 1 { 22 }		Tarifa	252	Cant.	17\newline
Pr  68	CantTr 1 { 22 }		Tarifa	278	Cant.	24\newline
Pr  69	CantTr 1 { 22 }		Tarifa	344	Cant.	30\newline
Pr  70	CantTr 1 { 22 }		Tarifa	530	Cant.	45\newline
Pr  71	CantTr 1 { 23 }		Tarifa	675	Cant.	20\newline
Pr  72	CantTr 1 { 23 }		Tarifa	1023	Cant.	30\newline
Pr  73	CantTr 1 { 23 }		Tarifa	1283	Cant.	30\newline
Pr  74	CantTr 2 { 22 23 }	Tarifa	927	Cant.	15\newline
Pr  75	CantTr 2 { 22 23 }	Tarifa	1301	Cant.	19\newline
Pr  76	CantTr 2 { 22 23 }	Tarifa	1627	Cant.	23\newline
Pr  77	CantTr 2 { 22 23 }	Tarifa	1813	Cant.	25\newline
Pr  78	CantTr 1 { 24 }		Tarifa	603	Cant.	15\newline
Pr  79	CantTr 1 { 24 }		Tarifa	629	Cant.	16\newline
Pr  80	CantTr 1 { 24 }		Tarifa	881	Cant.	28\newline
Pr  81	CantTr 2 { 22 24 }	Tarifa	855	Cant.	20\newline
Pr  82	CantTr 2 { 22 24 }	Tarifa	907	Cant.	20\newline
Pr  83	CantTr 2 { 22 24 }	Tarifa	1225	Cant.	40\newline
Pr  84	CantTr 1 { 25 }		Tarifa	276	Cant.	20\newline
Pr  85	CantTr 1 { 25 }		Tarifa	358	Cant.	20\newline
Pr  86	CantTr 1 { 25 }		Tarifa	444	Cant.	20\newline
Pr  87	CantTr 1 { 25 }		Tarifa	685	Cant.	35\newline
Pr  88	CantTr 1 { 26 }		Tarifa	316	Cant.	20\newline
Pr  89	CantTr 1 { 26 }		Tarifa	445	Cant.	20\newline
Pr  90	CantTr 1 { 26 }		Tarifa	685	Cant.	46\newline
Pr  91	CantTr 1 { 27 }		Tarifa	353	Cant.	20\newline
Pr  92	CantTr 1 { 27 }		Tarifa	445	Cant.	20\newline
Pr  93	CantTr 1 { 27 }		Tarifa	685	Cant.	20\newline
Pr  94	CantTr 1 { 28 }		Tarifa	316	Cant.	17\newline
Pr  95	CantTr 1 { 28 }		Tarifa	445	Cant.	19\newline
Pr  96	CantTr 1 { 28 }		Tarifa	685	Cant.	20\newline
Pr  97	CantTr 2 { 28 27 }	Tarifa	714	Cant.	20\newline
Pr  98	CantTr 2 { 28 27 }	Tarifa	890	Cant.	21\newline
Pr  99	CantTr 2 { 28 27 }	Tarifa	1370	Cant.	34\newline
Pr 100	CantTr 1 { 29 }		Tarifa	238	Cant.	20\newline
Pr 101	CantTr 1 { 29 }		Tarifa	295	Cant.	20\newline
Pr 102	CantTr 1 { 29 }		Tarifa	454	Cant.	20\newline
Pr 103	CantTr 1 { 30 }		Tarifa	255	Cant.	20\newline
Pr 104	CantTr 1 { 30 }		Tarifa	317	Cant.	20\newline
Pr 105	CantTr 1 { 30 }		Tarifa	488	Cant.	30\newline
Pr 106	CantTr 1 { 31 }		Tarifa	494	Cant.	20\newline
Pr 107	CantTr 1 { 31 }		Tarifa	612	Cant.	20\newline
Pr 108	CantTr 1 { 31 }		Tarifa	942	Cant.	30\newline
Pr 109	CantTr 1 { 32 }		Tarifa	486	Cant.	20\newline
Pr 110	CantTr 1 { 32 }		Tarifa	775	Cant.	22\newline
Pr 111	CantTr 1 { 32 }		Tarifa	934	Cant.	32\newline
Pr 112	CantTr 2 { 30 32 }	Tarifa	741	Cant.	30\newline
Pr 113	CantTr 2 { 30 32 }	Tarifa	1092	Cant.	30\newline
Pr 114	CantTr 2 { 30 32 }	Tarifa	1422	Cant.	60\newline
Pr 115	CantTr 1 { 33 }		Tarifa	526	Cant.	15\newline
Pr 116	CantTr 1 { 33 }		Tarifa	1019	Cant.	19\newline
Pr 117	CantTr 1 { 34 }		Tarifa	944	Cant.	75\newline
Pr 118	CantTr 1 { 35 }		Tarifa	612	Cant.	21\newline
Pr 119	CantTr 1 { 35 }		Tarifa	619	Cant.	22\newline
Pr 120	CantTr 1 { 35 }		Tarifa	678	Cant.	23\newline
Pr 121	CantTr 1 { 35 }		Tarifa	875	Cant.	32\newline
Pr 122	CantTr 1 { 35 }		Tarifa	1329	Cant.	46\newline
Pr 123	CantTr 1 { 35 }		Tarifa	1636	Cant.	55\newline
Pr 124	CantTr 1 { 35 }		Tarifa	3095	Cant.	66\newline
Pr 125	CantTr 1 { 36 }		Tarifa	333	Cant.	20\newline
Pr 126	CantTr 1 { 36 }		Tarifa	457	Cant.	20\newline
Pr 127	CantTr 1 { 36 }		Tarifa	704	Cant.	20\newline
Pr 128	CantTr 2 { 36 35 }	Tarifa	945	Cant.	10\newline
Pr 129	CantTr 2 { 36 35 }	Tarifa	952	Cant.	21\newline
Pr 130	CantTr 2 { 36 35 }	Tarifa	1011	Cant.	25\newline
Pr 131	CantTr 2 { 36 35 }	Tarifa	1332	Cant.	36\newline
Pr 132	CantTr 2 { 36 35 }	Tarifa	1786	Cant.	41\newline
Pr 133	CantTr 2 { 36 35 }	Tarifa	2340	Cant.	42\newline
Pr 134	CantTr 2 { 36 35 }	Tarifa	3789	Cant.	45\newline
Pr 135	CantTr 1 { 37 }		Tarifa	704	Cant.	67\newline
Pr 136	CantTr 1 { 38 }		Tarifa	653	Cant.	78\newline
Pr 137	CantTr 2 { 36 38 }	Tarifa	986	Cant.	30\newline
Pr 138	CantTr 2 { 36 38 }	Tarifa	1110	Cant.	35\newline
Pr 139	CantTr 2 { 36 38 }	Tarifa	1357	Cant.	35\newline
Pr 140	CantTr 1 { 39 }		Tarifa	642	Cant.	20\newline
Pr 141	CantTr 1 { 39 }		Tarifa	691	Cant.	20\newline
Pr 142	CantTr 1 { 39 }		Tarifa	1335	Cant.	20\newline
Pr 143	CantTr 1 { 39 }		Tarifa	1649	Cant.	30\newline
Pr 144	CantTr 1 { 39 }		Tarifa	1990	Cant.	40\newline
Pr 145	CantTr 1 { 39 }		Tarifa	3101	Cant.	50\newline
Pr 146	CantTr 1 { 40 }		Tarifa	324	Cant.	20\newline
Pr 147	CantTr 1 { 40 }		Tarifa	358	Cant.	20\newline
Pr 148	CantTr 1 { 40 }		Tarifa	514	Cant.	20\newline
Pr 149	CantTr 1 { 40 }		Tarifa	533	Cant.	20\newline
Pr 150	CantTr 1 { 40 }		Tarifa	944	Cant.	33\newline
Pr 151	CantTr 1 { 40 }		Tarifa	1285	Cant.	38\newline
Pr 152	CantTr 1 { 41 }		Tarifa	675	Cant.	43\newline
Pr 153	CantTr 1 { 41 }		Tarifa	710	Cant.	50\newline
Pr 154	CantTr 1 { 41 }		Tarifa	1798	Cant.	58\newline
Pr 155	CantTr 1 { 42 }		Tarifa	352	Cant.	60\newline
Pr 156	CantTr 1 { 43 }		Tarifa	986	Cant.	60\newline
Pr 157	CantTr 1 { 44 }		Tarifa	155	Cant.	70\newline
Pr 158	CantTr 1 { 44 }		Tarifa	300	Cant.	20\newline
Pr 159	CantTr 1 { 45 }		Tarifa	268	Cant.	30\newline
Pr 160	CantTr 1 { 45 }		Tarifa	519	Cant.	35\newline
Pr 161	CantTr 2 { 40 44 }	Tarifa	479	Cant.	20\newline
Pr 162	CantTr 2 { 40 44 }	Tarifa	513	Cant.	20\newline
Pr 163	CantTr 2 { 40 44 }	Tarifa	668	Cant.	30\newline
Pr 164	CantTr 2 { 40 44 }	Tarifa	688	Cant.	30\newline
Pr 165	CantTr 2 { 40 44 }	Tarifa	1244	Cant.	40\newline
Pr 166	CantTr 2 { 40 44 }	Tarifa	1555	Cant.	40\newline
\newline
End Productos\newline

\end{scriptsize}

\begin{center}
	\begin{tabular}{|c|c|}
	\hline
	Cantidad de tramos & 45 \\
	\hline
	Cantidad de productos & 166 \\
	\hline
	\end{tabular}
\end{center}
\newpage
%------------------------------------------------------------------------------------%

\subsection{Archivo de prueba 4}


\begin{scriptsize}
Red Aerolineas con 50 tramos y 241 productos.\newline
\newline
Init Tramos\newline
\newline
Tr 1	Cant.	200	BandaH. M\newline
Tr 2	Cant.	150	BandaH. M\newline
Tr 3	Cant.	180	BandaH. M\newline
Tr 4	Cant.	200	BandaH. M\newline
Tr 5	Cant.	200	BandaH. M\newline
Tr 6	Cant.	120	BandaH. M\newline
Tr 7	Cant.	160	BandaH. M\newline
Tr 8	Cant.	220	BandaH. M\newline
Tr 9	Cant.	180	BandaH. M\newline
Tr 10	Cant.	180	BandaH. M\newline
Tr 11	Cant.	200	BandaH. M\newline
Tr 12	Cant.	100	BandaH. M\newline
Tr 13	Cant.	150	BandaH. M\newline
Tr 14	Cant.	150	BandaH. M\newline
Tr 15	Cant.	150	BandaH. M\newline
Tr 16	Cant.	180	BandaH. M\newline
Tr 17	Cant.	160	BandaH. M\newline
Tr 18	Cant.	180	BandaH. M\newline
Tr 19	Cant.	200	BandaH. M\newline
Tr 20	Cant.	150	BandaH. M\newline
Tr 21	Cant.	200	BandaH. M\newline
Tr 22	Cant.	180	BandaH. M\newline
Tr 23	Cant.	160	BandaH. M\newline
Tr 24	Cant.	130	BandaH. M\newline
Tr 25	Cant.	150	BandaH. M\newline
Tr 26	Cant.	180	BandaH. M\newline
Tr 27	Cant.	180	BandaH. M\newline
Tr 28	Cant.	190	BandaH. M\newline
Tr 29	Cant.	200	BandaH. M\newline
Tr 30	Cant.	200	BandaH. M\newline
Tr 31	Cant.	200	BandaH. M\newline
Tr 32	Cant.	150	BandaH. M\newline
Tr 33	Cant.	200	BandaH. M\newline
Tr 34	Cant.	180	BandaH. M\newline
Tr 35	Cant.	200	BandaH. M\newline
Tr 36	Cant.	180	BandaH. M\newline
Tr 37	Cant.	180	BandaH. M\newline
Tr 38	Cant.	180	BandaH. M\newline
Tr 39	Cant.	150	BandaH. M\newline
Tr 40	Cant.	200	BandaH. M\newline
Tr 41	Cant.	150	BandaH. M\newline
Tr 42	Cant.	200	BandaH. M\newline
Tr 43	Cant.	200	BandaH. M\newline
Tr 44	Cant.	180	BandaH. M\newline
Tr 45	Cant.	190	BandaH. M\newline
Tr 46	Cant.	191	BandaH. M\newline
Tr 47	Cant.	200	BandaH. M\newline
Tr 48	Cant.	200	BandaH. M\newline
Tr 49	Cant.	200	BandaH. M\newline
Tr 50	Cant.	180	BandaH. M\newline
\newline
End Tramos\newline
\newline
Init Productos\newline
\newline
Pr   1	CantTr 1 { 1 }			Tarifa	288	Cant.	20\newline
Pr   2	CantTr 1 { 1 }			Tarifa	360	Cant.	20\newline
Pr   3	CantTr 1 { 1 }			Tarifa	555	Cant.	50\newline
Pr   4	CantTr 1 { 2 }			Tarifa	308	Cant.	70\newline
Pr   5	CantTr 1 { 3 }			Tarifa	597	Cant.	15\newline
Pr   6	CantTr 1 { 3 }			Tarifa	950	Cant.	15\newline
Pr   7	CantTr 1 { 3 }			Tarifa	1145	Cant.	85\newline
Pr   8	CantTr 2 { 1 3 }		Tarifa	885	Cant.	21\newline
Pr   9	CantTr 2 { 1 3 }		Tarifa	1310	Cant.	28\newline
Pr  10	CantTr 2 { 1 3 }		Tarifa	1700	Cant.	45\newline
Pr  11	CantTr 1 { 4 }			Tarifa	864	Cant.	103\newline
Pr  12	CantTr 2 { 2 4 }		Tarifa	1172	Cant.	54\newline
Pr  13	CantTr 1 { 5 }			Tarifa	321	Cant.	20\newline
Pr  14	CantTr 1 { 5 }			Tarifa	356	Cant.	23\newline
Pr  15	CantTr 1 { 5 }			Tarifa	440	Cant.	35\newline
Pr  16	CantTr 1 { 5 }			Tarifa	678	Cant.	67\newline
Pr  17	CantTr 1 { 7 }			Tarifa	289	Cant.	20\newline
Pr  18	CantTr 1 { 7 }			Tarifa	556	Cant.	30\newline
Pr  19	CantTr 1 { 6 }			Tarifa	329	Cant.	20\newline
Pr  20	CantTr 1 { 6 }			Tarifa	410	Cant.	20\newline
Pr  21	CantTr 1 { 6 }			Tarifa	633	Cant.	20\newline
Pr  22	CantTr 1 { 8 }			Tarifa	941	Cant.	62\newline
Pr  23	CantTr 2 { 2 8 }		Tarifa	1249	Cant.	35\newline
Pr  24	CantTr 1 { 9 }			Tarifa	197	Cant.	90\newline
Pr  25	CantTr 1 { 10 }			Tarifa	303	Cant.	15\newline
Pr  26	CantTr 1 { 10 }			Tarifa	377	Cant.	19\newline
Pr  27	CantTr 1 { 10 }			Tarifa	581	Cant.	33\newline
Pr  28	CantTr 1 { 11 }			Tarifa	551	Cant.	10\newline
Pr  29	CantTr 1 { 11 }			Tarifa	857	Cant.	15\newline
Pr  30	CantTr 1 { 11 }			Tarifa	1061	Cant.	68\newline
Pr  31	CantTr 1 { 12 }			Tarifa	358	Cant.	30\newline
Pr  32	CantTr 1 { 12 }			Tarifa	685	Cant.	40\newline
Pr  33	CantTr 1 { 13 }			Tarifa	859	Cant.	85\newline
Pr  34	CantTr 1 { 14 }			Tarifa	937	Cant.	10\newline
Pr  35	CantTr 1 { 14 }			Tarifa	985	Cant.	12\newline
Pr  36	CantTr 1 { 14 }			Tarifa	1555	Cant.	21\newline
Pr  37	CantTr 1 { 14 }			Tarifa	1888	Cant.	58\newline
Pr  38	CantTr 1 { 14 }			Tarifa	1437	Cant.	35\newline
Pr  39	CantTr 2 { 13 14 }		Tarifa	1769	Cant.	20\newline
Pr  40	CantTr 2 { 13 14 }		Tarifa	1844	Cant.	30\newline
Pr  41	CantTr 2 { 13 14 }		Tarifa	2414	Cant.	40\newline
Pr  42	CantTr 2 { 13 14 }		Tarifa	2747	Cant.	50\newline
Pr  43	CantTr 2 { 13 14 }		Tarifa	2296	Cant.	35\newline
Pr  44	CantTr 1 { 15 }			Tarifa	492	Cant.	20\newline
Pr  45	CantTr 1 { 15 }			Tarifa	944	Cant.	36\newline
Pr  46	CantTr 1 { 16 }			Tarifa	397	Cant.	30\newline
Pr  47	CantTr 1 { 16 }			Tarifa	493	Cant.	32\newline
Pr  48	CantTr 1 { 16 }			Tarifa	760	Cant.	33\newline
Pr  49	CantTr 1 { 17 }			Tarifa	353	Cant.	20\newline
Pr  50	CantTr 1 { 17 }			Tarifa	438	Cant.	40\newline
Pr  51	CantTr 1 { 17 }			Tarifa	675	Cant.	60\newline
Pr  52	CantTr 1 { 18 }			Tarifa	72	Cant.	10\newline
Pr  53	CantTr 1 { 18 }			Tarifa	89	Cant.	20\newline
Pr  54	CantTr 1 { 18 }			Tarifa	135	Cant.	30\newline
Pr  55	CantTr 2 { 17 18 }		Tarifa	469	Cant.	15\newline
Pr  56	CantTr 2 { 17 18 }		Tarifa	527	Cant.	26\newline
Pr  57	CantTr 2 { 17 18 }		Tarifa	810	Cant.	34\newline
Pr  58	CantTr 1 { 19 }			Tarifa	443	Cant.	20\newline
Pr  59	CantTr 1 { 19 }			Tarifa	480	Cant.	21\newline
Pr  60	CantTr 1 { 19 }			Tarifa	852	Cant.	21\newline
Pr  61	CantTr 1 { 20 }			Tarifa	392	Cant.	13\newline
Pr  62	CantTr 1 { 20 }			Tarifa	433	Cant.	15\newline
Pr  63	CantTr 1 { 20 }			Tarifa	538	Cant.	17\newline
Pr  64	CantTr 1 { 20 }			Tarifa	830	Cant.	19\newline
Pr  65	CantTr 1 { 21 }			Tarifa	342	Cant.	20\newline
Pr  66	CantTr 1 { 21 }			Tarifa	655	Cant.	29\newline
Pr  67	CantTr 1 { 22 }			Tarifa	252	Cant.	17\newline
Pr  68	CantTr 1 { 22 }			Tarifa	278	Cant.	24\newline
Pr  69	CantTr 1 { 22 }			Tarifa	344	Cant.	30\newline
Pr  70	CantTr 1 { 22 }			Tarifa	530	Cant.	45\newline
Pr  71	CantTr 1 { 23 }			Tarifa	675	Cant.	20\newline
Pr  72	CantTr 1 { 23 }			Tarifa	1023	Cant.	30\newline
Pr  73	CantTr 1 { 23 }			Tarifa	1283	Cant.	30\newline
Pr  74	CantTr 2 { 22 23 }		Tarifa	927	Cant.	15\newline
Pr  75	CantTr 2 { 22 23 }		Tarifa	1301	Cant.	19\newline
Pr  76	CantTr 2 { 22 23 }		Tarifa	1627	Cant.	23\newline
Pr  77	CantTr 2 { 22 23 }		Tarifa	1813	Cant.	25\newline
Pr  78	CantTr 1 { 24 }			Tarifa	603	Cant.	15\newline
Pr  79	CantTr 1 { 24 }			Tarifa	629	Cant.	16\newline
Pr  80	CantTr 1 { 24 }			Tarifa	881	Cant.	28\newline
Pr  81	CantTr 2 { 22 24 }		Tarifa	855	Cant.	20\newline
Pr  82	CantTr 2 { 22 24 }		Tarifa	907	Cant.	20\newline
Pr  83	CantTr 2 { 22 24 }		Tarifa	1225	Cant.	40\newline
Pr  84	CantTr 1 { 25 }			Tarifa	276	Cant.	20\newline
Pr  85	CantTr 1 { 25 }			Tarifa	358	Cant.	20\newline
Pr  86	CantTr 1 { 25 }			Tarifa	444	Cant.	20\newline
Pr  87	CantTr 1 { 25 }			Tarifa	685	Cant.	35\newline
Pr  88	CantTr 1 { 26 }			Tarifa	316	Cant.	20\newline
Pr  89	CantTr 1 { 26 }			Tarifa	445	Cant.	20\newline
Pr  90	CantTr 1 { 26 }			Tarifa	685	Cant.	46\newline
Pr  91	CantTr 1 { 27 }			Tarifa	353	Cant.	20\newline
Pr  92	CantTr 1 { 27 }			Tarifa	445	Cant.	20\newline
Pr  93	CantTr 1 { 27 }			Tarifa	685	Cant.	20\newline
Pr  94	CantTr 1 { 28 }			Tarifa	316	Cant.	17\newline
Pr  95	CantTr 1 { 28 }			Tarifa	445	Cant.	19\newline
Pr  96	CantTr 1 { 28 }			Tarifa	685	Cant.	20\newline
Pr  97	CantTr 2 { 28 27 }		Tarifa	714	Cant.	20\newline
Pr  98	CantTr 2 { 28 27 }		Tarifa	890	Cant.	21\newline
Pr  99	CantTr 2 { 28 27 }		Tarifa	1370	Cant.	34\newline
Pr 100	CantTr 1 { 29 }			Tarifa	238	Cant.	20\newline
Pr 101	CantTr 1 { 29 }			Tarifa	295	Cant.	20\newline
Pr 102	CantTr 1 { 29 }			Tarifa	454	Cant.	20\newline
Pr 103	CantTr 1 { 30 }			Tarifa	255	Cant.	20\newline
Pr 104	CantTr 1 { 30 }			Tarifa	317	Cant.	20\newline
Pr 105	CantTr 1 { 30 }			Tarifa	488	Cant.	30\newline
Pr 106	CantTr 1 { 31 }			Tarifa	494	Cant.	20\newline
Pr 107	CantTr 1 { 31 }			Tarifa	612	Cant.	20\newline
Pr 108	CantTr 1 { 31 }			Tarifa	942	Cant.	30\newline
Pr 109	CantTr 1 { 32 }			Tarifa	486	Cant.	20\newline
Pr 110	CantTr 1 { 32 }			Tarifa	775	Cant.	22\newline
Pr 111	CantTr 1 { 32 }			Tarifa	934	Cant.	32\newline
Pr 112	CantTr 2 { 30 32 }		Tarifa	741	Cant.	30\newline
Pr 113	CantTr 2 { 30 32 }		Tarifa	1092	Cant.	30\newline
Pr 114	CantTr 2 { 30 32 }		Tarifa	1422	Cant.	60\newline
Pr 115	CantTr 1 { 33 }			Tarifa	526	Cant.	15\newline
Pr 116	CantTr 1 { 33 }			Tarifa	1019	Cant.	19\newline
Pr 117	CantTr 1 { 34 }			Tarifa	944	Cant.	75\newline
Pr 118	CantTr 1 { 35 }			Tarifa	612	Cant.	21\newline
Pr 119	CantTr 1 { 35 }			Tarifa	619	Cant.	22\newline
Pr 120	CantTr 1 { 35 }			Tarifa	678	Cant.	23\newline
Pr 121	CantTr 1 { 35 }			Tarifa	875	Cant.	32\newline
Pr 122	CantTr 1 { 35 }			Tarifa	1329	Cant.	46\newline
Pr 123	CantTr 1 { 35 }			Tarifa	1636	Cant.	55\newline
Pr 124	CantTr 1 { 35 }			Tarifa	3095	Cant.	66\newline
Pr 125	CantTr 1 { 36 }			Tarifa	333	Cant.	20\newline
Pr 126	CantTr 1 { 36 }			Tarifa	457	Cant.	20\newline
Pr 127	CantTr 1 { 36 }			Tarifa	704	Cant.	20\newline
Pr 128	CantTr 2 { 36 35 }		Tarifa	945	Cant.	10\newline
Pr 129	CantTr 2 { 36 35 }		Tarifa	952	Cant.	21\newline
Pr 130	CantTr 2 { 36 35 }		Tarifa	1011	Cant.	25\newline
Pr 131	CantTr 2 { 36 35 }		Tarifa	1332	Cant.	36\newline
Pr 132	CantTr 2 { 36 35 }		Tarifa	1786	Cant.	41\newline
Pr 133	CantTr 2 { 36 35 }		Tarifa	2340	Cant.	42\newline
Pr 134	CantTr 2 { 36 35 }		Tarifa	3789	Cant.	45\newline
Pr 135	CantTr 1 { 37 }			Tarifa	704	Cant.	67\newline
Pr 136	CantTr 1 { 38 }			Tarifa	653	Cant.	78\newline
Pr 137	CantTr 2 { 36 38 }		Tarifa	986	Cant.	30\newline
Pr 138	CantTr 2 { 36 38 }		Tarifa	1110	Cant.	35\newline
Pr 139	CantTr 2 { 36 38 }		Tarifa	1357	Cant.	35\newline
Pr 140	CantTr 1 { 39 }			Tarifa	642	Cant.	20\newline
Pr 141	CantTr 1 { 39 }			Tarifa	691	Cant.	20\newline
Pr 142	CantTr 1 { 39 }			Tarifa	1335	Cant.	20\newline
Pr 143	CantTr 1 { 39 }			Tarifa	1649	Cant.	30\newline
Pr 144	CantTr 1 { 39 }			Tarifa	1990	Cant.	40\newline
Pr 145	CantTr 1 { 39 }			Tarifa	3101	Cant.	50\newline
Pr 146	CantTr 1 { 40 }			Tarifa	324	Cant.	20\newline
Pr 147	CantTr 1 { 40 }			Tarifa	358	Cant.	20\newline
Pr 148	CantTr 1 { 40 }			Tarifa	514	Cant.	20\newline
Pr 149	CantTr 1 { 40 }			Tarifa	533	Cant.	20\newline
Pr 150	CantTr 1 { 40 }			Tarifa	944	Cant.	33\newline
Pr 151	CantTr 1 { 40 }			Tarifa	1285	Cant.	38\newline
Pr 152	CantTr 1 { 41 }			Tarifa	675	Cant.	43\newline
Pr 153	CantTr 1 { 41 }			Tarifa	710	Cant.	50\newline
Pr 154	CantTr 1 { 41 }			Tarifa	1798	Cant.	58\newline
Pr 155	CantTr 1 { 42 }			Tarifa	352	Cant.	60\newline
Pr 156	CantTr 1 { 43 }			Tarifa	986	Cant.	60\newline
Pr 157	CantTr 1 { 44 }			Tarifa	155	Cant.	70\newline
Pr 158	CantTr 1 { 44 }			Tarifa	300	Cant.	20\newline
Pr 159	CantTr 1 { 45 }			Tarifa	268	Cant.	30\newline
Pr 160	CantTr 1 { 45 }			Tarifa	519	Cant.	35\newline
Pr 161	CantTr 2 { 40 44 }		Tarifa	479	Cant.	20\newline
Pr 162	CantTr 2 { 40 44 }		Tarifa	513	Cant.	20\newline
Pr 163	CantTr 2 { 40 44 }		Tarifa	668	Cant.	30\newline
Pr 164	CantTr 2 { 40 44 }		Tarifa	688	Cant.	30\newline
Pr 165	CantTr 2 { 40 44 }		Tarifa	1244	Cant.	40\newline
Pr 166	CantTr 2 { 40 44 }		Tarifa	1555	Cant.	40\newline
Pr 167	CantTr 1 { 46 }			Tarifa	150	Cant.	20\newline
Pr 168	CantTr 1 { 46 }			Tarifa	200	Cant.	20\newline
Pr 169	CantTr 1 { 46 }			Tarifa	300	Cant.	20\newline
Pr 170	CantTr 1 { 46 }			Tarifa	350	Cant.	20\newline
Pr 171	CantTr 2 { 22 46 }		Tarifa	402	Cant.	30\newline
Pr 172	CantTr 2 { 22 46 }		Tarifa	478	Cant.	30\newline
Pr 173	CantTr 2 { 22 46 }		Tarifa	644	Cant.	35\newline
Pr 174	CantTr 2 { 22 46 }		Tarifa	880	Cant.	45\newline
Pr 175	CantTr 3 { 22 46 33 }		Tarifa	928	Cant.	20\newline
Pr 176	CantTr 3 { 22 46 33 }		Tarifa	1004	Cant.	20\newline
Pr 177	CantTr 3 { 22 46 33 }		Tarifa	1663	Cant.	30\newline
Pr 178	CantTr 3 { 22 46 33 }		Tarifa	1899	Cant.	40\newline
Pr 179	CantTr 4 { 22 46 33 45 }	Tarifa	1196	Cant.	15\newline
Pr 180	CantTr 4 { 22 46 33 45 }	Tarifa	1272	Cant.	15\newline
Pr 181	CantTr 4 { 22 46 33 45 }	Tarifa	1523	Cant.	15\newline
Pr 182	CantTr 4 { 22 46 33 45 }	Tarifa	2182	Cant.	30\newline
Pr 183	CantTr 4 { 22 46 33 45 }	Tarifa	2418	Cant.	30\newline
Pr 184	CantTr 1 { 47 }			Tarifa	450	Cant.	10\newline
Pr 185	CantTr 1 { 47 }			Tarifa	550	Cant.	10\newline
Pr 186	CantTr 1 { 47 }			Tarifa	650	Cant.	10\newline
Pr 187	CantTr 1 { 47 }			Tarifa	750	Cant.	50\newline
Pr 188	CantTr 1 { 47 }			Tarifa	850	Cant.	50\newline
Pr 189	CantTr 1 { 48 }			Tarifa	300	Cant.	10\newline
Pr 190	CantTr 1 { 48 }			Tarifa	380	Cant.	11\newline
Pr 191	CantTr 1 { 48 }			Tarifa	500	Cant.	42\newline
Pr 192	CantTr 1 { 49 }			Tarifa	200	Cant.	10\newline
Pr 193	CantTr 1 { 49 }			Tarifa	230	Cant.	15\newline
Pr 194	CantTr 1 { 49 }			Tarifa	400	Cant.	20\newline
Pr 195	CantTr 1 { 49 }			Tarifa	560	Cant.	35\newline
Pr 196	CantTr 1 { 49 }			Tarifa	710	Cant.	45\newline
Pr 197	CantTr 5 { 46 47 1 48 49 }	Tarifa	2000	Cant.	20\newline
Pr 198	CantTr 5 { 46 47 1 48 49 }	Tarifa	2200	Cant.	20\newline
Pr 199	CantTr 5 { 46 47 1 48 49 }	Tarifa	2530	Cant.	20\newline
Pr 200	CantTr 5 { 46 47 1 48 49 }	Tarifa	2680	Cant.	30\newline
Pr 201	CantTr 5 { 46 47 1 48 49 }	Tarifa	2900	Cant.	30\newline
Pr 202	CantTr 1 { 50 }			Tarifa	230	Cant.	10\newline
Pr 203	CantTr 1 { 50 }			Tarifa	330	Cant.	10\newline
Pr 204	CantTr 1 { 50 }			Tarifa	430	Cant.	10\newline
Pr 205	CantTr 1 { 50 }			Tarifa	530	Cant.	20\newline
Pr 206	CantTr 1 { 50 }			Tarifa	630	Cant.	20\newline
Pr 207	CantTr 1 { 50 }			Tarifa	730	Cant.	20\newline
Pr 208	CantTr 6 { 46 47 1 48 49 50 }	Tarifa	2500	Cant.	20\newline
Pr 209	CantTr 6 { 46 47 1 48 49 50 }	Tarifa	2600	Cant.	20\newline
Pr 210	CantTr 6 { 46 47 1 48 49 50 }	Tarifa	2700	Cant.	20\newline
Pr 211	CantTr 6 { 46 47 1 48 49 50 }	Tarifa	2890	Cant.	20\newline
Pr 212	CantTr 6 { 46 47 1 48 49 50 }	Tarifa	3000	Cant.	35\newline
Pr 213	CantTr 6 { 46 47 1 48 49 50 }	Tarifa	3150	Cant.	35\newline
Pr 214	CantTr 3 { 48 49 50 }		Tarifa	600	Cant.	15\newline
Pr 215	CantTr 3 { 48 49 50 }		Tarifa	700	Cant.	15\newline
Pr 216	CantTr 3 { 48 49 50 }		Tarifa	750	Cant.	35\newline
Pr 217	CantTr 3 { 48 49 50 }		Tarifa	900	Cant.	35\newline
Pr 218	CantTr 4 { 1 48 49 50 }		Tarifa	850	Cant.	20\newline
Pr 219	CantTr 4 { 1 48 49 50 }		Tarifa	950	Cant.	21\newline
Pr 220	CantTr 4 { 1 48 49 50 }		Tarifa	1150	Cant.	25\newline
Pr 221	CantTr 4 { 1 48 49 50 }		Tarifa	1250	Cant.	30\newline
Pr 222	CantTr 4 { 1 48 49 50 }		Tarifa	1350	Cant.	33\newline
Pr 223	CantTr 4 { 1 48 49 50 }		Tarifa	1450	Cant.	40\newline
Pr 224	CantTr 3 { 47 1 48 }		Tarifa	580	Cant.	20\newline
Pr 225	CantTr 3 { 47 1 48 }		Tarifa	590	Cant.	20\newline
Pr 226	CantTr 3 { 47 1 48 }		Tarifa	600	Cant.	20\newline
Pr 227	CantTr 3 { 47 1 48 }		Tarifa	610	Cant.	25\newline
Pr 228	CantTr 3 { 47 1 48 }		Tarifa	650	Cant.	25\newline
Pr 229	CantTr 3 { 47 1 48 }		Tarifa	700	Cant.	30\newline
Pr 230	CantTr 3 { 47 1 48 }		Tarifa	750	Cant.	30\newline
Pr 231	CantTr 3 { 47 1 48 }		Tarifa	800	Cant.	30\newline
Pr 232	CantTr 2 { 49 50 }		Tarifa	430	Cant.	20\newline
Pr 233	CantTr 2 { 49 50 }		Tarifa	500	Cant.	21\newline
Pr 234	CantTr 2 { 49 50 }		Tarifa	560	Cant.	23\newline
Pr 235	CantTr 2 { 49 50 }		Tarifa	590	Cant.	26\newline
Pr 236	CantTr 2 { 49 50 }		Tarifa	610	Cant.	28\newline
Pr 237	CantTr 4 { 46 47 1 48 }		Tarifa	1500	Cant.	20\newline
Pr 238	CantTr 4 { 46 47 1 48 }		Tarifa	1600	Cant.	20\newline
Pr 239	CantTr 4 { 46 47 1 48 }		Tarifa	1700	Cant.	20\newline
Pr 240	CantTr 4 { 46 47 1 48 }		Tarifa	1830	Cant.	30\newline
Pr 241	CantTr 4 { 46 47 1 48 }		Tarifa	1850	Cant.	35\newline
\newline
End Productos\newline

\end{scriptsize}

\begin{center}
	\begin{tabular}{|c|c|}
	\hline
	Cantidad de tramos & 50 \\
	\hline
	Cantidad de productos & 241 \\
	\hline
	\end{tabular}
\end{center}
\newpage
%------------------------------------------------------------------------------------%

\subsection{Archivo de prueba 5}

\begin{scriptsize}
Red Aerolineas con 83 tramos y 396 productos.\newline
\newline
Init Tramos \newline
\newline
Tr 1	Cant.	200	BandaH. M\newline
Tr 2	Cant.	150	BandaH. M\newline
Tr 3	Cant.	180	BandaH. M\newline
Tr 4	Cant.	200	BandaH. M\newline
Tr 5	Cant.	200	BandaH. M\newline
Tr 6	Cant.	120	BandaH. M\newline
Tr 7	Cant.	160	BandaH. M\newline
Tr 8	Cant.	220	BandaH. M\newline
Tr 9	Cant.	180	BandaH. M\newline
Tr 10	Cant.	180	BandaH. M\newline
Tr 11	Cant.	200	BandaH. M\newline
Tr 12	Cant.	100	BandaH. M\newline
Tr 13	Cant.	150	BandaH. M\newline
Tr 14	Cant.	150	BandaH. M\newline
Tr 15	Cant.	150	BandaH. M\newline
Tr 16	Cant.	180	BandaH. M\newline
Tr 17	Cant.	160	BandaH. M\newline
Tr 18	Cant.	180	BandaH. M\newline
Tr 19	Cant.	200	BandaH. M\newline
Tr 20	Cant.	150	BandaH. M\newline
Tr 21	Cant.	200	BandaH. M\newline
Tr 22	Cant.	180	BandaH. M\newline
Tr 23	Cant.	160	BandaH. M\newline
Tr 24	Cant.	130	BandaH. M\newline
Tr 25	Cant.	150	BandaH. M\newline
Tr 26	Cant.	180	BandaH. M\newline
Tr 27	Cant.	180	BandaH. M\newline
Tr 28	Cant.	190	BandaH. M\newline
Tr 29	Cant.	200	BandaH. M\newline
Tr 30	Cant.	200	BandaH. M\newline
Tr 31	Cant.	200	BandaH. M\newline
Tr 32	Cant.	150	BandaH. M\newline
Tr 33	Cant.	200	BandaH. M\newline
Tr 34	Cant.	180	BandaH. M\newline
Tr 35	Cant.	200	BandaH. M\newline
Tr 36	Cant.	180	BandaH. M\newline
Tr 37	Cant.	180	BandaH. M\newline
Tr 38	Cant.	180	BandaH. M\newline
Tr 39	Cant.	150	BandaH. M\newline
Tr 40	Cant.	200	BandaH. M\newline
Tr 41	Cant.	150	BandaH. M\newline
Tr 42	Cant.	200	BandaH. M\newline
Tr 43	Cant.	200	BandaH. M\newline
Tr 44	Cant.	180	BandaH. M\newline
Tr 45	Cant.	190	BandaH. M\newline
Tr 46	Cant.	191	BandaH. M\newline
Tr 47	Cant.	200	BandaH. M\newline
Tr 48	Cant.	200	BandaH. M\newline
Tr 49	Cant.	200	BandaH. M\newline
Tr 50	Cant.	180	BandaH. M\newline
Tr 51	Cant.	150	BandaH. M\newline
Tr 52	Cant.	120	BandaH. M\newline
Tr 53	Cant.	130	BandaH. M\newline
Tr 54	Cant.	180	BandaH. M\newline
Tr 55	Cant.	190	BandaH. M\newline
Tr 56	Cant.	200	BandaH. M\newline
Tr 57	Cant.	160	BandaH. M\newline
Tr 58	Cant.	200	BandaH. M\newline
Tr 59	Cant.	180	BandaH. M\newline
Tr 60	Cant.	170	BandaH. M\newline
Tr 61	Cant.	200	BandaH. M\newline
Tr 62	Cant.	100	BandaH. M\newline
Tr 63	Cant.	120	BandaH. M\newline
Tr 64	Cant.	150	BandaH. M\newline
Tr 65	Cant.	180	BandaH. M\newline
Tr 66	Cant.	200	BandaH. M\newline
Tr 67	Cant.	180	BandaH. M\newline
Tr 68	Cant.	180	BandaH. M\newline
Tr 69	Cant.	200	BandaH. M\newline
Tr 70	Cant.	150	BandaH. M\newline
Tr 71	Cant.	180	BandaH. M\newline
Tr 72	Cant.	150	BandaH. M\newline
Tr 73	Cant.	180	BandaH. M\newline
Tr 74	Cant.	200	BandaH. M\newline
Tr 75	Cant.	180	BandaH. M\newline
Tr 76	Cant.	180	BandaH. M\newline
Tr 77	Cant.	200	BandaH. M\newline
Tr 78	Cant.	150	BandaH. M\newline
Tr 79	Cant.	180	BandaH. M\newline
Tr 80	Cant.	150	BandaH. M\newline
Tr 81	Cant.	180	BandaH. M\newline
Tr 82	Cant.	200	BandaH. M\newline
Tr 83	Cant.	180	BandaH. M\newline
\newline
End Tramos\newline
\newline
Init Productos \newline
\newline
Pr 1	CantTr 	1	{ 1 }	Tarifa	288	Cant.	20\newline
Pr 2	CantTr 	1	{ 1 }	Tarifa	360	Cant.	20\newline
Pr 3	CantTr 	1	{ 1 }	Tarifa	555	Cant.	50\newline
Pr 4	CantTr 	1	{ 2 }	Tarifa	308	Cant.	70\newline
Pr 5	CantTr 	1	{ 3 }	Tarifa	597	Cant.	15\newline
Pr 6	CantTr 	1	{ 3 }	Tarifa	950	Cant.	15\newline
Pr 7	CantTr 	1	{ 3 }	Tarifa	1145	Cant.	85\newline
Pr 8	CantTr 	2	{ 1 3 }	Tarifa	885	Cant.	21\newline
Pr 9	CantTr 	2	{ 1 3 }	Tarifa	1310	Cant.	28\newline
Pr 10	CantTr 	2	{ 1 3 }	Tarifa	1700	Cant.	45\newline
Pr 11	CantTr 	1	{ 4 }	Tarifa	864	Cant.	103\newline
Pr 12	CantTr 	2	{ 2 4 }	Tarifa	1172	Cant.	54\newline
Pr 13	CantTr 	1	{ 5 }	Tarifa	321	Cant.	20\newline
Pr 14	CantTr 	1	{ 5 }	Tarifa	356	Cant.	23\newline
Pr 15	CantTr 	1	{ 5 }	Tarifa	440	Cant.	35\newline
Pr 16	CantTr 	1	{ 5 }	Tarifa	678	Cant.	67\newline
Pr 17	CantTr 	1	{ 7 }	Tarifa	289	Cant.	20\newline
Pr 18	CantTr 	1	{ 7 }	Tarifa	556	Cant.	30\newline
Pr 19	CantTr 	1	{ 6 }	Tarifa	329	Cant.	20\newline
Pr 20	CantTr 	1	{ 6 }	Tarifa	410	Cant.	20\newline
Pr 21	CantTr 	1	{ 6 }	Tarifa	633	Cant.	20\newline
Pr 22	CantTr 	1	{ 8 }	Tarifa	941	Cant.	62\newline
Pr 23	CantTr 	2	{ 2 8 }	Tarifa	1249	Cant.	35\newline
Pr 24	CantTr 	1	{ 9 }	Tarifa	197	Cant.	90\newline
Pr 25	CantTr 	1	{ 10 }	Tarifa	303	Cant.	15\newline
Pr 26	CantTr 	1	{ 10 }	Tarifa	377	Cant.	19\newline
Pr 27	CantTr 	1	{ 10 }	Tarifa	581	Cant.	33\newline
Pr 28	CantTr 	1	{ 11 }	Tarifa	551	Cant.	10\newline
Pr 29	CantTr 	1	{ 11 }	Tarifa	857	Cant.	15\newline
Pr 30	CantTr 	1	{ 11 }	Tarifa	1061	Cant.	68\newline
Pr 31	CantTr 	1	{ 12 }	Tarifa	358	Cant.	30\newline
Pr 32	CantTr 	1	{ 12 }	Tarifa	685	Cant.	40\newline
Pr 33	CantTr 	1	{ 13 }	Tarifa	859	Cant.	85\newline
Pr 34	CantTr 	1	{ 14 }	Tarifa	937	Cant.	10\newline
Pr 35	CantTr 	1	{ 14 }	Tarifa	985	Cant.	12\newline
Pr 36	CantTr 	1	{ 14 }	Tarifa	1555	Cant.	21\newline
Pr 37	CantTr 	1	{ 14 }	Tarifa	1888	Cant.	58\newline
Pr 38	CantTr 	1	{ 14 }	Tarifa	1437	Cant.	35\newline
Pr 39	CantTr 	2	{ 13 14 }	Tarifa	1769	Cant.	20\newline
Pr 40	CantTr 	2	{ 13 14 }	Tarifa	1844	Cant.	30\newline
Pr 41	CantTr 	2	{ 13 14 }	Tarifa	2414	Cant.	40\newline
Pr 42	CantTr 	2	{ 13 14 }	Tarifa	2747	Cant.	50\newline
Pr 43	CantTr 	2	{ 13 14 }	Tarifa	2296	Cant.	35
Pr 44	CantTr 	1	{ 15 }	Tarifa	492	Cant.	20\newline
Pr 45	CantTr 	1	{ 15 }	Tarifa	944	Cant.	36\newline
Pr 46	CantTr 	1	{ 16 }	Tarifa	397	Cant.	30\newline
Pr 47	CantTr 	1	{ 16 }	Tarifa	493	Cant.	32\newline
Pr 48	CantTr 	1	{ 16 }	Tarifa	760	Cant.	33\newline
Pr 49	CantTr 	1	{ 17 }	Tarifa	353	Cant.	20\newline
Pr 50	CantTr 	1	{ 17 }	Tarifa	438	Cant.	40\newline
Pr 51	CantTr 	1	{ 17 }	Tarifa	675	Cant.	60\newline
Pr 52	CantTr 	1	{ 18 }	Tarifa	72	Cant.	10\newline
Pr 53	CantTr 	1	{ 18 }	Tarifa	89	Cant.	20\newline
Pr 54	CantTr 	1	{ 18 }	Tarifa	135	Cant.	30\newline
Pr 55	CantTr 	2	{ 17 18 }	Tarifa	469	Cant.	15\newline
Pr 56	CantTr 	2	{ 17 18 }	Tarifa	527	Cant.	26\newline
Pr 57	CantTr 	2	{ 17 18 }	Tarifa	810	Cant.	34\newline
Pr 58	CantTr 	1	{ 19 }	Tarifa	443	Cant.	20\newline
Pr 59	CantTr 	1	{ 19 }	Tarifa	480	Cant.	21\newline
Pr 60	CantTr 	1	{ 19 }	Tarifa	852	Cant.	21\newline
Pr 61	CantTr 	1	{ 20 }	Tarifa	392	Cant.	13\newline
Pr 62	CantTr 	1	{ 20 }	Tarifa	433	Cant.	15\newline
Pr 63	CantTr 	1	{ 20 }	Tarifa	538	Cant.	17\newline
Pr 64	CantTr 	1	{ 20 }	Tarifa	830	Cant.	19\newline
Pr 65	CantTr 	1	{ 21 }	Tarifa	342	Cant.	20\newline
Pr 66	CantTr 	1	{ 21 }	Tarifa	655	Cant.	29\newline
Pr 67	CantTr 	1	{ 22 }	Tarifa	252	Cant.	17\newline
Pr 68	CantTr 	1	{ 22 }	Tarifa	278	Cant.	24\newline
Pr 69	CantTr 	1	{ 22 }	Tarifa	344	Cant.	30\newline
Pr 70	CantTr 	1	{ 22 }	Tarifa	530	Cant.	45\newline
Pr 71	CantTr 	1	{ 23 }	Tarifa	675	Cant.	20\newline
Pr 72	CantTr 	1	{ 23 }	Tarifa	1023	Cant.	30\newline
Pr 73	CantTr 	1	{ 23 }	Tarifa	1283	Cant.	30\newline
Pr 74	CantTr 	2	{ 22 23 }	Tarifa	927	Cant.	15\newline
Pr 75	CantTr 	2	{ 22 23 }	Tarifa	1301	Cant.	19\newline
Pr 76	CantTr 	2	{ 22 23 }	Tarifa	1627	Cant.	23\newline
Pr 77	CantTr 	2	{ 22 23 }	Tarifa	1813	Cant.	25\newline
Pr 78	CantTr 	1	{ 24 }	Tarifa	603	Cant.	15\newline
Pr 79	CantTr 	1	{ 24 }	Tarifa	629	Cant.	16\newline
Pr 80	CantTr 	1	{ 24 }	Tarifa	881	Cant.	28\newline
Pr 81	CantTr 	2	{ 22 24 }	Tarifa	855	Cant.	20\newline
Pr 82	CantTr 	2	{ 22 24 }	Tarifa	907	Cant.	20\newline
Pr 83	CantTr 	2	{ 22 24 }	Tarifa	1225	Cant.	40\newline
Pr 84	CantTr 	1	{ 25 }	Tarifa	276	Cant.	20\newline
Pr 85	CantTr 	1	{ 25 }	Tarifa	358	Cant.	20\newline
Pr 86	CantTr 	1	{ 25 }	Tarifa	444	Cant.	20\newline
Pr 87	CantTr 	1	{ 25 }	Tarifa	685	Cant.	35\newline
Pr 88	CantTr 	1	{ 26 }	Tarifa	316	Cant.	20\newline
Pr 89	CantTr 	1	{ 26 }	Tarifa	445	Cant.	20\newline
Pr 90	CantTr 	1	{ 26 }	Tarifa	685	Cant.	46\newline
Pr 91	CantTr 	1	{ 27 }	Tarifa	353	Cant.	20\newline
Pr 92	CantTr 	1	{ 27 }	Tarifa	445	Cant.	20\newline
Pr 93	CantTr 	1	{ 27 }	Tarifa	685	Cant.	20\newline
Pr 94	CantTr 	1	{ 28 }	Tarifa	316	Cant.	17\newline
Pr 95	CantTr 	1	{ 28 }	Tarifa	445	Cant.	19\newline
Pr 96	CantTr 	1	{ 28 }	Tarifa	685	Cant.	20\newline
Pr 97	CantTr 	2	{ 28 27 }	Tarifa	714	Cant.	20\newline
Pr 98	CantTr 	2	{ 28 27 }	Tarifa	890	Cant.	21\newline
Pr 99	CantTr 	2	{ 28 27 }	Tarifa	1370	Cant.	34\newline
Pr 100	CantTr 	1	{ 29 }	Tarifa	238	Cant.	20\newline
Pr 101	CantTr 	1	{ 29 }	Tarifa	295	Cant.	20\newline
Pr 102	CantTr 	1	{ 29 }	Tarifa	454	Cant.	20\newline
Pr 103	CantTr 	1	{ 30 }	Tarifa	255	Cant.	20\newline
Pr 104	CantTr 	1	{ 30 }	Tarifa	317	Cant.	20\newline
Pr 105	CantTr 	1	{ 30 }	Tarifa	488	Cant.	30\newline
Pr 106	CantTr 	1	{ 31 }	Tarifa	494	Cant.	20\newline
Pr 107	CantTr 	1	{ 31 }	Tarifa	612	Cant.	20\newline
Pr 108	CantTr 	1	{ 31 }	Tarifa	942	Cant.	30\newline
Pr 109	CantTr 	1	{ 32 }	Tarifa	486	Cant.	20\newline
Pr 110	CantTr 	1	{ 32 }	Tarifa	775	Cant.	22\newline
Pr 111	CantTr 	1	{ 32 }	Tarifa	934	Cant.	32\newline
Pr 112	CantTr 	2	{ 30 32 }	Tarifa	741	Cant.	30\newline
Pr 113	CantTr 	2	{ 30 32 }	Tarifa	1092	Cant.	30\newline
Pr 114	CantTr 	2	{ 30 32 }	Tarifa	1422	Cant.	60\newline
Pr 115	CantTr 	1	{ 33 }	Tarifa	526	Cant.	15\newline
Pr 116	CantTr 	1	{ 33 }	Tarifa	1019	Cant.	19\newline
Pr 117	CantTr 	1	{ 34 }	Tarifa	944	Cant.	75\newline
Pr 118	CantTr 	1	{ 35 }	Tarifa	612	Cant.	21\newline
Pr 119	CantTr 	1	{ 35 }	Tarifa	619	Cant.	22\newline
Pr 120	CantTr 	1	{ 35 }	Tarifa	678	Cant.	23\newline
Pr 121	CantTr 	1	{ 35 }	Tarifa	875	Cant.	32\newline
Pr 122	CantTr 	1	{ 35 }	Tarifa	1329	Cant.	46\newline
Pr 123	CantTr 	1	{ 35 }	Tarifa	1636	Cant.	55\newline
Pr 124	CantTr 	1	{ 35 }	Tarifa	3095	Cant.	66\newline
Pr 125	CantTr 	1	{ 36 }	Tarifa	333	Cant.	20\newline
Pr 126	CantTr 	1	{ 36 }	Tarifa	457	Cant.	20\newline
Pr 127	CantTr 	1	{ 36 }	Tarifa	704	Cant.	20\newline
Pr 128	CantTr 	2	{ 36 35 }	Tarifa	945	Cant.	10\newline
Pr 129	CantTr 	2	{ 36 35 }	Tarifa	952	Cant.	21\newline
Pr 130	CantTr 	2	{ 36 35 }	Tarifa	1011	Cant.	25\newline
Pr 131	CantTr 	2	{ 36 35 }	Tarifa	1332	Cant.	36\newline
Pr 132	CantTr 	2	{ 36 35 }	Tarifa	1786	Cant.	41\newline
Pr 133	CantTr 	2	{ 36 35 }	Tarifa	2340	Cant.	42\newline
Pr 134	CantTr 	2	{ 36 35 }	Tarifa	3789	Cant.	45\newline
Pr 135	CantTr 	1	{ 37 }	Tarifa	704	Cant.	67\newline
Pr 136	CantTr 	1	{ 38 }	Tarifa	653	Cant.	78\newline
Pr 137	CantTr 	2	{ 36 38 }	Tarifa	986	Cant.	30\newline
Pr 138	CantTr 	2	{ 36 38 }	Tarifa	1110	Cant.	35\newline
Pr 139	CantTr 	2	{ 36 38 }	Tarifa	1357	Cant.	35\newline
Pr 140	CantTr 	1	{ 39 }	Tarifa	642	Cant.	20\newline
Pr 141	CantTr 	1	{ 39 }	Tarifa	691	Cant.	20\newline
Pr 142	CantTr 	1	{ 39 }	Tarifa	1335	Cant.	20\newline
Pr 143	CantTr 	1	{ 39 }	Tarifa	1649	Cant.	30\newline
Pr 144	CantTr 	1	{ 39 }	Tarifa	1990	Cant.	40\newline
Pr 145	CantTr 	1	{ 39 }	Tarifa	3101	Cant.	50\newline
Pr 146	CantTr 	1	{ 40 }	Tarifa	324	Cant.	20\newline
Pr 147	CantTr 	1	{ 40 }	Tarifa	358	Cant.	20\newline
Pr 148	CantTr 	1	{ 40 }	Tarifa	514	Cant.	20\newline
Pr 149	CantTr 	1	{ 40 }	Tarifa	533	Cant.	20\newline
Pr 150	CantTr 	1	{ 40 }	Tarifa	944	Cant.	33\newline
Pr 151	CantTr 	1	{ 40 }	Tarifa	1285	Cant.	38\newline
Pr 152	CantTr 	1	{ 41 }	Tarifa	675	Cant.	43\newline
Pr 153	CantTr 	1	{ 41 }	Tarifa	710	Cant.	50\newline
Pr 154	CantTr 	1	{ 41 }	Tarifa	1798	Cant.	58\newline
Pr 155	CantTr 	1	{ 42 }	Tarifa	352	Cant.	60\newline
Pr 156	CantTr 	1	{ 43 }	Tarifa	986	Cant.	60\newline
Pr 157	CantTr 	1	{ 44 }	Tarifa	155	Cant.	70\newline
Pr 158	CantTr 	1	{ 44 }	Tarifa	300	Cant.	20\newline
Pr 159	CantTr 	1	{ 45 }	Tarifa	268	Cant.	30\newline
Pr 160	CantTr 	1	{ 45 }	Tarifa	519	Cant.	35\newline
Pr 161	CantTr 	2	{ 40 44 }	Tarifa	479	Cant.	20\newline
Pr 162	CantTr 	2	{ 40 44 }	Tarifa	513	Cant.	20\newline
Pr 163	CantTr 	2	{ 40 44 }	Tarifa	668	Cant.	30\newline
Pr 164	CantTr 	2	{ 40 44 }	Tarifa	688	Cant.	30\newline
Pr 165	CantTr 	2	{ 40 44 }	Tarifa	1244	Cant.	40\newline
Pr 166	CantTr 	2	{ 40 44 }	Tarifa	1555	Cant.	40\newline
Pr 167	CantTr 	1	{ 46 }	Tarifa	150	Cant.	20\newline
Pr 168	CantTr 	1	{ 46 }	Tarifa	200	Cant.	20\newline
Pr 169	CantTr 	1	{ 46 }	Tarifa	300	Cant.	20\newline
Pr 170	CantTr 	1	{ 46 }	Tarifa	350	Cant.	20\newline
Pr 171	CantTr 	2	{ 22 46 }	Tarifa	402	Cant.	30\newline
Pr 172	CantTr 	2	{ 22 46 }	Tarifa	478	Cant.	30\newline
Pr 173	CantTr 	2	{ 22 46 }	Tarifa	644	Cant.	35\newline
Pr 174	CantTr 	2	{ 22 46 }	Tarifa	880	Cant.	45\newline
Pr 175	CantTr 	3	{ 22 46 33 }	Tarifa	928	Cant.	20\newline
Pr 176	CantTr 	3	{ 22 46 33 }	Tarifa	1004	Cant.	20\newline
Pr 177	CantTr 	3	{ 22 46 33 }	Tarifa	1663	Cant.	30\newline
Pr 178	CantTr 	3	{ 22 46 33 }	Tarifa	1899	Cant.	40\newline
Pr 179	CantTr 	4	{ 22 46 33 45 }	Tarifa	1196	Cant.	15\newline
Pr 180	CantTr 	4	{ 22 46 33 45 }	Tarifa	1272	Cant.	15\newline
Pr 181	CantTr 	4	{ 22 46 33 45 }	Tarifa	1523	Cant.	15\newline
Pr 182	CantTr 	4	{ 22 46 33 45 }	Tarifa	2182	Cant.	30\newline
Pr 183	CantTr 	4	{ 22 46 33 45 }	Tarifa	2418	Cant.	30\newline
Pr 184	CantTr 	1	{ 47 }	Tarifa	450	Cant.	10\newline
Pr 185	CantTr 	1	{ 47 }	Tarifa	550	Cant.	10\newline
Pr 186	CantTr 	1	{ 47 }	Tarifa	650	Cant.	10\newline
Pr 187	CantTr 	1	{ 47 }	Tarifa	750	Cant.	50\newline
Pr 188	CantTr 	1	{ 47 }	Tarifa	850	Cant.	50\newline
Pr 189	CantTr 	1	{ 48 }	Tarifa	300	Cant.	10\newline
Pr 190	CantTr 	1	{ 48 }	Tarifa	380	Cant.	11\newline
Pr 191	CantTr 	1	{ 48 }	Tarifa	500	Cant.	42\newline
Pr 192	CantTr 	1	{ 49 }	Tarifa	200	Cant.	10\newline
Pr 193	CantTr 	1	{ 49 }	Tarifa	230	Cant.	15\newline
Pr 194	CantTr 	1	{ 49 }	Tarifa	400	Cant.	20\newline
Pr 195	CantTr 	1	{ 49 }	Tarifa	560	Cant.	35\newline
Pr 196	CantTr 	1	{ 49 }	Tarifa	710	Cant.	45\newline
Pr 197	CantTr 	5	{ 46 47 1 48 49 }	Tarifa	2000	Cant.	20\newline
Pr 198	CantTr 	5	{ 46 47 1 48 49 }	Tarifa	2200	Cant.	20\newline
Pr 199	CantTr 	5	{ 46 47 1 48 49 }	Tarifa	2530	Cant.	20\newline
Pr 200	CantTr 	5	{ 46 47 1 48 49 }	Tarifa	2680	Cant.	30\newline
Pr 201	CantTr 	5	{ 46 47 1 48 49 }	Tarifa	2900	Cant.	30\newline
Pr 202	CantTr 	1	{ 50 }	Tarifa	230	Cant.	10\newline
Pr 203	CantTr 	1	{ 50 }	Tarifa	330	Cant.	10\newline
Pr 204	CantTr 	1	{ 50 }	Tarifa	430	Cant.	10\newline
Pr 205	CantTr 	1	{ 50 }	Tarifa	530	Cant.	20\newline
Pr 206	CantTr 	1	{ 50 }	Tarifa	630	Cant.	20\newline
Pr 207	CantTr 	1	{ 50 }	Tarifa	730	Cant.	20\newline
Pr 208	CantTr 	6	{ 46 47 1 48 49 50 }	Tarifa	2500	Cant.	20\newline
Pr 209	CantTr 	6	{ 46 47 1 48 49 50 }	Tarifa	2600	Cant.	20\newline
Pr 210	CantTr 	6	{ 46 47 1 48 49 50 }	Tarifa	2700	Cant.	20\newline
Pr 211	CantTr 	6	{ 46 47 1 48 49 50 }	Tarifa	2890	Cant.	20\newline
Pr 212	CantTr 	6	{ 46 47 1 48 49 50 }	Tarifa	3000	Cant.	35\newline
Pr 213	CantTr 	6	{ 46 47 1 48 49 50 }	Tarifa	3150	Cant.	35\newline
Pr 214	CantTr 	3	{ 48 49 50 }	Tarifa	600	Cant.	15\newline
Pr 215	CantTr 	3	{ 48 49 50 }	Tarifa	700	Cant.	15\newline
Pr 216	CantTr 	3	{ 48 49 50 }	Tarifa	750	Cant.	35\newline
Pr 217	CantTr 	3	{ 48 49 50 }	Tarifa	900	Cant.	35\newline
Pr 218	CantTr 	4	{ 1 48 49 50 }	Tarifa	850	Cant.	20\newline
Pr 219	CantTr 	4	{ 1 48 49 50 }	Tarifa	950	Cant.	21\newline
Pr 220	CantTr 	4	{ 1 48 49 50 }	Tarifa	1150	Cant.	25\newline
Pr 221	CantTr 	4	{ 1 48 49 50 }	Tarifa	1250	Cant.	30\newline
Pr 222	CantTr 	4	{ 1 48 49 50 }	Tarifa	1350	Cant.	33\newline
Pr 223	CantTr 	4	{ 1 48 49 50 }	Tarifa	1450	Cant.	40\newline
Pr 224	CantTr 	3	{ 47 1 48 }	Tarifa	580	Cant.	20\newline
Pr 225	CantTr 	3	{ 47 1 48 }	Tarifa	590	Cant.	20\newline
Pr 226	CantTr 	3	{ 47 1 48 }	Tarifa	600	Cant.	20\newline
Pr 227	CantTr 	3	{ 47 1 48 }	Tarifa	610	Cant.	25\newline
Pr 228	CantTr 	3	{ 47 1 48 }	Tarifa	650	Cant.	25\newline
Pr 229	CantTr 	3	{ 47 1 48 }	Tarifa	700	Cant.	30\newline
Pr 230	CantTr 	3	{ 47 1 48 }	Tarifa	750	Cant.	30\newline
Pr 231	CantTr 	3	{ 47 1 48 }	Tarifa	800	Cant.	30\newline
Pr 232	CantTr 	2	{ 49 50 }	Tarifa	430	Cant.	20\newline
Pr 233	CantTr 	2	{ 49 50 }	Tarifa	500	Cant.	21\newline
Pr 234	CantTr 	2	{ 49 50 }	Tarifa	560	Cant.	23\newline
Pr 235	CantTr 	2	{ 49 50 }	Tarifa	590	Cant.	26\newline
Pr 236	CantTr 	2	{ 49 50 }	Tarifa	610	Cant.	28\newline
Pr 237	CantTr 	4	{ 46 47 1 48 }	Tarifa	1500	Cant.	20\newline
Pr 238	CantTr 	4	{ 46 47 1 48 }	Tarifa	1600	Cant.	20\newline
Pr 239	CantTr 	4	{ 46 47 1 48 }	Tarifa	1700	Cant.	20\newline
Pr 240	CantTr 	4	{ 46 47 1 48 }	Tarifa	1830	Cant.	30\newline
Pr 241	CantTr 	4	{ 46 47 1 48 }	Tarifa	1850	Cant.	35\newline
Pr 242	CantTr 	1	{ 51 }	Tarifa	500	Cant.	35\newline
Pr 243	CantTr 	1	{ 51 }	Tarifa	550	Cant.	35\newline
Pr 244	CantTr 	1	{ 51 }	Tarifa	560	Cant.	15\newline
Pr 245	CantTr 	1	{ 51 }	Tarifa	570	Cant.	15\newline
Pr 246	CantTr 	1	{ 52 }	Tarifa	400	Cant.	35\newline
Pr 247	CantTr 	1	{ 52 }	Tarifa	420	Cant.	35\newline
Pr 248	CantTr 	1	{ 52 }	Tarifa	430	Cant.	20\newline
Pr 249	CantTr 	1	{ 52 }	Tarifa	460	Cant.	21\newline
Pr 250	CantTr 	1	{ 53 }	Tarifa	610	Cant.	25\newline
Pr 251	CantTr 	1	{ 53 }	Tarifa	650	Cant.	30\newline
Pr 252	CantTr 	1	{ 53 }	Tarifa	660	Cant.	33\newline
Pr 253	CantTr 	1	{ 53 }	Tarifa	680	Cant.	40\newline
Pr 254	CantTr 	1	{ 54 }	Tarifa	1200	Cant.	20\newline
Pr 255	CantTr 	1	{ 54 }	Tarifa	1300	Cant.	20\newline
Pr 256	CantTr 	1	{ 54 }	Tarifa	1560	Cant.	20\newline
Pr 257	CantTr 	1	{ 54 }	Tarifa	1800	Cant.	25\newline
Pr 258	CantTr 	1	{ 55 }	Tarifa	700	Cant.	25\newline
Pr 259	CantTr 	1	{ 55 }	Tarifa	750	Cant.	30\newline
Pr 260	CantTr 	1	{ 55 }	Tarifa	780	Cant.	30\newline
Pr 261	CantTr 	1	{ 55 }	Tarifa	790	Cant.	30\newline
Pr 262	CantTr 	1	{ 55 }	Tarifa	800	Cant.	20\newline
Pr 263	CantTr 	1	{ 56 }	Tarifa	822	Cant.	21\newline
Pr 264	CantTr 	1	{ 56 }	Tarifa	830	Cant.	23\newline
Pr 265	CantTr 	1	{ 56 }	Tarifa	860	Cant.	26\newline
Pr 266	CantTr 	1	{ 56 }	Tarifa	880	Cant.	28\newline
Pr 267	CantTr 	1	{ 56 }	Tarifa	890	Cant.	20\newline
Pr 268	CantTr 	1	{ 57 }	Tarifa	350	Cant.	20\newline
Pr 269	CantTr 	1	{ 57 }	Tarifa	360	Cant.	20\newline
Pr 270	CantTr 	1	{ 57 }	Tarifa	380	Cant.	30\newline
Pr 271	CantTr 	1	{ 57 }	Tarifa	400	Cant.	35\newline
Pr 272	CantTr 	1	{ 58 }	Tarifa	900	Cant.	35\newline
Pr 273	CantTr 	1	{ 58 }	Tarifa	930	Cant.	35\newline
Pr 274	CantTr 	1	{ 58 }	Tarifa	970	Cant.	15\newline
Pr 275	CantTr 	1	{ 58 }	Tarifa	990	Cant.	15\newline
Pr 276	CantTr 	1	{ 59 }	Tarifa	300	Cant.	35\newline
Pr 277	CantTr 	1	{ 59 }	Tarifa	330	Cant.	35\newline
Pr 278	CantTr 	1	{ 59 }	Tarifa	336	Cant.	20\newline
Pr 279	CantTr 	1	{ 59 }	Tarifa	355	Cant.	21\newline
Pr 280	CantTr 	1	{ 60 }	Tarifa	200	Cant.	25\newline
Pr 281	CantTr 	1	{ 60 }	Tarifa	250	Cant.	30\newline
Pr 282	CantTr 	1	{ 60 }	Tarifa	280	Cant.	33\newline
Pr 283	CantTr 	1	{ 60 }	Tarifa	290	Cant.	40\newline
Pr 284	CantTr 	1	{ 61 }	Tarifa	310	Cant.	20\newline
Pr 285	CantTr 	1	{ 61 }	Tarifa	350	Cant.	20\newline
Pr 286	CantTr 	1	{ 61 }	Tarifa	370	Cant.	20\newline
Pr 287	CantTr 	1	{ 61 }	Tarifa	380	Cant.	25\newline
Pr 288	CantTr 	1	{ 62 }	Tarifa	660	Cant.	25\newline
Pr 289	CantTr 	1	{ 62 }	Tarifa	670	Cant.	30\newline
Pr 290	CantTr 	1	{ 62 }	Tarifa	680	Cant.	30\newline
Pr 291	CantTr 	1	{ 62 }	Tarifa	690	Cant.	30\newline
Pr 292	CantTr 	1	{ 63 }	Tarifa	780	Cant.	20\newline
Pr 293	CantTr 	1	{ 63 }	Tarifa	790	Cant.	21\newline
Pr 294	CantTr 	1	{ 63 }	Tarifa	800	Cant.	23\newline
Pr 295	CantTr 	1	{ 63 }	Tarifa	850	Cant.	26\newline
Pr 296	CantTr 	1	{ 64 }	Tarifa	700	Cant.	28\newline
Pr 297	CantTr 	1	{ 64 }	Tarifa	750	Cant.	20\newline
Pr 298	CantTr 	1	{ 64 }	Tarifa	770	Cant.	20\newline
Pr 299	CantTr 	1	{ 64 }	Tarifa	780	Cant.	20\newline
Pr 300	CantTr 	1	{ 65 }	Tarifa	1100	Cant.	30\newline
Pr 301	CantTr 	1	{ 65 }	Tarifa	1150	Cant.	35\newline
Pr 302	CantTr 	1	{ 65 }	Tarifa	1350	Cant.	35\newline
Pr 303	CantTr 	1	{ 65 }	Tarifa	1450	Cant.	35\newline
Pr 304	CantTr 	1	{ 66 }	Tarifa	300	Cant.	15\newline
Pr 305	CantTr 	1	{ 66 }	Tarifa	350	Cant.	15\newline
Pr 306	CantTr 	1	{ 66 }	Tarifa	400	Cant.	35\newline
Pr 307	CantTr 	1	{ 66 }	Tarifa	450	Cant.	35\newline
Pr 308	CantTr 	1	{ 67 }	Tarifa	930	Cant.	20\newline
Pr 309	CantTr 	1	{ 67 }	Tarifa	940	Cant.	21\newline
Pr 310	CantTr 	1	{ 67 }	Tarifa	950	Cant.	25\newline
Pr 311	CantTr 	1	{ 67 }	Tarifa	960	Cant.	30\newline
Pr 312	CantTr 	1	{ 68 }	Tarifa	450	Cant.	33\newline
Pr 313	CantTr 	1	{ 68 }	Tarifa	460	Cant.	40\newline
Pr 314	CantTr 	1	{ 68 }	Tarifa	480	Cant.	20\newline
Pr 315	CantTr 	1	{ 68 }	Tarifa	490	Cant.	20\newline
Pr 316	CantTr 	1	{ 69 }	Tarifa	500	Cant.	20\newline
Pr 317	CantTr 	1	{ 69 }	Tarifa	510	Cant.	25\newline
Pr 318	CantTr 	1	{ 69 }	Tarifa	580	Cant.	25\newline
Pr 319	CantTr 	1	{ 69 }	Tarifa	590	Cant.	30\newline
Pr 320	CantTr 	1	{ 70 }	Tarifa	340	Cant.	30\newline
Pr 321	CantTr 	1	{ 70 }	Tarifa	440	Cant.	30\newline
Pr 322	CantTr 	1	{ 70 }	Tarifa	540	Cant.	20\newline
Pr 323	CantTr 	1	{ 70 }	Tarifa	640	Cant.	21\newline
Pr 324	CantTr 	1	{ 71 }	Tarifa	1350	Cant.	23\newline
Pr 325	CantTr 	1	{ 71 }	Tarifa	1400	Cant.	26\newline
Pr 326	CantTr 	1	{ 71 }	Tarifa	1550	Cant.	28\newline
Pr 327	CantTr 	1	{ 71 }	Tarifa	1600	Cant.	20\newline
Pr 328	CantTr 	1	{ 72 }	Tarifa	800	Cant.	20\newline
Pr 329	CantTr 	1	{ 72 }	Tarifa	820	Cant.	20\newline
Pr 330	CantTr 	1	{ 72 }	Tarifa	830	Cant.	20\newline
Pr 331	CantTr 	1	{ 72 }	Tarifa	850	Cant.	30\newline
Pr 332	CantTr 	1	{ 73 }	Tarifa	750	Cant.	35\newline
Pr 333	CantTr 	1	{ 73 }	Tarifa	770	Cant.	35\newline
Pr 334	CantTr 	1	{ 73 }	Tarifa	780	Cant.	35\newline
Pr 335	CantTr 	1	{ 73 }	Tarifa	790	Cant.	15\newline
Pr 336	CantTr 	1	{ 74 }	Tarifa	750	Cant.	15\newline
Pr 337	CantTr 	1	{ 74 }	Tarifa	770	Cant.	35\newline
Pr 338	CantTr 	1	{ 74 }	Tarifa	780	Cant.	35\newline
Pr 339	CantTr 	1	{ 74 }	Tarifa	790	Cant.	20\newline
Pr 340	CantTr 	1	{ 75 }	Tarifa	750	Cant.	21\newline
Pr 341	CantTr 	1	{ 75 }	Tarifa	770	Cant.	25\newline
Pr 342	CantTr 	1	{ 75 }	Tarifa	780	Cant.	30\newline
Pr 343	CantTr 	1	{ 75 }	Tarifa	790	Cant.	33\newline
Pr 344	CantTr 	1	{ 76 }	Tarifa	750	Cant.	40\newline
Pr 345	CantTr 	1	{ 76 }	Tarifa	770	Cant.	20\newline
Pr 346	CantTr 	1	{ 76 }	Tarifa	780	Cant.	20\newline
Pr 347	CantTr 	1	{ 76 }	Tarifa	790	Cant.	20\newline
Pr 348	CantTr 	1	{ 77 }	Tarifa	500	Cant.	25\newline
Pr 349	CantTr 	1	{ 77 }	Tarifa	560	Cant.	25\newline
Pr 350	CantTr 	1	{ 77 }	Tarifa	590	Cant.	30\newline
Pr 351	CantTr 	1	{ 77 }	Tarifa	610	Cant.	30\newline
Pr 352	CantTr 	1	{ 78 }	Tarifa	800	Cant.	30\newline
Pr 353	CantTr 	1	{ 78 }	Tarifa	840	Cant.	20\newline
Pr 354	CantTr 	1	{ 78 }	Tarifa	860	Cant.	21\newline
Pr 355	CantTr 	1	{ 78 }	Tarifa	890	Cant.	23\newline
Pr 356	CantTr 	1	{ 79 }	Tarifa	910	Cant.	20\newline
Pr 357	CantTr 	1	{ 79 }	Tarifa	930	Cant.	20\newline
Pr 358	CantTr 	1	{ 79 }	Tarifa	950	Cant.	20\newline
Pr 359	CantTr 	1	{ 79 }	Tarifa	960	Cant.	30\newline
Pr 360	CantTr 	1	{ 80 }	Tarifa	750	Cant.	35\newline
Pr 361	CantTr 	1	{ 80 }	Tarifa	770	Cant.	35\newline
Pr 362	CantTr 	1	{ 80 }	Tarifa	780	Cant.	35\newline
Pr 363	CantTr 	1	{ 80 }	Tarifa	790	Cant.	15\newline
Pr 364	CantTr 	1	{ 81 }	Tarifa	1105	Cant.	15\newline
Pr 365	CantTr 	1	{ 81 }	Tarifa	1190	Cant.	35\newline
Pr 366	CantTr 	1	{ 81 }	Tarifa	1310	Cant.	35\newline
Pr 367	CantTr 	1	{ 81 }	Tarifa	1450	Cant.	20\newline
Pr 368	CantTr 	1	{ 82 }	Tarifa	700	Cant.	21\newline
Pr 369	CantTr 	1	{ 82 }	Tarifa	750	Cant.	25\newline
Pr 370	CantTr 	1	{ 82 }	Tarifa	760	Cant.	30\newline
Pr 371	CantTr 	1	{ 82 }	Tarifa	790	Cant.	33\newline
Pr 372	CantTr 	1	{ 83 }	Tarifa	600	Cant.	40\newline
Pr 373	CantTr 	1	{ 83 }	Tarifa	610	Cant.	20\newline
Pr 374	CantTr 	1	{ 83 }	Tarifa	630	Cant.	20\newline
Pr 375	CantTr 	1	{ 83 }	Tarifa	660	Cant.	20\newline
Pr 376	CantTr 	2	{ 51 72 }	Tarifa	1300	Cant.	25\newline
Pr 377	CantTr 	2	{ 51 72 }	Tarifa	1350	Cant.	25\newline
Pr 378	CantTr 	2	{ 51 72 }	Tarifa	1410	Cant.	30\newline
Pr 379	CantTr 	2	{ 51 72 }	Tarifa	1450	Cant.	30\newline
Pr 380	CantTr 	3	{ 51 72 9  }	Tarifa	1500	Cant.	30\newline
Pr 381	CantTr 	3	{ 51 72 9  }	Tarifa	1570	Cant.	20\newline
Pr 382	CantTr 	3	{ 51 72 9  }	Tarifa	1610	Cant.	21\newline
Pr 383	CantTr 	3	{ 51 72 9  }	Tarifa	1630	Cant.	23\newline
Pr 384	CantTr 	3	{ 51 72 9  }	Tarifa	1705	Cant.	20\newline
Pr 385	CantTr 	2	{ 51 74 }	Tarifa	1250	Cant.	25\newline
Pr 386	CantTr 	2	{ 51 74 }	Tarifa	1300	Cant.	25\newline
Pr 387	CantTr 	2	{ 51 74 }	Tarifa	1360	Cant.	30\newline
Pr 388	CantTr 	2	{ 51 74 }	Tarifa	1400	Cant.	30\newline
Pr 389	CantTr 	2	{ 51 80 }	Tarifa	1250	Cant.	30\newline
Pr 390	CantTr 	2	{ 51 80 }	Tarifa	1300	Cant.	20\newline
Pr 391	CantTr 	2	{ 51 80 }	Tarifa	1360	Cant.	21\newline
Pr 392	CantTr 	2	{ 51 80 }	Tarifa	1400	Cant.	23\newline
Pr 393	CantTr 	2	{ 51 75 }	Tarifa	1250	Cant.	20\newline
Pr 394	CantTr 	2	{ 51 75 }	Tarifa	1300	Cant.	25\newline
Pr 395	CantTr 	2	{ 51 75 }	Tarifa	1360	Cant.	25\newline
Pr 396	CantTr 	2	{ 51 75 }	Tarifa	1400	Cant.	30\newline
\newline
End Productos\newline
\newline
\end{scriptsize}


\begin{center}
	\begin{tabular}{|c|c|}
	\hline
	Cantidad de tramos & 83 \\
	\hline
	Cantidad de productos & 396 \\
	\hline
	\end{tabular}
\end{center}
\newpage
%------------------------------------------------------------------------------------%

\subsection{Archivo de prueba 6}

\begin{scriptsize}
Red Aerolineas con 104 tramos y 519 productos.\newline
\newline
Init Tramos\newline
\newline
Tr 1	Cant.	200	BandaH. M\newline 
Tr 2	Cant.	150	BandaH. M\newline
Tr 3	Cant.	180	BandaH. M\newline
Tr 4	Cant.	200	BandaH. M\newline
Tr 5	Cant.	200	BandaH. M\newline
Tr 6	Cant.	120	BandaH. M\newline
Tr 7	Cant.	160	BandaH. M\newline
Tr 8	Cant.	220	BandaH. M\newline
Tr 9	Cant.	180	BandaH. M\newline
Tr 10	Cant.	180	BandaH. M\newline
Tr 11	Cant.	200	BandaH. M\newline
Tr 12	Cant.	100	BandaH. M\newline
Tr 13	Cant.	150	BandaH. M\newline
Tr 14	Cant.	150	BandaH. M\newline
Tr 15	Cant.	150	BandaH. M\newline
Tr 16	Cant.	180	BandaH. M\newline
Tr 17	Cant.	160	BandaH. M\newline
Tr 18	Cant.	180	BandaH. M\newline
Tr 19	Cant.	200	BandaH. M\newline
Tr 20	Cant.	150	BandaH. M\newline
Tr 21	Cant.	200	BandaH. M\newline
Tr 22	Cant.	180	BandaH. M\newline
Tr 23	Cant.	160	BandaH. M\newline
Tr 24	Cant.	130	BandaH. M\newline
Tr 25	Cant.	150	BandaH. M\newline
Tr 26	Cant.	180	BandaH. M\newline
Tr 27	Cant.	180	BandaH. M\newline
Tr 28	Cant.	190	BandaH. M\newline
Tr 29	Cant.	200	BandaH. M\newline
Tr 30	Cant.	200	BandaH. M\newline
Tr 31	Cant.	200	BandaH. M\newline
Tr 32	Cant.	150	BandaH. M\newline
Tr 33	Cant.	200	BandaH. M\newline
Tr 34	Cant.	180	BandaH. M\newline
Tr 35	Cant.	200	BandaH. M\newline
Tr 36	Cant.	180	BandaH. M\newline
Tr 37	Cant.	180	BandaH. M\newline
Tr 38	Cant.	180	BandaH. M\newline
Tr 39	Cant.	150	BandaH. M\newline
Tr 40	Cant.	200	BandaH. M\newline
Tr 41	Cant.	150	BandaH. M\newline
Tr 42	Cant.	200	BandaH. M\newline
Tr 43	Cant.	200	BandaH. M\newline
Tr 44	Cant.	180	BandaH. M\newline
Tr 45	Cant.	190	BandaH. M\newline
Tr 46	Cant.	191	BandaH. M\newline
Tr 47	Cant.	200	BandaH. M\newline
Tr 48	Cant.	200	BandaH. M\newline
Tr 49	Cant.	200	BandaH. M\newline
Tr 50	Cant.	180	BandaH. M\newline
Tr 51	Cant.	150	BandaH. M\newline
Tr 52	Cant.	120	BandaH. M\newline
Tr 53	Cant.	130	BandaH. M\newline
Tr 54	Cant.	180	BandaH. M\newline
Tr 55	Cant.	190	BandaH. M\newline
Tr 56	Cant.	200	BandaH. M\newline
Tr 57	Cant.	160	BandaH. M\newline
Tr 58	Cant.	200	BandaH. M\newline
Tr 59	Cant.	180	BandaH. M\newline
Tr 60	Cant.	170	BandaH. M\newline
Tr 61	Cant.	200	BandaH. M\newline
Tr 62	Cant.	100	BandaH. M\newline
Tr 63	Cant.	120	BandaH. M\newline
Tr 64	Cant.	150	BandaH. M\newline
Tr 65	Cant.	180	BandaH. M\newline
Tr 66	Cant.	200	BandaH. M\newline
Tr 67	Cant.	180	BandaH. M\newline
Tr 68	Cant.	180	BandaH. M\newline
Tr 69	Cant.	200	BandaH. M\newline
Tr 70	Cant.	150	BandaH. M\newline
Tr 71	Cant.	180	BandaH. M\newline
Tr 72	Cant.	150	BandaH. M\newline
Tr 73	Cant.	180	BandaH. M\newline
Tr 74	Cant.	200	BandaH. M\newline
Tr 75	Cant.	180	BandaH. M\newline
Tr 76	Cant.	180	BandaH. M\newline
Tr 77	Cant.	200	BandaH. M\newline
Tr 78	Cant.	150	BandaH. M\newline
Tr 79	Cant.	180	BandaH. M\newline
Tr 80	Cant.	150	BandaH. M\newline
Tr 81	Cant.	180	BandaH. M\newline
Tr 82	Cant.	200	BandaH. M\newline
Tr 83	Cant.	180	BandaH. M\newline
Tr 84	Cant.	200	BandaH. M\newline
Tr 85	Cant.	180	BandaH. M\newline
Tr 86	Cant.	180	BandaH. M\newline
Tr 87	Cant.	200	BandaH. M\newline
Tr 88	Cant.	150	BandaH. M\newline
Tr 89	Cant.	180	BandaH. M\newline
Tr 90	Cant.	150	BandaH. M\newline
Tr 91	Cant.	180	BandaH. M\newline
Tr 92	Cant.	200	BandaH. M\newline
Tr 93	Cant.	180	BandaH. M\newline
Tr 94	Cant.	200	BandaH. M\newline
Tr 95	Cant.	180	BandaH. M\newline
Tr 96	Cant.	180	BandaH. M\newline
Tr 97	Cant.	200	BandaH. M\newline
Tr 98	Cant.	150	BandaH. M\newline
Tr 99	Cant.	180	BandaH. M\newline
Tr 100	Cant.	150	BandaH. M\newline
Tr 101	Cant.	180	BandaH. M\newline
Tr 102	Cant.	200	BandaH. M\newline
Tr 103	Cant.	180	BandaH. M\newline
Tr 104	Cant.	181	BandaH. M\newline
\newline
End Tramos\newline
\newline
Init Productos\newline
\newline
Pr   1	CantTr  1 	{ 1 }	Tarifa	288	Cant.   20\newline
Pr   2	CantTr 	1	{ 1 }	Tarifa	360	Cant.	20\newline
Pr   3	CantTr 	1	{ 1 }	Tarifa	555	Cant.	50\newline
Pr   4	CantTr 	1	{ 2 }	Tarifa	308	Cant.	70\newline
Pr   5	CantTr 	1	{ 3 }	Tarifa	597	Cant.	15\newline
Pr   6	CantTr 	1	{ 3 }	Tarifa	950	Cant.	15\newline
Pr   7	CantTr 	1	{ 3 }	Tarifa	1145	Cant.	85\newline
Pr   8	CantTr 	2	{ 1 3 }	Tarifa	885	Cant.	21\newline
Pr   9	CantTr 	2	{ 1 3 }	Tarifa	1310	Cant.	28\newline
Pr  10	CantTr 	2	{ 1 3 }	Tarifa	1700	Cant.	45\newline
Pr  11	CantTr 	1	{ 4 }	Tarifa	864	Cant.	103\newline
Pr  12	CantTr 	2	{ 2 4 }	Tarifa	1172	Cant.	54\newline
Pr  13	CantTr 	1	{ 5 }	Tarifa	321	Cant.	20\newline
Pr  14	CantTr 	1	{ 5 }	Tarifa	356	Cant.	23\newline
Pr  15	CantTr 	1	{ 5 }	Tarifa	440	Cant.	35\newline
Pr  16	CantTr 	1	{ 5 }	Tarifa	678	Cant.	67\newline
Pr  17	CantTr 	1	{ 7 }	Tarifa	289	Cant.	20\newline
Pr  18	CantTr 	1	{ 7 }	Tarifa	556	Cant.	30\newline
Pr  19	CantTr 	1	{ 6 }	Tarifa	329	Cant.	20\newline
Pr  20	CantTr 	1	{ 6 }	Tarifa	410	Cant.	20\newline
Pr  21	CantTr 	1	{ 6 }	Tarifa	633	Cant.	20\newline
Pr  22	CantTr 	1	{ 8 }	Tarifa	941	Cant.	62\newline
Pr  23	CantTr 	2	{ 2 8 }	Tarifa	1249	Cant.	35\newline
Pr  24	CantTr 	1	{ 9 }	Tarifa	197	Cant.	90\newline
Pr  25	CantTr 	1	{ 10 }	Tarifa	303	Cant.	15\newline
Pr  26	CantTr 	1	{ 10 }	Tarifa	377	Cant.	19\newline
Pr  27	CantTr 	1	{ 10 }	Tarifa	581	Cant.	33\newline
Pr  28	CantTr 	1	{ 11 }	Tarifa	551	Cant.	10\newline
Pr  29	CantTr 	1	{ 11 }	Tarifa	857	Cant.	15\newline
Pr  30	CantTr 	1	{ 11 }	Tarifa	1061	Cant.	68\newline
Pr  31	CantTr 	1	{ 12 }	Tarifa	358	Cant.	30\newline
Pr  32	CantTr 	1	{ 12 }	Tarifa	685	Cant.	40\newline
Pr  33	CantTr 	1	{ 13 }	Tarifa	859	Cant.	85\newline
Pr  34	CantTr 	1	{ 14 }	Tarifa	937	Cant.	10\newline
Pr  35	CantTr 	1	{ 14 }	Tarifa	985	Cant.	12\newline
Pr  36	CantTr 	1	{ 14 }	Tarifa	1555	Cant.	21\newline
Pr  37	CantTr 	1	{ 14 }	Tarifa	1888	Cant.	58\newline
Pr  38	CantTr 	1	{ 14 }	Tarifa	1437	Cant.	35\newline
Pr  39	CantTr 	2	{ 13 14 }	Tarifa	1769	Cant.	20\newline
Pr  40	CantTr 	2	{ 13 14 }	Tarifa	1844	Cant.	30\newline
Pr  41	CantTr 	2	{ 13 14 }	Tarifa	2414	Cant.	40\newline
Pr  42	CantTr 	2	{ 13 14 }	Tarifa	2747	Cant.	50\newline
Pr  43	CantTr 	2	{ 13 14 }	Tarifa	2296	Cant.	35\newline
Pr  44	CantTr 	1	{ 15 }	Tarifa	492	Cant.	20\newline
Pr  45	CantTr 	1	{ 15 }	Tarifa	944	Cant.	36\newline
Pr  46	CantTr 	1	{ 16 }	Tarifa	397	Cant.	30\newline
Pr  47	CantTr 	1	{ 16 }	Tarifa	493	Cant.	32\newline
Pr  48	CantTr 	1	{ 16 }	Tarifa	760	Cant.	33\newline
Pr  49	CantTr 	1	{ 17 }	Tarifa	353	Cant.	20\newline
Pr  50	CantTr 	1	{ 17 }	Tarifa	438	Cant.	40\newline
Pr  51	CantTr 	1	{ 17 }	Tarifa	675	Cant.	60\newline
Pr  52	CantTr 	1	{ 18 }	Tarifa	72	Cant.	10\newline
Pr  53	CantTr 	1	{ 18 }	Tarifa	89	Cant.	20\newline
Pr  54	CantTr 	1	{ 18 }	Tarifa	135	Cant.	30\newline
Pr  55	CantTr 	2	{ 17 18 }	Tarifa	469	Cant.	15\newline
Pr  56	CantTr 	2	{ 17 18 }	Tarifa	527	Cant.	26\newline
Pr  57	CantTr 	2	{ 17 18 }	Tarifa	810	Cant.	34\newline
Pr  58	CantTr 	1	{ 19 }	Tarifa	443	Cant.	20\newline
Pr  59	CantTr 	1	{ 19 }	Tarifa	480	Cant.	21\newline
Pr  60	CantTr 	1	{ 19 }	Tarifa	852	Cant.	21\newline
Pr  61	CantTr 	1	{ 20 }	Tarifa	392	Cant.	13\newline
Pr  62	CantTr 	1	{ 20 }	Tarifa	433	Cant.	15\newline
Pr  63	CantTr 	1	{ 20 }	Tarifa	538	Cant.	17\newline
Pr  64	CantTr 	1	{ 20 }	Tarifa	830	Cant.	19\newline
Pr  65	CantTr 	1	{ 21 }	Tarifa	342	Cant.	20\newline
Pr  66	CantTr 	1	{ 21 }	Tarifa	655	Cant.	29\newline
Pr  67	CantTr 	1	{ 22 }	Tarifa	252	Cant.	17\newline
Pr  68	CantTr 	1	{ 22 }	Tarifa	278	Cant.	24\newline
Pr  69	CantTr 	1	{ 22 }	Tarifa	344	Cant.	30\newline
Pr  70	CantTr 	1	{ 22 }	Tarifa	530	Cant.	45\newline
Pr  71	CantTr 	1	{ 23 }	Tarifa	675	Cant.	20\newline
Pr  72	CantTr 	1	{ 23 }	Tarifa	1023	Cant.	30\newline
Pr  73	CantTr 	1	{ 23 }	Tarifa	1283	Cant.	30\newline
Pr  74	CantTr 	2	{ 22 23 }	Tarifa	927	Cant.	15\newline
Pr  75	CantTr 	2	{ 22 23 }	Tarifa	1301	Cant.	19\newline
Pr  76	CantTr 	2	{ 22 23 }	Tarifa	1627	Cant.	23\newline
Pr  77	CantTr 	2	{ 22 23 }	Tarifa	1813	Cant.	25\newline
Pr  78	CantTr 	1	{ 24 }	Tarifa	603	Cant.	15\newline
Pr  79	CantTr 	1	{ 24 }	Tarifa	629	Cant.	16\newline
Pr  80	CantTr 	1	{ 24 }	Tarifa	881	Cant.	28\newline
Pr  81	CantTr 	2	{ 22 24 }	Tarifa	855	Cant.	20\newline
Pr  82	CantTr 	2	{ 22 24 }	Tarifa	907	Cant.	20\newline
Pr  83	CantTr 	2	{ 22 24 }	Tarifa	1225	Cant.	40\newline
Pr  84	CantTr 	1	{ 25 }	Tarifa	276	Cant.	20\newline
Pr  85	CantTr 	1	{ 25 }	Tarifa	358	Cant.	20\newline
Pr  86	CantTr 	1	{ 25 }	Tarifa	444	Cant.	20\newline
Pr  87	CantTr 	1	{ 25 }	Tarifa	685	Cant.	35\newline
Pr  88	CantTr 	1	{ 26 }	Tarifa	316	Cant.	20\newline
Pr  89	CantTr 	1	{ 26 }	Tarifa	445	Cant.	20\newline
Pr  90	CantTr 	1	{ 26 }	Tarifa	685	Cant.	46\newline
Pr  91	CantTr 	1	{ 27 }	Tarifa	353	Cant.	20\newline
Pr  92	CantTr 	1	{ 27 }	Tarifa	445	Cant.	20\newline
Pr  93	CantTr 	1	{ 27 }	Tarifa	685	Cant.	20\newline
Pr  94	CantTr 	1	{ 28 }	Tarifa	316	Cant.	17\newline
Pr  95	CantTr 	1	{ 28 }	Tarifa	445	Cant.	19\newline
Pr  96	CantTr 	1	{ 28 }	Tarifa	685	Cant.	20\newline
Pr  97	CantTr 	2	{ 28 27 }	Tarifa	714	Cant.	20\newline
Pr  98	CantTr 	2	{ 28 27 }	Tarifa	890	Cant.	21\newline
Pr  99	CantTr 	2	{ 28 27 }	Tarifa	1370	Cant.	34\newline
Pr 100	CantTr 	1	{ 29 }	Tarifa	238	Cant.	20\newline
Pr 101	CantTr 	1	{ 29 }	Tarifa	295	Cant.	20\newline
Pr 102	CantTr 	1	{ 29 }	Tarifa	454	Cant.	20\newline
Pr 103	CantTr 	1	{ 30 }	Tarifa	255	Cant.	20\newline
Pr 104	CantTr 	1	{ 30 }	Tarifa	317	Cant.	20\newline
Pr 105	CantTr 	1	{ 30 }	Tarifa	488	Cant.	30\newline
Pr 106	CantTr 	1	{ 31 }	Tarifa	494	Cant.	20\newline
Pr 107	CantTr 	1	{ 31 }	Tarifa	612	Cant.	20\newline
Pr 108	CantTr 	1	{ 31 }	Tarifa	942	Cant.	30\newline
Pr 109	CantTr 	1	{ 32 }	Tarifa	486	Cant.	20\newline
Pr 110	CantTr 	1	{ 32 }	Tarifa	775	Cant.	22\newline
Pr 111	CantTr 	1	{ 32 }	Tarifa	934	Cant.	32\newline
Pr 112	CantTr 	2	{ 30 32 }	Tarifa	741	Cant.	30\newline
Pr 113	CantTr 	2	{ 30 32 }	Tarifa	1092	Cant.	30\newline
Pr 114	CantTr 	2	{ 30 32 }	Tarifa	1422	Cant.	60\newline
Pr 115	CantTr 	1	{ 33 }	Tarifa	526	Cant.	15\newline
Pr 116	CantTr 	1	{ 33 }	Tarifa	1019	Cant.	19\newline
Pr 117	CantTr 	1	{ 34 }	Tarifa	944	Cant.	75\newline
Pr 118	CantTr 	1	{ 35 }	Tarifa	612	Cant.	21\newline
Pr 119	CantTr 	1	{ 35 }	Tarifa	619	Cant.	22\newline
Pr 120	CantTr 	1	{ 35 }	Tarifa	678	Cant.	23\newline
Pr 121	CantTr 	1	{ 35 }	Tarifa	875	Cant.	32\newline
Pr 122	CantTr 	1	{ 35 }	Tarifa	1329	Cant.	46\newline
Pr 123	CantTr 	1	{ 35 }	Tarifa	1636	Cant.	55\newline
Pr 124	CantTr 	1	{ 35 }	Tarifa	3095	Cant.	66\newline
Pr 125	CantTr 	1	{ 36 }	Tarifa	333	Cant.	20\newline
Pr 126	CantTr 	1	{ 36 }	Tarifa	457	Cant.	20\newline
Pr 127	CantTr 	1	{ 36 }	Tarifa	704	Cant.	20\newline
Pr 128	CantTr 	2	{ 36 35 }	Tarifa	945	Cant.	10\newline
Pr 129	CantTr 	2	{ 36 35 }	Tarifa	952	Cant.	21\newline
Pr 130	CantTr 	2	{ 36 35 }	Tarifa	1011	Cant.	25\newline
Pr 131	CantTr 	2	{ 36 35 }	Tarifa	1332	Cant.	36\newline
Pr 132	CantTr 	2	{ 36 35 }	Tarifa	1786	Cant.	41\newline
Pr 133	CantTr 	2	{ 36 35 }	Tarifa	2340	Cant.	42\newline
Pr 134	CantTr 	2	{ 36 35 }	Tarifa	3789	Cant.	45\newline
Pr 135	CantTr 	1	{ 37 }	Tarifa	704	Cant.	67\newline
Pr 136	CantTr 	1	{ 38 }	Tarifa	653	Cant.	78\newline
Pr 137	CantTr 	2	{ 36 38 }	Tarifa	986	Cant.	30\newline
Pr 138	CantTr 	2	{ 36 38 }	Tarifa	1110	Cant.	35\newline
Pr 139	CantTr 	2	{ 36 38 }	Tarifa	1357	Cant.	35\newline
Pr 140	CantTr 	1	{ 39 }	Tarifa	642	Cant.	20\newline
Pr 141	CantTr 	1	{ 39 }	Tarifa	691	Cant.	20\newline
Pr 142	CantTr 	1	{ 39 }	Tarifa	1335	Cant.	20\newline
Pr 143	CantTr 	1	{ 39 }	Tarifa	1649	Cant.	30\newline
Pr 144	CantTr 	1	{ 39 }	Tarifa	1990	Cant.	40\newline
Pr 145	CantTr 	1	{ 39 }	Tarifa	3101	Cant.	50\newline
Pr 146	CantTr 	1	{ 40 }	Tarifa	324	Cant.	20\newline
Pr 147	CantTr 	1	{ 40 }	Tarifa	358	Cant.	20\newline
Pr 148	CantTr 	1	{ 40 }	Tarifa	514	Cant.	20\newline
Pr 149	CantTr 	1	{ 40 }	Tarifa	533	Cant.	20\newline
Pr 150	CantTr 	1	{ 40 }	Tarifa	944	Cant.	33\newline
Pr 151	CantTr 	1	{ 40 }	Tarifa	1285	Cant.	38\newline
Pr 152	CantTr 	1	{ 41 }	Tarifa	675	Cant.	43\newline
Pr 153	CantTr 	1	{ 41 }	Tarifa	710	Cant.	50\newline
Pr 154	CantTr 	1	{ 41 }	Tarifa	1798	Cant.	58\newline
Pr 155	CantTr 	1	{ 42 }	Tarifa	352	Cant.	60\newline
Pr 156	CantTr 	1	{ 43 }	Tarifa	986	Cant.	60\newline
Pr 157	CantTr 	1	{ 44 }	Tarifa	155	Cant.	70\newline
Pr 158	CantTr 	1	{ 44 }	Tarifa	300	Cant.	20\newline
Pr 159	CantTr 	1	{ 45 }	Tarifa	268	Cant.	30\newline
Pr 160	CantTr 	1	{ 45 }	Tarifa	519	Cant.	35\newline
Pr 161	CantTr 	2	{ 40 44 }	Tarifa	479	Cant.	20\newline
Pr 162	CantTr 	2	{ 40 44 }	Tarifa	513	Cant.	20\newline
Pr 163	CantTr 	2	{ 40 44 }	Tarifa	668	Cant.	30\newline
Pr 164	CantTr 	2	{ 40 44 }	Tarifa	688	Cant.	30\newline
Pr 165	CantTr 	2	{ 40 44 }	Tarifa	1244	Cant.	40\newline
Pr 166	CantTr 	2	{ 40 44 }	Tarifa	1555	Cant.	40\newline
Pr 167	CantTr 	1	{ 46 }	Tarifa	150	Cant.	20\newline
Pr 168	CantTr 	1	{ 46 }	Tarifa	200	Cant.	20\newline
Pr 169	CantTr 	1	{ 46 }	Tarifa	300	Cant.	20\newline
Pr 170	CantTr 	1	{ 46 }	Tarifa	350	Cant.	20\newline
Pr 171	CantTr 	2	{ 22 46 }	Tarifa	402	Cant.	30\newline
Pr 172	CantTr 	2	{ 22 46 }	Tarifa	478	Cant.	30\newline
Pr 173	CantTr 	2	{ 22 46 }	Tarifa	644	Cant.	35\newline
Pr 174	CantTr 	2	{ 22 46 }	Tarifa	880	Cant.	45\newline
Pr 175	CantTr 	3	{ 22 46 33 }	Tarifa	928	Cant.	20\newline
Pr 176	CantTr 	3	{ 22 46 33 }	Tarifa	1004	Cant.	20\newline
Pr 177	CantTr 	3	{ 22 46 33 }	Tarifa	1663	Cant.	30\newline
Pr 178	CantTr 	3	{ 22 46 33 }	Tarifa	1899	Cant.	40\newline
Pr 179	CantTr 	4	{ 22 46 33 45 }	Tarifa	1196	Cant.	15\newline
Pr 180	CantTr 	4	{ 22 46 33 45 }	Tarifa	1272	Cant.	15\newline
Pr 181	CantTr 	4	{ 22 46 33 45 }	Tarifa	1523	Cant.	15\newline
Pr 182	CantTr 	4	{ 22 46 33 45 }	Tarifa	2182	Cant.	30\newline
Pr 183	CantTr 	4	{ 22 46 33 45 }	Tarifa	2418	Cant.	30\newline
Pr 184	CantTr 	1	{ 47 }	Tarifa	450	Cant.	10\newline
Pr 185	CantTr 	1	{ 47 }	Tarifa	550	Cant.	10\newline
Pr 186	CantTr 	1	{ 47 }	Tarifa	650	Cant.	10\newline
Pr 187	CantTr 	1	{ 47 }	Tarifa	750	Cant.	50\newline
Pr 188	CantTr 	1	{ 47 }	Tarifa	850	Cant.	50\newline
Pr 189	CantTr 	1	{ 48 }	Tarifa	300	Cant.	10\newline
Pr 190	CantTr 	1	{ 48 }	Tarifa	380	Cant.	11\newline
Pr 191	CantTr 	1	{ 48 }	Tarifa	500	Cant.	42\newline
Pr 192	CantTr 	1	{ 49 }	Tarifa	200	Cant.	10\newline
Pr 193	CantTr 	1	{ 49 }	Tarifa	230	Cant.	15\newline
Pr 194	CantTr 	1	{ 49 }	Tarifa	400	Cant.	20\newline
Pr 195	CantTr 	1	{ 49 }	Tarifa	560	Cant.	35\newline
Pr 196	CantTr 	1	{ 49 }	Tarifa	710	Cant.	45\newline
Pr 197	CantTr 	5	{ 46 47 1 48 49 }	Tarifa	2000	Cant.	20\newline
Pr 198	CantTr 	5	{ 46 47 1 48 49 }	Tarifa	2200	Cant.	20\newline
Pr 199	CantTr 	5	{ 46 47 1 48 49 }	Tarifa	2530	Cant.	20\newline
Pr 200	CantTr 	5	{ 46 47 1 48 49 }	Tarifa	2680	Cant.	30\newline
Pr 201	CantTr 	5	{ 46 47 1 48 49 }	Tarifa	2900	Cant.	30\newline
Pr 202	CantTr 	1	{ 50 }	Tarifa	230	Cant.	10\newline
Pr 203	CantTr 	1	{ 50 }	Tarifa	330	Cant.	10\newline
Pr 204	CantTr 	1	{ 50 }	Tarifa	430	Cant.	10\newline
Pr 205	CantTr 	1	{ 50 }	Tarifa	530	Cant.	20\newline
Pr 206	CantTr 	1	{ 50 }	Tarifa	630	Cant.	20\newline
Pr 207	CantTr 	1	{ 50 }	Tarifa	730	Cant.	20\newline
Pr 208	CantTr 	6	{ 46 47 1 48 49 50 }	Tarifa	2500	Cant.	20\newline
Pr 209	CantTr 	6	{ 46 47 1 48 49 50 }	Tarifa	2600	Cant.	20\newline
Pr 210	CantTr 	6	{ 46 47 1 48 49 50 }	Tarifa	2700	Cant.	20\newline
Pr 211	CantTr 	6	{ 46 47 1 48 49 50 }	Tarifa	2890	Cant.	20\newline
Pr 212	CantTr 	6	{ 46 47 1 48 49 50 }	Tarifa	3000	Cant.	35\newline
Pr 213	CantTr 	6	{ 46 47 1 48 49 50 }	Tarifa	3150	Cant.	35\newline
Pr 214	CantTr 	3	{ 48 49 50 }	Tarifa	600	Cant.	15\newline
Pr 215	CantTr 	3	{ 48 49 50 }	Tarifa	700	Cant.	15\newline
Pr 216	CantTr 	3	{ 48 49 50 }	Tarifa	750	Cant.	35\newline
Pr 217	CantTr 	3	{ 48 49 50 }	Tarifa	900	Cant.	35\newline
Pr 218	CantTr 	4	{ 1 48 49 50 }	Tarifa	850	Cant.	20\newline
Pr 219	CantTr 	4	{ 1 48 49 50 }	Tarifa	950	Cant.	21\newline
Pr 220	CantTr 	4	{ 1 48 49 50 }	Tarifa	1150	Cant.	25\newline
Pr 221	CantTr 	4	{ 1 48 49 50 }	Tarifa	1250	Cant.	30\newline
Pr 222	CantTr 	4	{ 1 48 49 50 }	Tarifa	1350	Cant.	33\newline
Pr 223	CantTr 	4	{ 1 48 49 50 }	Tarifa	1450	Cant.	40\newline
Pr 224	CantTr 	3	{ 47 1 48 }	Tarifa	580	Cant.	20\newline
Pr 225	CantTr 	3	{ 47 1 48 }	Tarifa	590	Cant.	20\newline
Pr 226	CantTr 	3	{ 47 1 48 }	Tarifa	600	Cant.	20\newline
Pr 227	CantTr 	3	{ 47 1 48 }	Tarifa	610	Cant.	25\newline
Pr 228	CantTr 	3	{ 47 1 48 }	Tarifa	650	Cant.	25\newline
Pr 229	CantTr 	3	{ 47 1 48 }	Tarifa	700	Cant.	30\newline
Pr 230	CantTr 	3	{ 47 1 48 }	Tarifa	750	Cant.	30\newline
Pr 231	CantTr 	3	{ 47 1 48 }	Tarifa	800	Cant.	30\newline
Pr 232	CantTr 	2	{ 49 50 }	Tarifa	430	Cant.	20\newline
Pr 233	CantTr 	2	{ 49 50 }	Tarifa	500	Cant.	21\newline
Pr 234	CantTr 	2	{ 49 50 }	Tarifa	560	Cant.	23\newline
Pr 235	CantTr 	2	{ 49 50 }	Tarifa	590	Cant.	26\newline
Pr 236	CantTr 	2	{ 49 50 }	Tarifa	610	Cant.	28\newline
Pr 237	CantTr 	4	{ 46 47 1 48 }	Tarifa	1500	Cant.	20\newline
Pr 238	CantTr 	4	{ 46 47 1 48 }	Tarifa	1600	Cant.	20\newline
Pr 239	CantTr 	4	{ 46 47 1 48 }	Tarifa	1700	Cant.	20\newline
Pr 240	CantTr 	4	{ 46 47 1 48 }	Tarifa	1830	Cant.	30\newline
Pr 241	CantTr 	4	{ 46 47 1 48 }	Tarifa	1850	Cant.	35\newline
Pr 242	CantTr 	1	{ 51 }	Tarifa	500	Cant.	35\newline
Pr 243	CantTr 	1	{ 51 }	Tarifa	550	Cant.	35\newline
Pr 244	CantTr 	1	{ 51 }	Tarifa	560	Cant.	15\newline
Pr 245	CantTr 	1	{ 51 }	Tarifa	570	Cant.	15\newline
Pr 246	CantTr 	1	{ 52 }	Tarifa	400	Cant.	35\newline
Pr 247	CantTr 	1	{ 52 }	Tarifa	420	Cant.	35\newline
Pr 248	CantTr 	1	{ 52 }	Tarifa	430	Cant.	20\newline
Pr 249	CantTr 	1	{ 52 }	Tarifa	460	Cant.	21\newline
Pr 250	CantTr 	1	{ 53 }	Tarifa	610	Cant.	25\newline
Pr 251	CantTr 	1	{ 53 }	Tarifa	650	Cant.	30\newline
Pr 252	CantTr 	1	{ 53 }	Tarifa	660	Cant.	33\newline
Pr 253	CantTr 	1	{ 53 }	Tarifa	680	Cant.	40\newline
Pr 254	CantTr 	1	{ 54 }	Tarifa	1200	Cant.	20\newline
Pr 255	CantTr 	1	{ 54 }	Tarifa	1300	Cant.	20\newline
Pr 256	CantTr 	1	{ 54 }	Tarifa	1560	Cant.	20\newline
Pr 257	CantTr 	1	{ 54 }	Tarifa	1800	Cant.	25\newline
Pr 258	CantTr 	1	{ 55 }	Tarifa	700	Cant.	25\newline
Pr 259	CantTr 	1	{ 55 }	Tarifa	750	Cant.	30\newline
Pr 260	CantTr 	1	{ 55 }	Tarifa	780	Cant.	30\newline
Pr 261	CantTr 	1	{ 55 }	Tarifa	790	Cant.	30\newline
Pr 262	CantTr 	1	{ 55 }	Tarifa	800	Cant.	20\newline
Pr 263	CantTr 	1	{ 56 }	Tarifa	822	Cant.	21\newline
Pr 264	CantTr 	1	{ 56 }	Tarifa	830	Cant.	23\newline
Pr 265	CantTr 	1	{ 56 }	Tarifa	860	Cant.	26\newline
Pr 266	CantTr 	1	{ 56 }	Tarifa	880	Cant.	28\newline
Pr 267	CantTr 	1	{ 56 }	Tarifa	890	Cant.	20\newline
Pr 268	CantTr 	1	{ 57 }	Tarifa	350	Cant.	20\newline
Pr 269	CantTr 	1	{ 57 }	Tarifa	360	Cant.	20\newline
Pr 270	CantTr 	1	{ 57 }	Tarifa	380	Cant.	30\newline
Pr 271	CantTr 	1	{ 57 }	Tarifa	400	Cant.	35\newline
Pr 272	CantTr 	1	{ 58 }	Tarifa	900	Cant.	35\newline
Pr 273	CantTr 	1	{ 58 }	Tarifa	930	Cant.	35\newline
Pr 274	CantTr 	1	{ 58 }	Tarifa	970	Cant.	15\newline
Pr 275	CantTr 	1	{ 58 }	Tarifa	990	Cant.	15\newline
Pr 276	CantTr 	1	{ 59 }	Tarifa	300	Cant.	35\newline
Pr 277	CantTr 	1	{ 59 }	Tarifa	330	Cant.	35\newline
Pr 278	CantTr 	1	{ 59 }	Tarifa	336	Cant.	20\newline
Pr 279	CantTr 	1	{ 59 }	Tarifa	355	Cant.	21\newline
Pr 280	CantTr 	1	{ 60 }	Tarifa	200	Cant.	25\newline
Pr 281	CantTr 	1	{ 60 }	Tarifa	250	Cant.	30\newline
Pr 282	CantTr 	1	{ 60 }	Tarifa	280	Cant.	33\newline
Pr 283	CantTr 	1	{ 60 }	Tarifa	290	Cant.	40\newline
Pr 284	CantTr 	1	{ 61 }	Tarifa	310	Cant.	20\newline
Pr 285	CantTr 	1	{ 61 }	Tarifa	350	Cant.	20\newline
Pr 286	CantTr 	1	{ 61 }	Tarifa	370	Cant.	20\newline
Pr 287	CantTr 	1	{ 61 }	Tarifa	380	Cant.	25\newline
Pr 288	CantTr 	1	{ 62 }	Tarifa	660	Cant.	25\newline
Pr 289	CantTr 	1	{ 62 }	Tarifa	670	Cant.	30\newline
Pr 290	CantTr 	1	{ 62 }	Tarifa	680	Cant.	30\newline
Pr 291	CantTr 	1	{ 62 }	Tarifa	690	Cant.	30\newline
Pr 292	CantTr 	1	{ 63 }	Tarifa	780	Cant.	20\newline
Pr 293	CantTr 	1	{ 63 }	Tarifa	790	Cant.	21\newline
Pr 294	CantTr 	1	{ 63 }	Tarifa	800	Cant.	23\newline
Pr 295	CantTr 	1	{ 63 }	Tarifa	850	Cant.	26\newline
Pr 296	CantTr 	1	{ 64 }	Tarifa	700	Cant.	28\newline
Pr 297	CantTr 	1	{ 64 }	Tarifa	750	Cant.	20\newline
Pr 298	CantTr 	1	{ 64 }	Tarifa	770	Cant.	20\newline
Pr 299	CantTr 	1	{ 64 }	Tarifa	780	Cant.	20\newline
Pr 300	CantTr 	1	{ 65 }	Tarifa	1100	Cant.	30\newline
Pr 301	CantTr 	1	{ 65 }	Tarifa	1150	Cant.	35\newline
Pr 302	CantTr 	1	{ 65 }	Tarifa	1350	Cant.	35\newline
Pr 303	CantTr 	1	{ 65 }	Tarifa	1450	Cant.	35\newline
Pr 304	CantTr 	1	{ 66 }	Tarifa	300	Cant.	15\newline
Pr 305	CantTr 	1	{ 66 }	Tarifa	350	Cant.	15\newline
Pr 306	CantTr 	1	{ 66 }	Tarifa	400	Cant.	35\newline
Pr 307	CantTr 	1	{ 66 }	Tarifa	450	Cant.	35\newline
Pr 308	CantTr 	1	{ 67 }	Tarifa	930	Cant.	20\newline
Pr 309	CantTr 	1	{ 67 }	Tarifa	940	Cant.	21\newline
Pr 310	CantTr 	1	{ 67 }	Tarifa	950	Cant.	25\newline
Pr 311	CantTr 	1	{ 67 }	Tarifa	960	Cant.	30\newline
Pr 312	CantTr 	1	{ 68 }	Tarifa	450	Cant.	33\newline
Pr 313	CantTr 	1	{ 68 }	Tarifa	460	Cant.	40\newline
Pr 314	CantTr 	1	{ 68 }	Tarifa	480	Cant.	20\newline
Pr 315	CantTr 	1	{ 68 }	Tarifa	490	Cant.	20\newline
Pr 316	CantTr 	1	{ 69 }	Tarifa	500	Cant.	20\newline
Pr 317	CantTr 	1	{ 69 }	Tarifa	510	Cant.	25\newline
Pr 318	CantTr 	1	{ 69 }	Tarifa	580	Cant.	25\newline
Pr 319	CantTr 	1	{ 69 }	Tarifa	590	Cant.	30\newline
Pr 320	CantTr 	1	{ 70 }	Tarifa	340	Cant.	30\newline
Pr 321	CantTr 	1	{ 70 }	Tarifa	440	Cant.	30\newline
Pr 322	CantTr 	1	{ 70 }	Tarifa	540	Cant.	20\newline
Pr 323	CantTr 	1	{ 70 }	Tarifa	640	Cant.	21\newline
Pr 324	CantTr 	1	{ 71 }	Tarifa	1350	Cant.	23\newline
Pr 325	CantTr 	1	{ 71 }	Tarifa	1400	Cant.	26\newline
Pr 326	CantTr 	1	{ 71 }	Tarifa	1550	Cant.	28\newline
Pr 327	CantTr 	1	{ 71 }	Tarifa	1600	Cant.	20\newline
Pr 328	CantTr 	1	{ 72 }	Tarifa	800	Cant.	20\newline
Pr 329	CantTr 	1	{ 72 }	Tarifa	820	Cant.	20\newline
Pr 330	CantTr 	1	{ 72 }	Tarifa	830	Cant.	20\newline
Pr 331	CantTr 	1	{ 72 }	Tarifa	850	Cant.	30\newline
Pr 332	CantTr 	1	{ 73 }	Tarifa	750	Cant.	35\newline
Pr 333	CantTr 	1	{ 73 }	Tarifa	770	Cant.	35\newline
Pr 334	CantTr 	1	{ 73 }	Tarifa	780	Cant.	35\newline
Pr 335	CantTr 	1	{ 73 }	Tarifa	790	Cant.	15\newline
Pr 336	CantTr 	1	{ 74 }	Tarifa	750	Cant.	15\newline
Pr 337	CantTr 	1	{ 74 }	Tarifa	770	Cant.	35\newline
Pr 338	CantTr 	1	{ 74 }	Tarifa	780	Cant.	35\newline
Pr 339	CantTr 	1	{ 74 }	Tarifa	790	Cant.	20\newline
Pr 340	CantTr 	1	{ 75 }	Tarifa	750	Cant.	21\newline
Pr 341	CantTr 	1	{ 75 }	Tarifa	770	Cant.	25\newline
Pr 342	CantTr 	1	{ 75 }	Tarifa	780	Cant.	30\newline
Pr 343	CantTr 	1	{ 75 }	Tarifa	790	Cant.	33\newline
Pr 344	CantTr 	1	{ 76 }	Tarifa	750	Cant.	40\newline
Pr 345	CantTr 	1	{ 76 }	Tarifa	770	Cant.	20\newline
Pr 346	CantTr 	1	{ 76 }	Tarifa	780	Cant.	20\newline
Pr 347	CantTr 	1	{ 76 }	Tarifa	790	Cant.	20\newline
Pr 348	CantTr 	1	{ 77 }	Tarifa	500	Cant.	25\newline
Pr 349	CantTr 	1	{ 77 }	Tarifa	560	Cant.	25\newline
Pr 350	CantTr 	1	{ 77 }	Tarifa	590	Cant.	30\newline
Pr 351	CantTr 	1	{ 77 }	Tarifa	610	Cant.	30\newline
Pr 352	CantTr 	1	{ 78 }	Tarifa	800	Cant.	30\newline
Pr 353	CantTr 	1	{ 78 }	Tarifa	840	Cant.	20\newline
Pr 354	CantTr 	1	{ 78 }	Tarifa	860	Cant.	21\newline
Pr 355	CantTr 	1	{ 78 }	Tarifa	890	Cant.	23\newline
Pr 356	CantTr 	1	{ 79 }	Tarifa	910	Cant.	20\newline
Pr 357	CantTr 	1	{ 79 }	Tarifa	930	Cant.	20\newline
Pr 358	CantTr 	1	{ 79 }	Tarifa	950	Cant.	20\newline
Pr 359	CantTr 	1	{ 79 }	Tarifa	960	Cant.	30\newline
Pr 360	CantTr 	1	{ 80 }	Tarifa	750	Cant.	35\newline
Pr 361	CantTr 	1	{ 80 }	Tarifa	770	Cant.	35\newline
Pr 362	CantTr 	1	{ 80 }	Tarifa	780	Cant.	35\newline
Pr 363	CantTr 	1	{ 80 }	Tarifa	790	Cant.	15\newline
Pr 364	CantTr 	1	{ 81 }	Tarifa	1105	Cant.	15\newline
Pr 365	CantTr 	1	{ 81 }	Tarifa	1190	Cant.	35\newline
Pr 366	CantTr 	1	{ 81 }	Tarifa	1310	Cant.	35\newline
Pr 367	CantTr 	1	{ 81 }	Tarifa	1450	Cant.	20\newline
Pr 368	CantTr 	1	{ 82 }	Tarifa	700	Cant.	21\newline
Pr 369	CantTr 	1	{ 82 }	Tarifa	750	Cant.	25\newline
Pr 370	CantTr 	1	{ 82 }	Tarifa	760	Cant.	30\newline
Pr 371	CantTr 	1	{ 82 }	Tarifa	790	Cant.	33\newline
Pr 372	CantTr 	1	{ 83 }	Tarifa	600	Cant.	40\newline
Pr 373	CantTr 	1	{ 83 }	Tarifa	610	Cant.	20\newline
Pr 374	CantTr 	1	{ 83 }	Tarifa	630	Cant.	20\newline
Pr 375	CantTr 	1	{ 83 }	Tarifa	660	Cant.	20\newline
Pr 376	CantTr 	2	{ 51 72 }	Tarifa	1300	Cant.	25\newline
Pr 377	CantTr 	2	{ 51 72 }	Tarifa	1350	Cant.	25\newline
Pr 378	CantTr 	2	{ 51 72 }	Tarifa	1410	Cant.	30\newline
Pr 379	CantTr 	2	{ 51 72 }	Tarifa	1450	Cant.	30\newline
Pr 380	CantTr 	3	{ 51 72 9  }	Tarifa	1500	Cant.	30\newline
Pr 381	CantTr 	3	{ 51 72 9  }	Tarifa	1570	Cant.	20\newline
Pr 382	CantTr 	3	{ 51 72 9  }	Tarifa	1610	Cant.	21\newline
Pr 383	CantTr 	3	{ 51 72 9  }	Tarifa	1630	Cant.	23\newline
Pr 384	CantTr 	3	{ 51 72 9  }	Tarifa	1705	Cant.	20\newline
Pr 385	CantTr 	2	{ 51 74 }	Tarifa	1250	Cant.	25\newline
Pr 386	CantTr 	2	{ 51 74 }	Tarifa	1300	Cant.	25\newline
Pr 387	CantTr 	2	{ 51 74 }	Tarifa	1360	Cant.	30\newline
Pr 388	CantTr 	2	{ 51 74 }	Tarifa	1400	Cant.	30\newline
Pr 389	CantTr 	2	{ 51 80 }	Tarifa	1250	Cant.	30\newline
Pr 390	CantTr 	2	{ 51 80 }	Tarifa	1300	Cant.	20\newline
Pr 391	CantTr 	2	{ 51 80 }	Tarifa	1360	Cant.	21\newline
Pr 392	CantTr 	2	{ 51 80 }	Tarifa	1400	Cant.	23\newline
Pr 393	CantTr 	2	{ 51 75 }	Tarifa	1250	Cant.	20\newline
Pr 394	CantTr 	2	{ 51 75 }	Tarifa	1300	Cant.	25\newline
Pr 395	CantTr 	2	{ 51 75 }	Tarifa	1360	Cant.	25\newline
Pr 396	CantTr 	2	{ 51 75 }	Tarifa	1400	Cant.	30\newline
Pr 397	CantTr 	2	{ 51 79 }	Tarifa	1480	Cant.	15\newline
Pr 398	CantTr 	2	{ 51 79 }	Tarifa	1500	Cant.	15\newline
Pr 399	CantTr 	2	{ 51 79 }	Tarifa	1550	Cant.	85\newline
Pr 400	CantTr 	2	{ 51 79 }	Tarifa	1600	Cant.	21\newline
Pr 401	CantTr 	2	{ 22 51 }	Tarifa	850	Cant.	28\newline
Pr 402	CantTr 	2	{ 22 51 }	Tarifa	860	Cant.	45\newline
Pr 403	CantTr 	2	{ 22 51 }	Tarifa	870	Cant.	103\newline
Pr 404	CantTr 	2	{ 22 51 }	Tarifa	880	Cant.	54\newline
Pr 405	CantTr 	3	{ 22 51 79 }	Tarifa	1700	Cant.	20\newline
Pr 406	CantTr 	3	{ 22 51 79 }	Tarifa	1780	Cant.	23\newline
Pr 407	CantTr 	3	{ 22 51 79 }	Tarifa	1800	Cant.	35\newline
Pr 408	CantTr 	3	{ 22 51 79 }	Tarifa	1830	Cant.	67\newline
Pr 409	CantTr 	3	{ 22 51 79 }	Tarifa	1870	Cant.	20\newline
Pr 410	CantTr 	3	{ 22 51 79 }	Tarifa	1900	Cant.	30\newline
Pr 411	CantTr 	1	{ 84 }	Tarifa	300	Cant.	20\newline
Pr 412	CantTr 	1	{ 84 }	Tarifa	330	Cant.	20\newline
Pr 413	CantTr 	1	{ 84 }	Tarifa	350	Cant.	20\newline
Pr 414	CantTr 	1	{ 84 }	Tarifa	360	Cant.	62\newline
Pr 415	CantTr 	1	{ 85 }	Tarifa	500	Cant.	15\newline
Pr 416	CantTr 	1	{ 85 }	Tarifa	520	Cant.	15\newline
Pr 417	CantTr 	1	{ 85 }	Tarifa	550	Cant.	85\newline
Pr 418	CantTr 	1	{ 85 }	Tarifa	580	Cant.	21\newline
Pr 419	CantTr 	1	{ 86 }	Tarifa	1660	Cant.	28\newline
Pr 420	CantTr 	1	{ 86 }	Tarifa	1680	Cant.	45\newline
Pr 421	CantTr 	1	{ 86 }	Tarifa	1730	Cant.	103\newline
Pr 422	CantTr 	1	{ 86 }	Tarifa	1800	Cant.	54\newline
Pr 423	CantTr 	1	{ 87 }	Tarifa	1510	Cant.	20\newline
Pr 424	CantTr 	1	{ 87 }	Tarifa	1550	Cant.	23\newline
Pr 425	CantTr 	1	{ 87 }	Tarifa	1580	Cant.	35\newline
Pr 426	CantTr 	1	{ 87 }	Tarifa	1610	Cant.	67\newline
Pr 427	CantTr 	1	{ 88 }	Tarifa	700	Cant.	20\newline
Pr 428	CantTr 	1	{ 88 }	Tarifa	710	Cant.	30\newline
Pr 429	CantTr 	1	{ 88 }	Tarifa	730	Cant.	20\newline
Pr 430	CantTr 	1	{ 88 }	Tarifa	770	Cant.	20\newline
Pr 431	CantTr 	1	{ 89 }	Tarifa	820	Cant.	20\newline
Pr 432	CantTr 	1	{ 89 }	Tarifa	830	Cant.	62\newline
Pr 433	CantTr 	1	{ 89 }	Tarifa	850	Cant.	20\newline
Pr 434	CantTr 	1	{ 89 }	Tarifa	870	Cant.	20\newline
Pr 435	CantTr 	1	{ 90 }	Tarifa	700	Cant.	20\newline
Pr 436	CantTr 	1	{ 90 }	Tarifa	720	Cant.	25\newline
Pr 437	CantTr 	1	{ 90 }	Tarifa	730	Cant.	25\newline
Pr 438	CantTr 	1	{ 90 }	Tarifa	750	Cant.	30\newline
Pr 439	CantTr 	1	{ 91 }	Tarifa	520	Cant.	30\newline
Pr 440	CantTr 	1	{ 91 }	Tarifa	540	Cant.	30\newline
Pr 441	CantTr 	1	{ 91 }	Tarifa	560	Cant.	20\newline
Pr 442	CantTr 	1	{ 91 }	Tarifa	580	Cant.	21\newline
Pr 443	CantTr 	1	{ 91 }	Tarifa	590	Cant.	23\newline
Pr 444	CantTr 	1	{ 92 }	Tarifa	300	Cant.	20\newline
Pr 445	CantTr 	1	{ 92 }	Tarifa	320	Cant.	25\newline
Pr 446	CantTr 	1	{ 92 }	Tarifa	330	Cant.	25\newline
Pr 447	CantTr 	1	{ 92 }	Tarifa	350	Cant.	30\newline
Pr 448	CantTr 	1	{ 93 }	Tarifa	410	Cant.	30\newline
Pr 449	CantTr 	1	{ 93 }	Tarifa	430	Cant.	30\newline
Pr 450	CantTr 	1	{ 93 }	Tarifa	460	Cant.	20\newline
Pr 451	CantTr 	1	{ 93 }	Tarifa	470	Cant.	20\newline
Pr 452	CantTr 	1	{ 94 }	Tarifa	300	Cant.	20\newline
Pr 453	CantTr 	1	{ 94 }	Tarifa	330	Cant.	20\newline
Pr 454	CantTr 	1	{ 94 }	Tarifa	370	Cant.	25\newline
Pr 455	CantTr 	1	{ 94 }	Tarifa	390	Cant.	25\newline
Pr 456	CantTr 	1	{ 95 }	Tarifa	600	Cant.	30\newline
Pr 457	CantTr 	1	{ 95 }	Tarifa	610	Cant.	30\newline
Pr 458	CantTr 	1	{ 95 }	Tarifa	650	Cant.	30\newline
Pr 459	CantTr 	1	{ 95 }	Tarifa	680	Cant.	20\newline
Pr 460	CantTr 	1	{ 96 }	Tarifa	800	Cant.	21\newline
Pr 461	CantTr 	1	{ 96 }	Tarifa	820	Cant.	23\newline
Pr 462	CantTr 	1	{ 96 }	Tarifa	840	Cant.	20\newline
Pr 463	CantTr 	1	{ 96 }	Tarifa	860	Cant.	25\newline
Pr 464	CantTr 	1	{ 97 }	Tarifa	1900	Cant.	25\newline
Pr 465	CantTr 	1	{ 97 }	Tarifa	1960	Cant.	30\newline
Pr 466	CantTr 	1	{ 97 }	Tarifa	1980	Cant.	30\newline
Pr 467	CantTr 	1	{ 97 }	Tarifa	2000	Cant.	30\newline
Pr 468	CantTr 	1	{ 98 }	Tarifa	910	Cant.	20\newline
Pr 469	CantTr 	1	{ 98 }	Tarifa	950	Cant.	20\newline
Pr 470	CantTr 	1	{ 98 }	Tarifa	970	Cant.	20\newline
Pr 471	CantTr 	1	{ 98 }	Tarifa	990	Cant.	20\newline
Pr 472	CantTr 	1	{ 99 }	Tarifa	250	Cant.	25\newline
Pr 473	CantTr 	1	{ 99 }	Tarifa	270	Cant.	25\newline
Pr 474	CantTr 	1	{ 99 }	Tarifa	290	Cant.	30\newline
Pr 475	CantTr 	1	{ 99 }	Tarifa	300	Cant.	30\newline
Pr 476	CantTr 	1	{ 100 }	Tarifa	350	Cant.	30\newline
Pr 477	CantTr 	1	{ 100 }	Tarifa	360	Cant.	20\newline
Pr 478	CantTr 	1	{ 100 }	Tarifa	380	Cant.	21\newline
Pr 479	CantTr 	1	{ 100 }	Tarifa	390	Cant.	23\newline
Pr 480	CantTr 	1	{ 101 }	Tarifa	450	Cant.	20\newline
Pr 481	CantTr 	1	{ 101 }	Tarifa	460	Cant.	25\newline
Pr 482	CantTr 	1	{ 101 }	Tarifa	470	Cant.	25\newline
Pr 483	CantTr 	1	{ 101 }	Tarifa	480	Cant.	30\newline
Pr 484	CantTr 	1	{ 101 }	Tarifa	490	Cant.	30\newline
Pr 485	CantTr 	1	{ 102 }	Tarifa	600	Cant.	30\newline
Pr 486	CantTr 	1	{ 102 }	Tarifa	630	Cant.	20\newline
Pr 487	CantTr 	1	{ 102 }	Tarifa	660	Cant.	20\newline
Pr 488	CantTr 	1	{ 102 }	Tarifa	680	Cant.	20\newline
Pr 489	CantTr 	1	{ 102 }	Tarifa	690	Cant.	20\newline
Pr 490	CantTr 	1	{ 103 }	Tarifa	300	Cant.	25\newline
Pr 491	CantTr 	1	{ 103 }	Tarifa	330	Cant.	25\newline
Pr 492	CantTr 	1	{ 103 }	Tarifa	350	Cant.	30\newline
Pr 493	CantTr 	1	{ 103 }	Tarifa	380	Cant.	30\newline
Pr 494	CantTr 	1	{ 104 }	Tarifa	329	Cant.	20\newline
Pr 495	CantTr 	1	{ 104 }	Tarifa	410	Cant.	25\newline
Pr 496	CantTr 	1	{ 104 }	Tarifa	600	Cant.	25\newline
Pr 497	CantTr 	1	{ 104 }	Tarifa	610	Cant.	30\newline
Pr 498	CantTr 	2	{ 85 51 }	Tarifa	1000	Cant.	85\newline
Pr 499	CantTr 	2	{ 85 51 }	Tarifa	1050	Cant.	21\newline
Pr 500	CantTr 	2	{ 85 51 }	Tarifa	1150	Cant.	28\newline
Pr 501	CantTr 	2	{ 85 51 }	Tarifa	1200	Cant.	45\newline
Pr 502	CantTr 	2	{ 85 51 }	Tarifa	1250	Cant.	73\newline
Pr 503	CantTr 	3	{ 85 51 79 }	Tarifa	1900	Cant.	54\newline
Pr 504	CantTr 	3	{ 85 51 79 }	Tarifa	1950	Cant.	20\newline
Pr 505	CantTr 	3	{ 85 51 79 }	Tarifa	1980	Cant.	23\newline
Pr 506	CantTr 	3	{ 85 51 79 }	Tarifa	2000	Cant.	35\newline
Pr 507	CantTr 	2	{ 104 22 }	Tarifa	700	Cant.	67\newline
Pr 508	CantTr 	2	{ 104 22 }	Tarifa	740	Cant.	20\newline
Pr 509	CantTr 	2	{ 104 22 }	Tarifa	780	Cant.	30\newline
Pr 510	CantTr 	2	{ 104 22 }	Tarifa	790	Cant.	20\newline
Pr 511	CantTr 	3	{ 104 22 51 }	Tarifa	1250	Cant.	20\newline
Pr 512	CantTr 	3	{ 104 22 51 }	Tarifa	1300	Cant.	20\newline
Pr 513	CantTr 	3	{ 104 22 51 }	Tarifa	1350	Cant.	62\newline
Pr 514	CantTr 	3	{ 104 22 51 }	Tarifa	1400	Cant.	20\newline
Pr 515	CantTr 	4	{ 104 22 51 79 }	Tarifa	2050	Cant.	20\newline
Pr 516	CantTr 	4	{ 104 22 51 79 }	Tarifa	2100	Cant.	20\newline
Pr 517	CantTr 	4	{ 104 22 51 79 }	Tarifa	2150	Cant.	25\newline
Pr 518	CantTr 	4	{ 104 22 51 79 }	Tarifa	2300	Cant.	25\newline
Pr 519	CantTr 	4	{ 104 22 51 79 }	Tarifa	2350	Cant.	30\newline
\newline
End Productos\newline
\end{scriptsize}


\begin{center}
	\begin{tabular}{|c|c|}
	\hline
	Cantidad de tramos & 104 \\
	\hline
	Cantidad de productos & 519 \\
	\hline
	\end{tabular}
\end{center}

%------------------------------------------------------------------------------------%

\subsection{Resumen}
La siguiente tabla va a mostrar el tiempo, en segundos, empleado por cada archivo de prueba para hallar el resultado \'optimo seg\'un el $\beta $ utilizado.

\begin{center}
	\begin{tabular}{|c|c|c|c|c|c|c|}
	\hline
	\#Tramos & \#Productos & $\beta=0.6$ & $\beta=0.8$ & $\beta=1.0$ & $\beta= 1.2$ & $\beta=1.4$\\
	\hline
	$7$ & $22$ & $0$ & $0.016$ & $0$ & $0.015$ & $0$\\
	\hline
	$11$ & $108$ & $0.516$ & $0.484$ & $0.703$ & $0.875$ & $0.938$ \\
	\hline
  $45$ & $166$ & $3.265$ & $3.391$ & $3.297$ & $3.281$ & $3.079$ \\
  \hline
  $50$ & $241$ & $9.703$ & $10.75$ & $11.297$ & $11.157$ & $10.203$ \\
  \hline
	$83$ & $396$ & $46.234$ & $47.203$ & $49.14$ & $48.469$ & $45.656$ \\
  \hline
	$104$ & $519$ & $94.921$ & $96.281$ & $99.5$ & $99.781$ & $96.829$ \\
  \hline
	\end{tabular}
\end{center}


\newpage
\section{Discusi\'on}

% PAUTAS: Se incluir� aqu� un an�lisis de los resultados obtenidos en la secci�n anterior (se
% analizara su validez, coherencia, etc). Deben analizarse como m�nimo los items pedidos en el
% enunciado. No es aceptable decir que "los recultados fueron los esperados", sin hacer clara 
% referencia a la te�ria a la cual se ajustan. Adem�s se deben mencionar los resultados
% interesantes y los casos "patol�gicos" encontrados.


No era necesario hacer un Spline general, hubiera sido m\'as eficiente en el contexto del problema que los valores de la variable independiente sean equidistantes.

Debimos hacer un Spline de frontera sujeta en ambos extremos para poder acelerar hasta el final de la pista, en lugar de ir desacelerando hasta llegar a $0$ en ese punto.


\newpage
\section{Conclusiones}

% PAUTAS: esta secci�n debe contener las conclusiones generales del trabajo. Se deben 
% mencionar las relaciones de la discuci�n sobre las que se tiene certeza, junto con 
% comentarios y observaciones generales aplicables a todo el proceso. Mencionar tambi�n 
% posibles extensiones a los m�todos, experimentos que hayan quedado pendientes, etc.

% TAREAS: conclusiones generales sobre certezas. Mejoras y cosas pendientes.

El valor del cual depende muy fuertemente el tiempo de ejecuci�n del algoritmo que resulve el problema planteado para una red el\'ctrica con las restricciones de Morley es la cantidad de ejes del grafo que representada dicha red. En un principio esto nos tendr\'ia que haber resultado bastante razonable, sin tener en cuenta ninguno de los resultados obtenidos, ya que los valores del flujo de cada uno de los ejes eran inicialmente nuestras inc\'ognitas. De esta manera la cantidad de ejes representa una buena medida de la complejidad del problema planteado.\\
Las instancias de prueba utilizadas son instancias poco densas. En general la relaci�n entre la cantidad de nodos $n$ y ejes $m$ del grafo que representa a la red se corresponde con $m$ aprox $3n$.\\
Hubiese sido muy interesante poder probar instancias con diferente cantidd de ejes para una misma cantidad de nodos. De esta manera hubi\'eramos podido notar m\'as claramente la influencia de la variaci�n en la densidad de un grafo con respecto al tiempo de ejecuci�n de cada una de las instancias. Llamamos densidad de un grafo (representaci\'on matem\'atica de la red el\'ectrica) a la cantidad de ejes que contiene dicho grafo, dividido la cantidad m\'axima posible de ese mismo grafo. Esto ser\'ia $m / (n*(n-1)/2)$.\\
La reducci\'on en el espacio de almacenamiento debido a la nueva representac\'on del grafo mediante una matriz para poder resolver el problema fue del $25$ porciento. Esta reducci\'on en el espacio de almacenamiento trajo aparejada una disminuci\'on en el tiempo de ejecuci\'on y la posible inclusi\'on de instancias de mayor tama�o para las pruebas realizadas.\\
El tiempo de ejecuci\'on, medido en segundos, en funci\'on de la cantidad de nodos o ejes, ya que ambas distribuciones pertenecen a una misma familia, se corresponden con una distribuci\'on exponencial.\\
No existe relaci\'on aparente a simple vista entre el tiempo de ejecuci\'on y el n\'umero de condici\'on de la matriz. Como nosotros sabemos el n\'umero de condici\'on nos otorga una medida de confiabilidad sobre la soluci\'on encontrada al sistema de ecuaciones planteado para obtener la soluc\'on del problema de flujos sobre las interconexiones de una red el\'ectrica.\\




\newpage
\section{Ap�ndice A}

\subsection{Enunciado}

\begin{centering}
\bf Laboratorio de M\'etodos Num\'ericos - Primer cuatrimestre 2007 \\
\bf Trabajo Pr\'actico N\'umero 1: Sin margen de error (num\'erico) \\
\end{centering}

\vskip 25pt
\hrule
\vskip 11pt

El objetivo del trabajo pr\'actico es analizar el comportamiento de la
aritm\'etica de punto flotante para el c\'alculo de una funci\'on sobre
el conjunto de los n\'umeros reales. Consideremos la funci\'on
$G:\real\to\real$ definida por:
\begin{eqnarray}
T(x) & = & \left\{ \begin{array}{cl}
                      1 & \hbox{si } x = 0 \\
                      \frac{e^x-1}{x} & \hbox{en caso contrario}
                    \end{array} \right. \nonumber \\
Q(x) & = & \Big|x - \sqrt{x^2+1} \Big| - \frac{1}{x+\sqrt{x^2+1}} \nonumber \\
G(x) & = & T( Q(x)^2 ) \nonumber
\end{eqnarray}
En este trabajo pr\'actico se pide realizar un an\'alisis emp\'\i rico
del comportamiento de la funci\'on $G(x)$ para distintos valores de $x$,
calculada con aritm\'etica de punto flotante. Para esto, se pide implementar
el c\'alculo de esta sucesi\'on en aritm\'etica de $t$ d\'\i gitos binarios
de precisi\'on y, sobre la base de la implementaci\'on, realizar las
siguientes mediciones:
\begin{enumerate}
\item Graficar el valor de $G(x)$ en funci\'on de $x$ para diferentes
valores de $t$. >Se observa alguna regularidad?
\item Medir el valor de $G(x)$ en funci\'on de la cantidad $t$ de d\'\i gitos
binarios de precisi\'on en la aritm\'etica de punto flotante, para diferentes
valores de $x$. >Se observa alguna influencia de la cantidad de d\'\i gitos
de precisi\'on en el c\'alculo de $G(x)$?
\end{enumerate}

>Es posible obtener el valor de $G(x)$ anal\'\i ticamente? En caso afirmativo,
>c\'omo se compara este valor exacto con la aproximaci\'on lograda con
aritm\'etica de punto flotante? En caso de que se observen diferencias
importantes, analizar las causas que generan este comportamiento.

El informe debe contener una descripci\'on detallada de las distintas
alternativas que el grupo haya considerado para la implementaci\'on de la
aritm\'etica de punto flotante de $t$ d\'\i gitos binarios de precisi\'on, junto
con una discusi\'on de estas alternativas que justifique la opci\'on
implementada. Por otra parte, se debe incluir en la secci\'on correspondiente
el c\'odigo que implementa esta aritm\'etica, junto con todos los comentarios
y decisiones relevantes acerca de esta implementaci\'on.

\vskip 15pt

\hrule

\vskip 11pt

Fecha de entrega: Lunes 9 de Abril



\newpage
\section{Ap�ndice B}

\subsection{Funciones Relevantes}

\begin{scriptsize} 
Matriz spline(Matriz tablaValores, long double derivPrimerPunto) {
    // Planteo
    // 1) crear sistema de ecuaciones para obtener los valores c_j
    // 2) resolver (siempre puede resolverse en la teor�a, o hubo error de datos
    //    de entrada)
    // 3) con los valores despejados de c_j y los a_j obtenidos al principio
    //    despejar b_j y d_j

    // 1) crear sistema para obtener c_j
    // El sistema a plantear tiene esta forma:
    // - una fila 2*h_0*c_0 + h_0*c_1 = val(0) donde val(0) es la expresi�n
    //   val(0) = 3(a_1 - a_0)/h_0 - 3*derivPrimerPunto
    // - ecuaciones h_{j-1}*c_{j-1} + 2(h_{j-1}+h_{j})*c_j + h_{j}*c_{j+1} = val(j)
    // para j entre 1 y n-1 donde val(j) es 3(a_{j+1}-a_j)/h_j - 3(a_j - a_{j-1})/h_{j-1}
    // - una fila 0 0 ... 0 1 con valor de t�rmino independiente 0
    // Esta matriz es diagonal dominante - el sistema siempre tiene soluci�n
    // Este spline es de frontera sujeta en el primer segmento, y de frontera
    // libre en el �ltimo.

    assert(tablaValores.getCol() == 2);
    assert(tablaValores.getFil() >= 3); // al menos 3 filas, o no puede hacerse un spline!

    // Definici�n: n es la cantidad de segmentos entre puntos, y la cantidad de
    // ecuaciones de grado 3 que deben generarse. La tabla de valores de entrada
    // proporciona entonces n+1 puntos
    unsigned int n = tablaValores.getFil() - 1;

    // Crear las matrices para el sistema de ecuaciones
    Matriz A(n+1, n+1); // coeficientes de las variables
    Matriz b(n+1, 1); // t�rminos independientes

    // Completar las matrices del sistema lineal

	// Asignar la primera ecuaci�n
	// Frontera libre
	// A.sub(1, 1) = 1.0; // c_0 == 0 (frontera libre)
	// Frontera sujeta en el primer segmento del spline
	// - calcular h0 para la ecuaci�n en la primera fila

    // Obtener h_0 = x_{0+1} - x_0
    long double hjmas1 = tablaValores.sub(2, 1) - tablaValores.sub(1, 1); // x_1 - x_0
    assert(hjmas1 > 0); // pedimos que los x_j est�n en orden ascendente y no se repitan (!)
    // Obtener a_j - a_{j-1} con j=1  ---------> obtener a_1 - a_0
    long double Dajmas1 = tablaValores.sub(2, 2) - tablaValores.sub(1, 2);
    // Obtener 3(a_j - a_{j-1}) / h_{j-1} con j=1  ---------> obtener 3(a_1 - a_0) / h_0
    long double t3Dajmas1Dhjmas1 = (3.0*Dajmas1)/hjmas1;

	A.sub(1, 1) = 2.0*hjmas1;
	A.sub(1, 2) = hjmas1;
	b.sub(1, 1) = t3Dajmas1Dhjmas1 - 3.0*derivPrimerPunto;

	// Obtener 3(a_{j+1} - a_j) / h_j - 3(a_j - a_{j-1}) / h_{j-1} para j>=1
    for (int j = 1; j<=n-1; j++) {
        // Guardar los valores previos para esta iteraci�n
        long double prevhj = hjmas1;
        long double prevDaj = Dajmas1;
        long double prev3DajDhj = t3Dajmas1Dhjmas1;
        // Calcular nuevos valores
        hjmas1 = tablaValores.sub(j+1+1, 1) - tablaValores.sub(j+1, 1); // x_{j+1} - x_j
        assert(hjmas1 > 0); // pedimos que los x_j est�n en orden ascendente y no se repitan (!)
        Dajmas1 = tablaValores.sub(j+1+1, 2) - tablaValores.sub(j+1, 2);
        t3Dajmas1Dhjmas1 = (3.0*Dajmas1)/hjmas1;
        // Guardarlo en el vector de t�rminos independientes
        b.sub(j+1, 1) = t3Dajmas1Dhjmas1 - prev3DajDhj;
        // Guardar en la matriz asociada del sistema
        A.sub(j+1, j) = prevhj; // h_j
        A.sub(j+1, j+1) = 2.0*(prevhj+hjmas1); // 2(h_j + h_{j+1})
        A.sub(j+1, j+2) = hjmas1; // h_{j+1}
    }

	// Asignar la �ltima ecuaci�n
	// Tenemos un spline con frontera sujeta en el �ltimo segmento
	A.sub(n+1, n+1) = 1.0; // c_{n+1} == 0 (idem)
	b.sub(n+1, 1) = 0.0;
	// Resolver el sistema
	assert(A.triangular(b));
	A.despejar(b);
	// No necesito m�s la matriz A - la convierto en la matriz resultado
    // Crear la matriz con los coeficientes de cada polinomio
	A.resize(n, 5); // Devolver una matriz de n filas por 5 columnas

    for (int jmas1 = 1; jmas1 <= n; jmas1++) {
		// Asignar los valores de x_j
		A.sub(jmas1, 1) = tablaValores.sub(jmas1, 1); // definici�n de x_j
		// Asignar los valores de a_j
		// Estos valores son sencillamente f(x_j) para j entre 0 y n-1
		A.sub(jmas1, 2) = tablaValores.sub(jmas1, 2); // definici�n de a_j
		// Asignar los valores de b_j
		// Estos valores son:
		// b_j = (a_{j+1} - a_j)/h_j - h_j*((2/3)*c_j + c_{j+1}/3) para j entre 0 y n-1
		long double hj = tablaValores.sub(jmas1+1, 1) - tablaValores.sub(jmas1, 1);
		A.sub(jmas1, 3) = (tablaValores.sub(jmas1+1, 2) - tablaValores.sub(jmas1, 2))/hj;
		A.sub(jmas1, 3) -= hj*(2.0*b.sub(jmas1, 1) + b.sub(jmas1+1, 1))/3.0;
		// Asignar los valores de c_j
		// Estos valores son las inc�gnitas del sistema resuelto, y est�n en b
		A.sub(jmas1, 4) = b.sub(jmas1, 1);
		// Asignar los valores de d_j
		// d_j = (c_{j+1} - c_j) / (3.0*h_j)
		A.sub(jmas1, 5) = (b.sub(jmas1+1, 1) - b.sub(jmas1, 1)) / (3.0*hj);
	}
return A;
}
\end{scriptsize} 
 \newpage 
\begin{scriptsize} 
 bool chequearPositivoBisec(long double (*func)(long double), long double (*deriv)(long double), long double a, long double b, const long double& tolBiseccion, long double& valorNegativo, std::ostream* pOutput) {
        if (a >= b)
            return true; // intervalo incorrecto - informar que la funci�n se mantiene positiva

        long double valF = func(a);
        if (valF < 0) {
            valorNegativo = a;
            return false; // encontramos un negativo
        }

        valF = func(b);
        if (valF < 0) {
            valorNegativo = b;
            return false; // encontramos un negativo
        }

        long double valDerivA = deriv(a);
        long double valDerivB = deriv(b);
        int signDerivA = sign(valDerivA);
        int signDerivB = sign(valDerivB);
        if (signDerivA == 0) {
            a = salirDelCero(deriv, a, tolBiseccion); // omitir el punto cr�tico en la pr�xima iteraci�n
        }
        if (signDerivB == 0) {
            b = salirDelCero(deriv, b, -tolBiseccion); // omitir el punto cr�tico en la pr�xima iteraci�n

        }
        // Iterar sobre un intervalo que excluya a el o los puntos cr�ticos en los
        // bordes anteriores del intervalo
        if (signDerivA == 0 || signDerivB == 0)
            return chequearPositivoBisec(func, deriv, a, b, tolBiseccion, valorNegativo, pOutput);

        if (signDerivA != signDerivB) {
            // ra�z de la derivada en el medio! Usar bisecci�n para encontrarla
            long double nuevoa = a;
            long double nuevob = b;
            long double aproxRaiz = biseccion(deriv, nuevoa, nuevob, tolBiseccion);
            cout << "Encontre una raiz de la derivada en " << aproxRaiz << endl;
            *pOutput << "Encontre una raiz de la derivada en " << aproxRaiz << endl;
            // chequear si la funcion se hace negativa
            long double valF = func(aproxRaiz);
            if (valF < 0) {
                cout << "Encontr� un valor negativo de la funci�n en " << aproxRaiz << endl;
								*pOutput << "Encontr� un valor negativo de la funci�n en " << aproxRaiz << endl;
                valorNegativo = aproxRaiz;
                //return false; // encontramos el negativo
            }
            // buscar recursivamente en [a, nuevoa] y [nuevob, b]
            bool rv = chequearPositivoBisec(func, deriv, a, nuevoa, tolBiseccion,valorNegativo,pOutput);
            if (rv) {
                rv = chequearPositivoBisec(func, deriv, nuevob, b, tolBiseccion, valorNegativo,pOutput);
            }
            return rv;
        }
        else {
            // subdividir el intervalo [a,b]
            long double ancho = (b - a)/2.0;
            if (ancho < tolBiseccion)
                return true; // el intervalo ya era demasiado chico para subdividirlo
            long double medio = a + ancho;
            // buscar recursivamente en [a, medio] y [medio, b]
            bool rv = chequearPositivoBisec(func, deriv, a, medio, tolBiseccion, valorNegativo,pOutput);
            if (rv) {
                rv = chequearPositivoBisec(func, deriv, medio, b, tolBiseccion, valorNegativo, pOutput);
            }
            return rv;
        }
    }

\end{scriptsize} 
 \newpage 
\begin{scriptsize} 

long double biseccion(long double (*func)(long double), long double& a, long double& b, long double tolBiseccion) {
        long double valA = func(a);
        int signA = sign(valA);
        if (signA == 0) {
            // Encontr� la ra�z: es a
            return a;
        }

        while (true) {
            long double mitad = (b - a)/2.0;
            if (mitad < tolBiseccion) {
                return a + mitad;
            }
            long double p = a + mitad;
            long double valP = func(p);
            int signP = sign(valP);
            if (signA == signP)
                a = p;
            else
                b = p;
        }
    }
\end{scriptsize} 


\begin{scriptsize} 

 long double evalH(long double t) {
        return (datos.masa*G + evalV2(t)*datos.coef_carga_dinam)*datos.coef_rozamiento - datos.masa*evalA(t);
    }

\end{scriptsize} 

\begin{scriptsize} 
long double evalDiffH(long double t) {
        return (2.0*evalDiffV2(t)*datos.coef_carga_dinam*datos.coef_rozamiento - datos.masa*evalDiffA(t));
    }
\end{scriptsize} 
 \newpage 
\begin{scriptsize} 

	void calcular(ofstream& salida) {
		datos.spline_x = spline(datos.matriz_x, datos.dir_inicial_x);
		datos.spline_y = spline(datos.matriz_y, datos.dir_inicial_y);
		pOut = &salida;
    // Buscar ceros de la derivada y evaluar la funci�n en los puntos cr�ticos
		// Tambi�n verificar el signo de la funci�n H en los extremos del intervalo
    long double tolBiseccion = 0.0001;
    long double a, b, valorNegativo = 0.0;
  	bool esPositivo = true;
  
    b = datos.matriz_x.sub(1, 1);
    for (int i = 2; i <= datos.matriz_x.getFil(); i++) {
			a = b;
			b = datos.matriz_x.sub(i, 1);
			esPositivo = calc::chequearPositivoBisec(evalH, evalDiffH, a, b, tolBiseccion, valorNegativo, pOut);
			if (!esPositivo)
		    	break;
		}
		if (!esPositivo) {
			cout << "SE PUDRE TODO en t = " << valorNegativo << endl;
			cout << "Ese punto cae dentro del segmento numero " << valorNegativo / datos.omega + 1.0 << endl;
			salida << "SE PUDRE TODO en t = " << valorNegativo << endl;
			salida << "Ese punto cae dentro del segmento numero " << valorNegativo / datos.omega + 1.0 << endl;
			// Sugerir un punto
			DatosTP4 orig = datos;
			int problema = (int) ( 1.1 + valorNegativo / datos.omega );

			datos.cant_muestras = problema + 1;
			// Definir una matriz de problema + 1 puntos, los primeros problema - 1 son iguales
			Matriz mX(problema + 1, 2);
			Matriz mY(problema + 1, 2);
			for (int i = 1; i <= problema + 1; i++) {
				mX.sub(i, 1) = mY.sub(i, 1) = datos.omega * (i-1);
				if (i < problema) {
					mX.sub(i, 2) = datos.matriz_x.sub(i, 2);
					mY.sub(i, 2) = datos.matriz_y.sub(i, 2);
				}
			}
			datos.matriz_x = mX;
			datos.matriz_y = mY;
			// Hay que calcular los puntos problema y problema + 1
			// 1era condici�n: ambos sobre la trayectoria original
			// 2da condici�n: el punto problema debe hacer que la funci�n H no reviente
			// 3ra condici�n: el punto problema + 1 debe hacer que el vector aceleraci�n en
			// el punto problema no tenga mucho valor absoluto
			datos = orig;
		}
        mostrarGraficos();
 }
\end{scriptsize}


\newpage
\section{Instancias de pruebas para el M\'etodo Simplex Revisado}

\subsection{Casos de prueba del M\'etodo Simplex Revisado}

Estos casos de prueba fueron utilizados para probar el funcionamiento de del algoritmo independientemente del modelo utilizado para representar el problema.

Para ver los ejemplos con los cuales fue probado el M\'etodo Simplex Revisado ver el archivo testSimplexCHVATAL.cpp

\subsection{Unica Soluci\'on (acotado)}

\subsection{M\'as de una soluci\'on (no acotado)}

Tener en cuenta que dado que el vector b es siempre positivo para los casos 
de estudio de nuestro TP tenemos solo problemas factibles. Con una o varias
soluciones:
              
Todos los ejemplos que utilizamos son factibles con una unica soluci�n.
               
Aqui hay un ejemplo para probar que tiene mas de una solucion:
 
$maximizar   x1 + 3x2 - x3      $
$sujeto a   2x1 + 2x2 - x3 <= 10$
$           3x1 - 2x2 + x3 <= 10$
$            x1 - 3x2 + x3 <= 10$
                    
              $  x1,x2,x3 >= 0  $
 
Soluci�n: probar con $ x1 = 0 ; x2 = 5+0.5t ; x3 = t.$

\newpage
\section{Referencias}

\begin{itemize}

\item Am\'alisis Num\'erico de Richard L. Burden y J. Douglas Faires.
\item V. Chvatal, Linear programming, Freeman, 1983.

\end{itemize}

\end{document}


